	% Small command to replace the thesis title wherever it is used
	\pagestyle{empty}
	\pagenumbering{none}

	\begin{center}
		\begin{figure}[h]
			\centering
				\includegraphics[width=4cm]
				{fig/EscudoUniversidadColorSombraBisel.png}
		\end{figure}
		{\LARGE Universidad de Granada\\}
		\vspace{0.2cm}
		{\large Departamento de Teoría de la Señal,\\
		        Telemática y Telecomunicaciones\\}
		
		\vfill
		
		{\LARGE\textbf{\thesistitle}\\}
		
		\vfill
		
		{\Large \thesisauthor\\
		{\large Departamento de Astronomía Extragaláctica,\\
		Instituto de Astrofísica de Andalucía-CSIC}}
		
		\vfill
		
		\begin{minipage}{3cm}
			\begin{center}
				\includegraphics[width=3cm]
				{fig/LogoIAA.pdf}
			\end{center}
		\end{minipage}
		\begin{minipage}{6cm}
			\begin{center}
				\vfill
				{\Large Tesis Doctoral}
				\vfill
			\end{center}
		\end{minipage}
		\begin{minipage}{3cm}
			\begin{center}
				\includegraphics[width=2.5cm]
				{fig/LogoTSTC.pdf}
			\end{center}
		\end{minipage}
	\end{center}

	\cleardoublepage

	\begin{center}
		{\Large Departamento de Teoría de la Señal,\\
		        Telemática y Telecomunicaciones\\}
		\vspace{0.2cm}
		{\large Universidad de Granada\\}
		\vfill
		{\Large Departamento de Astronomía Extragaláctica\\}
		\vspace{0.2cm}
		{\large Instituto de Astrofísica de Andalucía - CSIC\\}
		\vfill
		{\LARGE\textbf{\thesistitle}\\}
		\vfill
		{\large Memoria presentada por:\\
				\vspace{0.5cm}
		        {\Large \thesisauthor}\\
		\vspace{0.5cm}
		para optar al grado de\\
		\vspace{0.5cm}
		{\Large Doctor por la Universidad de Granada}
		}
		
		\vfill
		{\large Dirigida por:\\
				\vspace{0.5cm}
				Lourdes Verdes-Montenegro Atalaya (IAA-CSIC)\\
				Enrique Solano Márquez (LAEX-CAB/INTA-CSIC)}
		\vfill
	\end{center}

	\vfill

	\begin{center}
		{\large Granada, \fechaDeposito}
	\end{center}
	
	\vfill

	\cleardoublepage

	\noindent Como directores de la tesis titulada \emph{\thesistitle},
	presentada por \textbf{D.~\thesisauthor},

	\vspace{0.25cm}

	\begin{adjustwidth}{1cm}{2cm}
		\noindent \textbf{Dña.~Lourdes Verdes-Montenegro Atalaya},
		Doctora en Ciencias Físicas y Científico Titular del
		Departamento de Astronomía Extragaláctica del \IAA, y
		\textbf{D.~Enrique Solano Márquez}, Doctor en Ciencias
		Matemáticas \invisiblenote{y Científico Titular} del
		Laboratorio de Astrofísica Estelar y Exoplanetas del
		Centro de Astrobiología (LAEX-CAB/INTA-CSIC).
	\end{adjustwidth}

	\vspace{2cm}

	\begin{adjustwidth}{2cm}{0cm}
		\noindent \textsc{Declaran:\\}
	
		\vspace{0.5cm}
		
		\noindent Que la presente memoria, titulada
		\emph{\thesistitle} ha sido realizada por
		\textbf{D.~\thesisauthor} bajo su dirección en el \IAA.
		Esta memoria constituye la tesis que
		\textbf{D.~\thesisauthor} presenta para optar al grado de
		\textbf{Doctor por la Universidad de Granada}.
		
		\begin{flushright}
			Granada, a \fechaDeposito\\
		\end{flushright}
	\end{adjustwidth}
	
	\vspace{3cm}
	
	\begin{minipage}{6.5cm}
		\begin{center}
			{\small Fdo:\\
			Lourdes Verdes-Montenegro Atalaya}
		\end{center}
	\end{minipage}
	\begin{minipage}{6.5cm}
		\begin{center}
			{\small Fdo:\\
			Enrique Solano Márquez}
		\end{center}
	\end{minipage}
	
	\cleardoublepage
	
	\vspace{5cm}
	\noindent \textbf{\thesisauthor},  autor de la tesis  
	\emph{\thesistitle}, autoriza a que un ejemplar de la misma
	quede ubicada en la Biblioteca de la Escuela Superior de
	Ingeniería Informática de Granada.
	\vspace{5cm}
	\begin{center}
		Fdo.:~\thesisauthor\\
		Granada, a \fechaDeposito
	\end{center}
	
	\cleardoublepage
	
	\vspace{5cm}
	\begin{adjustwidth}{0.6\textwidth}{0cm}
		\begin{flushright}
			A mi abuela, que nunca creyó vivir para ver este momento. 
			Y a quienes siempre han estado a mi lado, incluso cuando 
			menos me lo merecía: sé que siempre, de una forma u otra, 
			podré contar con vosotros pase lo que pase.
		\end{flushright}
	\end{adjustwidth}

	\cleardoublepage

	\vspace{5cm}
	\begin{adjustwidth}{0.075\textwidth}{0.075\textwidth}
	
		\emph{Desde que orbitaron los primeros satélites, hacía unos
        cincuenta años, billones y cuatrillones de impulsos de
        información habían estado llegando del espacio, para ser
        almacenados para el día en que pudieran contribuir al avance
        del conocimiento. Sólo una minúscula fracción de esa materia
        prima sería tratada; pero no había manera de decir qué
        observación podría desear consultar algún científico, dentro de
        diez, o de cincuenta, o de cien años. […] Formaban parte del
        auténtico tesoro de la Humanidad, más valioso que todo el oro
        encerrado inútilmente en los sótanos de los bancos.}
		
		\vspace{1.5\baselineskip} % like 1 1/2 Line Feeds
	
		\begin{flushright}
			Arthur C. Clarke (1917-2008),\\ 
			\emph{2001: Una Odisea Espacial} (1968).
		\end{flushright}
	\end{adjustwidth}

	\cleardoublepage

%%	% Start roman numbering

	\pagestyle{headings}
	\pagenumbering{roman}
	\tableofcontents* % we don't want the ToC to be part of itself

	\cleardoublepage

	\listoffigures

	\cleardoublepage

	\listoftables

	\cleardoublepage

	\lstlistoflistings
	
	\cleardoublepage