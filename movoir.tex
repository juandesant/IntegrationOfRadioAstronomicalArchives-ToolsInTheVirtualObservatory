% chapter movoir (fold)
\chapter[MOVOIR: MOdular VO InteRface]
		[The MOVOIR]
		{MOVOIR:
         MOdular Virtual Observatory InteRface, and VO APIs}
\label{cha:movoir}

	\attributedquote{
		\dictionarydef
		{modular}
		{adjective}
		{
			\begin{itemize}
				\item employing or involving a module or modules
				as the basis of design or construction.
				\invisiblenote{: \emph{modular housing units}.} 
			\end{itemize}
		}
		\dictionarydef
		{module}
		{noun}
		{
			\begin{itemize}
				\item each of a set of standardized parts or 
				independent units that can be used to 
				construct a more complex structure.
				
				\item \textsf{Computing} any of a number of distinct
				but interrelated units from which a 
				program may be built up or into which a complex 
				activity may be analyzed.
			\end{itemize}
		}
		\dictionarydef
		{interface}
		{noun}
		{
			%\begin{enumerate}
			%	\item
				\begin{itemize}
					\item a point where two systems, subjects,
					organizations, etc., meet and interact.
					\invisiblenote{:
					\emph{the interface between accountancy
					and the law}.}
					
					\invisiblenote
					{\item \textsf{chiefly Physics} a surface
					forming a common boundary between two portions
					of matter or space, e.g., between two
					immiscible liquids:
					\emph{the surface tension of a liquid at 
					its air/liquid interface}.} 
			%	\end{itemize}
				
			%	\item 
			%	\begin{itemize}
					\invisiblenote{
					\item a device or program
					enabling a user to communicate with a computer.}
					
					\item \textsf{Computing}
					a device or program for connecting two
					items of hardware or software so that 
					they can be operated jointly or communicate
					with each other.
				\end{itemize} 
			%\end{enumerate}
		}
	}
	{The New American English Dictionary, \emph{2nd Edition}}
	
	\invisiblenote{
	The MOVOIR (acronym for MOdular Virtual Observatory InteRface)
	}
	
	In the previous chapter we have seen that by creating
	applications which support VO messaging, using the SAMP
	messaging protocol, the application can be completely
	VO-enabled as long as there are outside modules performing
	certain functions.
	
	 In this chapter, we will show a set of such modules, which we
	call MOVOIR (as acronym for MOdular Virtual Observatory
	InteRface). We will see which are the messages used by already
	existing applications, and how can we create our own messages
	for supporting an interface to the VO, but also the messages
	defined by the existing applications can be rethought so that
	the support special services, defining that way a plug-in API
	for the VO.
	
	 An additional remark: in this, and the following chapters,
	the meaning of the words “MUST”, “MUST NOT”,
	“REQUIRED”, “SHALL”, “SHALL  NOT”, “SHOULD”, “SHOULD NOT”,
	“RECOMMENDED”, “MAY”, and “OPTIONAL”, when they appear in
	capital letters, are to be interpreted as described by
	Internet Engineering Task Force (IETF)
	RFC~2119\urlnote{http://www.ietf.org/rfc/rfc2119.txt}~\cite{
	rfc2119}.
	
	\section{SAMP Messages and MTypes} % (fold)
	\label{sec:samp_messages_mtypes}
		
		SAMP is a hub-based messaging system which supports many
		different mechanisms for messaging: message broadcasting,
		point-to-point messaging, and a publish/subscribe scheme
		where messages are sent to all interested applications,
		which have declared such interest when registering with
		the hub.
		
		All of these operations are performed by sending MTypes,
		which are messages with particular codes, so that they
		have an associated meaning. In the SAMP standard profile,
		they correspond to XML-RPC calls to the hub to the
		corresponding \method{call}, \method{callAndWait}, or
		\method{notify} methods, provided as XML-RPC services by
		the hub. In those calls, there is a message parameter
		which correspond to a map (in the sense of a set of
		key-value pairs, where the keys are strings) with the
		following keys:
		
		\begin{description}
			\item[\sampmtype] A \stringtype{} which
			defines the meaning and parameters of the message.
			All messages sent with the same \sampmtype{}
			need to provide the same mandatory parameters, with
			the same data types, with the same expected behaviour.
			
			\item[\sampparams] A \maptype{} of
			the parameters needed for the correct interpretation
			of a message with the specified \sampmtype.
			As mentioned above, when an MType is defined, their
			mandatory parameters have to be defined, too.
			Keys are \stringtype{}s representing the parameter
			name, and the data type of the value depends on the
			actual definition of the MType, but has to be one of
			the supported data types.
		\end{description}
		
		\begin{table}[t]
		\begin{minipage}{\linewidth}
			\caption[SAMP data types]
			{SAMP Data Types. The corresponding Backus-Naur Form
			(BNF)
			for each data type can be found on section 3.3 of the
			\emph{SAMP IVOA Recommendation}~\cite{2009samp.ivoav0904T}.
			Data range for \sampint{} or \sampfloat{} types depends
			on the encoding and decoding applications.}
			\label{tabSAMPdataTypes}
		\begin{center}
		\begin{smalltabular}{rp{0.5\textwidth}}
			\textbf{Data type} & \textbf{Description} \\ \midrule
		
			\stringtype{} & alphanumeric data.\\ \addlinespace
			
			\listtype{} & ordered array of data
			items of the same type.\\ \addlinespace
			
			\sampint{} & a \stringtype{} containing a
			representation of an integer number.\\ \addlinespace
			
			\sampfloat{} & a \stringtype{} containing a
			representation of a floating point number.\\
			\addlinespace
			
			\sampbool{} & a \stringtype{} containing either
			\datatype{0} or \datatype{1}, for false or true logical
			values, respectively.\\ \addlinespace
			
			\maptype{} & an unordered associative
			array of key-value pairs, in which each key is a
			\stringtype, and each value is given in one of
			the supported data types above.\\ \addlinespace
		\end{smalltabular}
		\end{center}
		\end{minipage}
		\end{table}
		
		When a message of a given type returns values, they are
		returned as map with the following keys:
		
		\begin{description}
			\item[\sampstatus] This is a REQUIRED key.
			Its value is a \stringtype{} summarising the result
			of the processing. It may take one of the following
			predefined values:
			
			\begin{description}
				\item[\sampok] Signals total
				success. In this case, the \sampresult{} key
				SHOULD be present, and the \samperror{} key
				SHOULD NOT appear.
				
				\item[\sampwarning] Partial success. Both
				\sampresult{} and \samperror{} keys
				SHOULD be present.
				
				\item[\samperror] Processing of the
				message failed. The \samperror{} key MUST
				be present, and the \sampresult{} MUST NOT
				appear.
			\end{description}
			
			\item[\sampresult] This key is REQUIRED in case
			of full (\sampstatus\ equal to \sampok) or partial
			(\sampstatus\ equal to \sampwarning) success. The
			value is a map containing the values for the
			named return values, which are determined by the
			value of \sampmtype{} (the MType). Even for MTypes
			which return no value, the key must be present,
			with its value set to empty.
			
			\item[\samperror] This key is REQUIRED in case
			of full (\sampstatus\ equal to \samperror) or
			partial (\sampstatus\ equal to \sampwarning) error.
			The value for this key is a map with the following
			keys:
			
			\begin{description}
				\item[\mapkey{samp.errortxt}] This key is
				REQUIRED in this map. Its value is short
				\stringtype{} describing the problem, to be
				presented to the user.
				
				
				\item[\mapkey{samp.usertxt}] This key is
				optional, and its value is a free-form
				\stringtype, with additional text the
				called application wishes to append to the error.
				It could be appended to the \mapkey{samp.errortxt},
				but it is undefined what to do with it.
				
				\item[\mapkey{samp.debugtxt}] This key is optional,
				and its value is a free-form \stringtype{} of interest
				for debugging purposes (e.g. a stack trace).
				
				\item[\mapkey{samp.code}] This key is optional, and
				its value is a \stringtype{} containing a code
				(numeric or textual) identifying the error.
			\end{description}
		\end{description}
		
		In order to enhance interoperability, SAMP data types
		are specified as encoded strings, instead of having
		a binary encoding, or using platform specific types such
		as native XML-RPC \mapkey{int} or \mapkey{float} types.
		Allowed data types are shown in table~\ref{tabSAMPdataTypes}.
		
	% section samp_messages_mtypes (end)
	
	\section{Standard SAMP message types (MTypes)} % (fold)
	\label{sec:already_defined_mtypes}
		
		\begin{figure}[tb]
			\centering
				\begin{minipage}{0.85\linewidth}
				\begin{framed}
					\begin{small}
						\mtypedef{String encoding an MType in
						\sampmtype}
								{
									Description of the meaning of 
									the message, including the
									expected behaviour of receiving
									applications.
								}
								{
									\mtypeparam{parameter name}
									{data type}
									{
										An entry describing every
										allowed \mapkey{parameter
										name}
										in \sampparams.
										All parameters are mandatory,
										unless otherwise stated.
									}
								}
								{
									\mtypeparam{parameter name}
									{data type}
									{
										\emph{None}, if nothing is
										returned, or one entry for
										each named
										returned parameter supported
										in \sampparams, 
										describing it.
									}
								}
					\end{small}
				\end{framed}
				\end{minipage}
				
			\caption{Format for describing MTypes.}
			\label{fig:sampMTypeDescFormat}
		\end{figure}
		
		As SAMP is an evolution of the PLASTIC messaging protocol,
		there have been defined some MTypes which represent the
		kind of messages, with their parameters, that PLASTIC
		applications were capable of sending.
		
		\newcommand{\sampmtypesurl}[0]
		{http://www.ivoa.net/cgi-bin/twiki/bin/view/IVOA/SampMTypes}
		The Applications Working Group of the IVOA has created a
		wiki page\urlnote{\sampmtypesurl} for declaring the
		MTypes being publicly supported by different applications,
		so that applications can open up for ad-hoc messages. Once
		this thesis is published, the MTypes for the MOVOIR will be
		incorporated to this page.
		
		In particular, only the following MTypes are officially
		supported by clients such as TOPCAT and Aladin, and
		maintained by the IVOA Applications WG:
		\method{ta\-ble.load.vo\-ta\-ble}, \method{ta\-ble.load.fits},
		\method{ta\-ble.high\-light.row},
		\method{ta\-ble.se\-lect.row\-List},
		\method{im\-age.load.fits},
		\method{coord.point\-At.sky}, and
		\method{spec\-trum.load. ssa-generic}. They are shown,
		following the format of
		figure~\ref{fig:sampMTypeDescFormat}, in
		figures~\ref{fig:tableLoadVotableMtype},
		\ref{fig:tableLoadFitsMtype},
		\ref{fig:tableHighlightRowMtype},
		\ref{fig:tableSelectRowListMtype},
		\ref{fig:imageLoadFitsMtype},
		\ref{fig:coordPointAtSkyMtype}, and
		\ref{fig:spectrumLoadSsaGenericMtype}.
		
		\begin{figure}[tbp]
			\centering
			\begin{minipage}{0.9\textwidth}
				\begin{framed}
					\mtypedef{table.load.votable}
							{
								Load (possibly display, or otherwise
								acknowledge the receipt of) a table in
								VOTable format.
							}
							{
								\mtypeparam{url}{string}{
									URL of the VOTable document to
									load.
								}

								\mtypeparam[optional]{table-id}{string}
								{
									Identifier which may be used to
									refer to the loaded table in
									subsequent messages.
								}

								\mtypeparam[optional]{name}{string}{
									Name which may be used to label the
									loaded table in the application
									GUI.
								}
							}
							{\mtypeparamnone}
				\end{framed}
			\end{minipage}
			
			\caption[\method{table.load.votable} MType description]
			{Description of the \method{table.load.votable} MType.}
			\label{fig:tableLoadVotableMtype}
		\end{figure}
		
		\begin{figure}[tbp]
			\centering
			\begin{minipage}{0.9\textwidth}
				\begin{framed}
					\mtypedef{table.load.fits}
							{
								Load (possibly display, or otherwise
								acknowledge the receipt of) a data
								table in FITS format.
							}
							{
								\mtypeparam{url}{string}{
									URL of the FITS file to load.
								}
							
								\mtypeparam[optional]{table-id}{string}
								{
									Identifier which may be used to
									refer to the loaded table in
									subsequent messages.
								}
							
								\mtypeparam[optional]{name}{string}
								{
									Name which may be used to label the
									loaded table in the application GUI.
								}
							}
							{\mtypeparamnone}
				\end{framed}
			\end{minipage}
			\caption[\method{table.load.fits} MType description]
			{Description of the \method{table.load.fits} MType.}
			\label{fig:tableLoadFitsMtype}
		\end{figure}
		
		\begin{figure}[tbp]
			\centering
			\begin{minipage}{0.9\textwidth}
				\begin{framed}
					\mtypedef{table.highlight.row}
							{
								Highlights a single row of an
								identified table by row index. The
								table to operate on is identified by
								one or both of the \mapkey{table-id} or
								\mapkey{url} arguments. At least one of
								these MUST be supplied; if both are
								given they should refer to the same
								thing. Exactly what highlighting means
								is left to the receiving application.
							}
							{
								\mtypeparam[optional, if \mapkey{url} is
								specified]{table-id}{string}
								{
									identifier associated with a table,
									established by a previous message
									(e.g. \mapkey{ta\-ble.load.*})
								}

								\mtypeparam[optional, if
								\mapkey{table-id} is
								specified]{url}{string}{URL of a table.}

								\mtypeparam{row}{SAMP int}
								{
									Row index (zero-based) of the row
									to highlight.
								}
							}
							{\mtypeparamnone}
				\end{framed}
			\end{minipage}
			\caption[\method{table.highlight.row} MType description]
			{Description of the \method{table.highlight.row} MType.}
			\label{fig:tableHighlightRowMtype}
		\end{figure}
		
		\begin{figure}[tbp]
			\centering
			\begin{minipage}{0.9\textwidth}
				\begin{framed}
					\mtypedef{table.select.rowList}
							{
								Selects a list of rows of an identified
								table by row index. The table to
								operate on is identified by one or both
								of the \mapkey{table-id} or
								\mapkey{url} arguments. At least one of
								these MUST be supplied; if both are
								given they SHOULD refer to the same
								thing. Exactly what selection means is
								left to the receiving application.
							}
							{
								\mtypeparam[optional, if \mapkey{url} is
								specified]{table-id}{string}
								{
									Identifier associated with a table,
									established by a previous message
									(e.g. \mapkey{ta\-ble.load.*})
								}

								\mtypeparam[optional, if
								\mapkey{table-id} is
								specified]{url}{string}{URL of a table.}

								\mtypeparam{row}{list of SAMP int}
								{
									List of row indices (zero-based)
									defining which table rows are to
									form the selection
								}
							}
							{\mtypeparamnone}
				\end{framed}
			\end{minipage}
			
			\caption[\method{table.select.rowList} MType description]
			{Description of the \method{table.select.rowList} MType.}
			\label{fig:tableSelectRowListMtype}
		\end{figure}
		
		\begin{figure}[tbp]
			\centering
			\begin{minipage}{0.9\textwidth}
				\begin{framed}
					\mtypedef{image.load.fits}
							{
								Load (possibly display, or otherwise
								acknowledge) a two-dimensional FITS
								image.
							}
							{
								\mtypeparam{url}{string}
								{URL of the FITS image to be loaded.}

								\mtypeparam[optional]{image-id}{string}
								{
									Identifier which may be used to
									refer to the loaded FITS image in
									subsequent messages.
								}

								\mtypeparam[optional]{name}{string}
								{
									Name which may be used to label the
									loaded FITS image in the
									application GUI.
								}
							}
							{\mtypeparamnone}
				\end{framed}
			\end{minipage}
			
			\caption[\method{image.load.fits} MType description]
			{Description of the \method{image.load.fits} MType.}
			\label{fig:imageLoadFitsMtype}
			
		\end{figure}
		
		\begin{figure}[tbp]
			\centering
			\begin{minipage}{0.9\textwidth}
				\begin{framed}
					\mtypedef{coord.pointAt.sky}
							{
								Directs attention (e.g. by moving a
								cursor or shifting the field of view)
								to a given point on the celestial
								sphere.
							}
							{
								\mtypeparam{ra}{SAMP float}
								{Right ascension in degrees.}

								\mtypeparam{dec}{SAMP float}
								{Declination in degrees.}
							}
							{\mtypeparamnone}
				\end{framed}
			\end{minipage}
			
			\caption[\method{coord.pointAt.sky} MType description]
			{Description of the \method{coord.pointAt.sky} MType.}
			\label{fig:coordPointAtSkyMtype}
		\end{figure}
		
		\begin{figure}[tbp]
			\centering
			\begin{minipage}{0.9\textwidth}
				\begin{framed}
					\mtypedef{spectrum.load.ssa-generic}
							{
								Load (possibly display, or otherwise
								acknowledge) a spectrum or SED. The
								name refers to the fact that the
								metadata passed with this MType is
								based on the Simple Spectral Access
								protocol, but not on any particular
								version of it. The arguments are chosen
								such that it is convenient to use this
								MType for passing the results of an SSA
								query from an SSA client to a spectrum
								viewer (particularly to an SSA-capable
								spectrum viewer). However it is not
								necessary for SSA to be involved;
								SSA-like metadata may be faked and used
								to message loading of a spectrum from
								any source. In the latter case it is
								RECOMMENDED to provide at least the
								\mapkey{Access.Format} entry in the
								\mapkey{meta} map.
							}
							{
								\mtypeparam{url}{string}
								{URL of the spectrum to load}

								\mtypeparam{meta}{map}{
									Additional metadata describing the
									spectral data found at the URL.
									Key/Value pairs represent either
									Utypes or UCDs as defined or used
									in some version of the SSA
									specification or its predecessors.
									Example map keys are
									\mapkey{Access.Format} (SSA 1.0
									MIME type Utype) or
									\mapkey{VOX:Spectrum\_Format}
									(pre-1.0 SSA MIME type UCD). It is
									up to the recipient to make sense
									of these and, for instance, deal
									with the possibility that given
									expected keys are present or that
									apparently contradictory
									information is presented. Most
									existing SSA-aware spectrum viewing
									clients already contain this
									functionality.
								}

								\mtypeparam[optional]{spectrum-id}
								{string}
								{
									Identifier which may be used to
									refer to the loaded spectrum in
									subsequent messages.
								}

								\mtypeparam[optional]{name}{string}
								{
									Name which may be used to label the
									loaded spectrum in the application
									GUI.
								}
							}
							{\mtypeparamnone}
				\end{framed}
			\end{minipage}
			
			\caption[\method{spectrum.load.ssa-generic} MType
			description]
			{Description of the \method{spectrum.load.ssa-generic}
			MType.}
			\label{fig:spectrumLoadSsaGenericMtype}
		\end{figure}
		
				
		We can see in the message descriptions that applications
		developers' flexibility is encouraged, but an additional
		property of all of these messages collaborates in that
		flexibility: none of the MTypes above return anything, and
		the actions to be carried out by the receiving  application
		upon receipt of these messages can be completely arbitrary.
		
		We will use this flexibility to define a set of
		\emph{behaviours}, upon receipt of standard MTypes, which
		will aid in calculations, and will provide the API for
		the MOVOIR operations.
		
	% section already_defined_mtypes (end)
	
	% section movoir_alternative_behaviours (fold)
	\section{Creating alternative response patterns}
	\label{sec:movoir_alternative_behaviours}
		
		We have seen in the previous section that there are a 
		number of already standardised MTypes, together with their
		corresponding responses upon receipt of those MTypes.
		For all of those messages applications returned
		nothing\footnote{Formally, asynchronously called modules
		which return nothing SHOULD call the callers' \method{reply}
		method with a \maptype\ with keys \sampstatus\ set to
		\sampok\, and \sampresult\ set to an empty map, but
		applications
		SHOULD NOT behave incorrectly if they do not receive it.},
		and there were clear indications that applications should
		perform certain tasks upon receipt of such MTypes, but
		there is nothing the caller application can do in order to
		ensure a particular response, and this is a feature of this
		messaging system: applications are notified of an event,
		and what to do upon reception is completely up to them.
		
		One clear case of SAMP-based application which subscribes
		to the messages above, but does not really perform the
		actions specified by the MType description, are logger
		application. A typical logger application would register
		with a SAMP hub, and subscribe to every possible message.
		However, for all messages received the message handler
		is the same, and just logs the \sampmtype{} and \sampparams{}
		of the call, together with information about the caller
		application.
		
		The lesson to learn from this example is that applications
		which subscribe to broadcasted messages, and specially
		those who require no answer, are completely transparent to
		caller applications: we can perform any action it makes
		sense to perform upon receipt of these messages, as long
		as the formal behaviour expected by the calling application
		(in terms of answering a synchronous message, not letting
		the calling application blocked, filling all required
		keywords in the response...) is fulfilled.
		
		Let us explore a few examples of new functionality which
		can be added by two means: performing a somewhat different
		action upon receiving one of the standard messages, or
		letting called applications see who has called them, and
		send messages back to applications which support them.
		
		\subsection{Modifying or enhancing response actions} % (fold)
		\label{sub:modifying_or_enhancing_response_actions}
			The first case (modifying what is done by the receiving
			applications) was illustrated by our previous example of
			a logger application, and we will also use this way to
			create a VODownloader application.
			
			A VODownloader is, essentially, an application which,
			instead of actually working with the VOTables and FITS
			files which are sent to it, it downloads and saves them
			to a particular folder. This requires being able to
			respond to the \mapkey{table.load.votable},
			\mapkey{table.load.fits}, and \mapkey{image.load.fits},
			and use the \mapkey{url} parameter as a data source to
			download the file sent. This changes somehow the
			meaning of the \mapkey{table.load.*} and
			\mapkey{image.load.fits} MTypes, but in a sense which
			is compatible with the message definition, and provides
			a valuable service for integrating legacy applications
			for which there is no access to source code, and which
			cannot be integrated into the VO. For them, a
			VODownloader application is the only possible data
			access bridge to the VO.
		% subsection modifying_or_enhancing_response_actions (end)
		
		\subsection{Performing same-message call-backs} % (fold)
		\label{sub:performing_arbitrary_client_callbacks}
			When sending an asynchronous message, the caller
			application provides a function the called
			service needs to call back in order to provide
			the actual result.
			
			However, that is only possible for asynchronous
			messages with MTypes expecting return values, and none
			of the standard messages provide nothing back.
			Of course, we can create our own, custom MTypes, but
			those will only be useful for the subset of applications
			which known about those additional MTypes. If we want to
			provide an extension mechanism for already existing
			VO applications, and for those applications whose only
			connection to the VO is messaging, we need to provide
			extensions through the existing MTypes.
			
			Nonetheless, SAMP provides the mechanisms for that:
			as described in section 3.12, a callable client must
			support the \method{receiveNotification},
			\method{receiveCall}, and \method{reply} methods, and
			all of them receive a \mapkey{sender-id} parameter.
			
			This allows called applications to send messages to
			the clients who called them first, upon receipt of a
			particular message.
			
			In this way, we can think of different possibilities
			upon receipt of a given MType which can be replied by
			sending back the same MType:
			
			\begin{description}
				\item[\method{image.load.fits}] In this message,
				only the \mapkey{url} parameter is mandatory. Upon
				receipt of such a message, the called module would
				perform some processing: for instance, image
				inversion, rotation, reflection, automatic level
				adjustments, adaptive wavelet filtering, FFTs,
				et cetera. Then, it would send the application
				which sent the original message another
				\method{image.load.fits} MType with the processed
				image.
				
				\item[\method{table.load.votable}] Again, in this
				message only the \mapkey{url} parameter is
				mandatory. Upon receipt of such a message, the
				called module would perform some processing: for
				instance, FFT of VOTable columns (in order to
				get spectra from auto-correlation data), statistics
				on the columns (min, max, average, median, standard
				deviation, et cetera), et cetera. Then, it would
				send the application which sent the original table
				another \method{table.load.votable} MType with 
				the processed table.
			\end{description}
			
			In order for users to discover the actual behaviour of
			each module, the actual processing being performed
			(including default parameters being applied) SHOULD be
			available via the \mapkey{samp.description} metadata
			declared with the hub.
			
		% subsection performing_arbitrary_client_callbacks (end)
		
		\subsection{Performing different-message call-backs} % (fold)
		\label{sub:performing_different_message_call_backs}
			
			We are not restricted to create responses as messages
			of the same time to be back to the calling client. We
			might have, for example, a module which takes an
			\method{image.load.fits} MType, and uses
			SExtractor\urlnote{http://terapix.iap.fr/soft/sextractor/}
			to create a catalogue of sources detected on that FITS.
			The catalogue could be sent to other table loading
			applications (e.g. TOPCAT, but it could be broadcast to
			all subscribers of the \method{table.load.votable} MType)
			for display via a \method{table.load.votable} message,
			or simple stored as a file in a particular folder.
			
			Another interesting possibility is that of receiving
			\mapkey{ra} and \mapkey{dec} coordinates via a 
			\method{coords.pointAt.sky} MType, and performing a
			ConeSearch query on those coordinates for selected
			services, sending results back as a
			\method{table.load.votable} message.
			
			With the schema described up to now we could only build
			modules limited to process just one input, be it a
			table, an image, or a pair of coordinates. 
			
			However, we can push this concept further, so that SAMP
			modules which need to operate on two inputs would wait
			for two messages of the same type from the same client,
			with different data, in order to use both data inputs.
			
			One such example would be a module for cross-matching
			sources from two different catalogues. The sequence, in
			this case, would be:
			
			\begin{enumerate}
				\item An application (e.g. TOPCAT) sends a VOTable
				to an application (which we can call XMatch) which
				supports the \method{table.load.votable} MType.
				
				\item XMatch starts preparing for a XMatch of that
				first table with another table yet to be sent by 
				the same client.
				
				\item The same application sends a second VOTable
				to the XMatch application via the
				\method{table.load.votable} MType.
				
				\item XMatch performs a cross-matching of both
				tables, using default parameters, creating a third
				table, with all fields from both tables for
				sources which were successfully cross-matched.
				
				\item When the cross-matching table has been
				created, a message is sent to the caller
				application using the same
				\method{table.load.votable} MType.
			\end{enumerate}
			
			Such an XMatch module would be able to provide
			cross-matching services to every SAMP-enabled
			application able to send \method{table.load.votable}
			MTypes, as most applications being able to send
			\method{table.load.votable} MTypes can handle them as
			well. This means that just by supporting the
			standard messages, applications can be enhanced by
			modules which can act upon receipt of standard
			messages. We have, in fact, created a plug-in API for
			the VO.
			
			This operation model has the drawback of needing a
			series of default parameters which cannot be specified
			by the calling applications, as they are using a
			generic message for loading data.
			
			That problem could be overcome in two ways:
			first, as the data is sent in the VOTable format,
			if there are additional metadata on spatial resolution,
			we can build \emph{intelligence} into the cross-matching
			module to let it adjust its default parameters from the
			available information.
			
			Second, specially formatted VOTables could be sent to
			the modules via the same \method{table.load.votable}
			messages. The formatting would include precise field
			names corresponding to the parameters to be set for
			processing data from subsequent messages.
			
			A final goal of this thesis would be the development
			of an IVOA-sanctioned plug-in architecture,
			were specific MTypes would be created for getting
			application capabilities (default values set-up,
			methods for discovery of plug-in messages, et cetera)
			the creation of MTypes specific for handling these
			module properties.
			
			In that case, another tool which could be performed
			would be the automatic creation of messages for 
			applications, by just selecting the message, and
			providing links to the data being created.
			
			In addition, by publishing into the IVOA wiki the
			messages used by the MOVOIR, they could become more
			generalised, and available from many more modules.
			
			\invisiblenote
			{Apart from having these \emph{workarounds}, for
			being able to obtain extra functionality from already
			existing MTypes, providing specific MTypes
			(for instance \method{tables.xmatch.votables}) allows
			for better interfacing from scripts and other
			applications with the actual function being performed.}
			
		% subsection performing_different_message_call_backs (end)
		
	% section movoir_alternative_behaviours (end)
	
	\section{Describing MType parameters} % (fold)
	\label{sec:describing_mtype_parameters}
		
		The main idea permeating this part of the thesis is that
		of using SAMP as the way to provide a VO API to enhance
		existing VO applications, and those applications which
		have been updated to make use of SAMP messaging.
		
		In order for applications and developers to have access
		to the different functions and MTypes available from
		different active SAMP applications and/or modules,
		a discovery mechanism is needed.
		
		The SAMP hub aids in the discovery of available modules
		and of the different MTypes supported, but provides no way
		for applications to discover mandatory and optional
		parameters for particular MTypes. And the ability to get
		the data types, and possibly meanings, for each parameter,
		should also be taken into account.
		
		Hence, we propose a new MType,
		\method{movoir.describe.mtype}. The name has been chosen
		using the guidelines for MType identifiers set on
		the SAMP MTypes wiki page\urlnote{\sampmtypesurl}. This
		MType definition is shown in
		figure~\ref{fig:movoirDescribeMtype}.
		
		\begin{figure}[htbp]
			\centering
			\begin{minipage}{0.9\textwidth}
				\begin{framed}
					\mtypedef{movoir.describe.mtype}
							{
								Applications receiving this message
								need to provide a description of the
								MType specified by the \mapkey{mtype}
								parameter, including message
								parameters, their type, and whether
								they are optional or not, and the
								result map, also including message
								type. If the application does not
								support that MType, it MUST return an
								error condition.
							}
							{
								\mtypeparam{mtype}{string}
								{MType identifier to be described.}

								\mtypeparam[optional]{verbose}
								{SAMP boolean}
								{
								Boolean string stating whether the
								answer should be plain
								(\mapkey{false}), or verbose
								(\mapkey{true}), indicating an
								extra description is desired for
								parameters and the MType. Default
								value is \mapkey{false}.
								}
							}
							{
								\mtypeparam{parameters}{map}
								{
								Map where the keys are the
								different parameters supported by
								the specified \mapkey{mtype}, and
								values are maps with three
								mandatory keys: \mapkey{type},
								giving the SAMP type ---as in
								table~\ref{tabSAMPdataTypes}--- of
								the parameter; \mapkey{optional}, a
								SAMP boolean stating whether the
								parameter is optional or not; and
								\mapkey{ucd}, which provides a UCD
								for the kind of parameter. If
								\mapkey{verbose} is set to
								\mapkey{true}, an additional
								\mapkey{movoir.description} key
								SHOULD appear for each parameter.
								}
								
								\mtypeparam[optional]{movoir.result}
								{map}
								{
								If present, it indicates the return
								values provided my the
								\mapkey{mtype} MType, in the same
								way as parameters are described. It
								MUST appear for MTypes with return
								values.
								}
								
								\mtypeparam[optional]
								{movoir.description}{string}
								{
								If present, it provides a general
								description of the MType being
								analysed. It SHOULD appear when the
								\mapkey{verbose} parameter is set
								to \mapkey{true}.
								}
							}
				\end{framed}
			\end{minipage}
			\caption[\method{movoir.describe.mtype} MType description]
			{Description of the \method{movoir.describe.mtype} MType.}
			\label{fig:movoirDescribeMtype}
		\end{figure}
		
		We will show examples of use of the
		\method{movoir.describe.mtype} message, with both the
		constructed and received message map shown. For instance,
		the result for the \method{table.load.votable} message is
		shown in figure~\ref{fig:describeMtypeVOTableLoad}.
		
		\begin{figure}[tbp]
			\centering
			
			\subfloat[][]
			{
				\begin{minipage}{0.27\columnwidth}
					\label{subfig:describeTableLoadVotableSent}
					\begin{small}
						\begin{description}
							\item[\sampmtype]:
							\mapkey{"movoir.describe.mtype"}
							
							\item[\sampparams]:
							\begin{description}
								\item[\mapkey{mtype}]:
								\mapkey{"table.load.votable"}
							\end{description}
						\end{description}
					\end{small}
				\end{minipage}
			} 
			\hfill 
			\subfloat[][]
			{
				\begin{minipage}{0.67\columnwidth}
					\label{subfig:describeTableLoadVotableReceived}
					\begin{small}
					\begin{description}
						\item[\sampstatus]: \mapkey{"samp.ok"}
						
						\item[\sampresult]:
						\begin{description}
							\item[\mapkey{parameters}]:
							\begin{description}
								\item[\mapkey{url}]:
								\begin{description}
									\item[\mapkey{type}]:
									\mapkey{"string"}
									\item[\mapkey{optional}]:
									\mapkey{"false"}
									\item[\mapkey{ucd}]:
									\mapkey{"meta.url"}
								\end{description}
								
								\item[\mapkey{table-id}]:
								\begin{description}
									\item[\mapkey{type}]:
									\mapkey{"string"}
									\item[\mapkey{optional}]:
									\mapkey{"true"}
									\item[\mapkey{ucd}]:
									\mapkey{"meta.id"}
								\end{description}
								
								\item[\mapkey{name}]:
								\begin{description}
									\item[\mapkey{type}]:
									\mapkey{"string"}
									\item[\mapkey{optional}]:
									\mapkey{"true"}
									\item[\mapkey{ucd}]:
									\mapkey{"meta.name"}
								\end{description}
							\end{description} 
						\end{description}
					\end{description}
					\end{small}
				\end{minipage}
			} 
			\caption[Using \texttt{movoir.describe.mtype}
			in \mapkey{table.load.votable}.]
			{
				Example of use of the
				\method{movoir.describe.mtype} message to learn how
				a particular message works for a particular
				application. Subfigure
				\subref{subfig:describeTableLoadVotableSent} shows
				the map built to send the
				\method{movoir.describe.mtype} mesage,
				while~\subref{subfig:describeTableLoadVotableReceived}
				shows the resulting map. Compare this result with
				the description of the \method{table.load.votable}
				MType in section~\ref{sec:already_defined_mtypes}.
			}
			\label{fig:describeMtypeVOTableLoad}
		\end{figure}
		
		Figure~\ref{fig:describeVerboseFITSload}, on the other
		hand, shows the result of calling
		\method{mo\-voir.des\-cribe.mtype} with the additional
		\mapkey{verbose} parameter, while
		figure~\ref{fig:describeSelf} shows the result of calling
		\method{mo\-voir.des\-cribe.mtype} with itself as MType. This
		example allows us to see how to handle both maps as 
		parameters and results. For maps, additional information 
		needs to be provided in the form of UCD so that applications
		can learn automatically how to handle the map.
		
		\begin{figure}[tbp]
			\centering
			
			\subfloat[][]
			{
				\begin{minipage}{0.27\columnwidth}
					\label{subfig:describeTableLoadFITSSent}
					\begin{small}
					\begin{description}
						\item[\sampmtype]:
						\mapkey{"movoir.describe.mtype"}
						
						\item[\sampparams]:
						\begin{description}
							\item[\mapkey{mtype}]:
							\mapkey{"table.load.fits"}
							\item[\mapkey{verbose}]:
							\mapkey{"true"}
						\end{description}
					\end{description}
					\end{small}
				\end{minipage}
			} 
			\hfill 
			\subfloat[][]
			{
				\begin{minipage}{0.67\columnwidth}
					\label{subfig:describeTableLoadFITSReceived}
					\begin{small}
					\begin{description}
						\item[\sampstatus]: \mapkey{"samp.ok"}
						
						\item[\sampresult]:
						\begin{description}
							\item[\mapkey{movoir.description}]:
							\mapkey{"Load (possibly display, or
							otherwise acknowledge the receipt of) a
							data table in FITS format."}
							
							\item[\mapkey{parameters}]:
							\begin{description}
								\item[\mapkey{url}]:
								\begin{description}
									\item[\mapkey{movoir.description}]:
									\mapkey{"URL of the FITS file to be
									loaded."}
									\item[\mapkey{type}]:
									\mapkey{"string"}
									\item[\mapkey{optional}]:
									\mapkey{"false"}
									\item[\mapkey{ucd}]:
									\mapkey{"meta.url"}
								\end{description}
								
								\item[\mapkey{table-id}]:
								\begin{description}
									\item[\mapkey{movoir.description}]:
									\mapkey{"Identifier which may be
									used to refer to the loaded table
									in subsequent messages."}
									\item[\mapkey{type}]:
									\mapkey{"string"}
									\item[\mapkey{optional}]:
									\mapkey{"true"}
									\item[\mapkey{ucd}]:
									\mapkey{"meta.id"}
								\end{description}
								
								\item[\mapkey{name}]:
								\begin{description}
									\item[\mapkey{movoir.description}]:
									\mapkey{"Name which may be used to 
									label the loaded table in the
									application GUI."}
									\item[\mapkey{type}]:
									\mapkey{"string"}
									\item[\mapkey{optional}]:
									\mapkey{"true"}
									\item[\mapkey{ucd}]:
									\mapkey{"meta.name"}
								\end{description}
							\end{description} 
						\end{description}
					\end{description}
					\end{small}
				\end{minipage}
			} 
			\caption[Verbose description of the
			\mapkey{table.load.fits} method.]
			{
				Another example of use of the
				\method{movoir.describe.mtype} message. Subfigure
				\subref{subfig:describeTableLoadFITSSent} shows
				the map built to send the
				\method{movoir.describe.mtype} mesage with the
				\mapkey{verbose} parameter set to \mapkey{true},
				while~\subref{subfig:describeTableLoadFITSReceived}
				shows the resulting map. In this case, we
				have additional \mapkey{movoir.description} keys
				both for parameters and the message itself.
			}
			\label{fig:describeVerboseFITSload}
		\end{figure}
		
		\begin{figure}[tbp]
			\centering
			
			\begin{minipage}{\columnwidth}
				\begin{small}
				\begin{description}
					\item[\sampstatus]: \mapkey{"samp.ok"}
					
					\item[\sampresult]:
					\begin{description}
						
						\item[\mapkey{movoir.description}]: 
						\mapkey{"Applications receiving this
						message need to provide a description of
						the MType […]"}
						
						\item[\mapkey{parameters}]:
						\begin{description}
							\item[\mapkey{mtype}]:
							\begin{description}
								\item[\mapkey{movoir.description}]:
								\mapkey{"MType code to be
								described."}
								\item[\mapkey{type}]:
								\mapkey{"string"}
								\item[\mapkey{optional}]:
								\mapkey{"false"}
								\item[\mapkey{ucd}]:
								\mapkey{"meta.code; samp.mtype"}
							\end{description}
							\item[\mapkey{verbose}]:
							\begin{description}
								\item[\mapkey{movoir.description}]:
								\mapkey{"Whether to add
								\mapkey{movoir.description} […]."}
								\item[\mapkey{type}]:
								\mapkey{"SAMP boolean"}
								\item[\mapkey{optional}]:
								\mapkey{"true"}
								\item[\mapkey{ucd}]:
								\mapkey{"meta.code"}
							\end{description}
						\end{description}
						
						\item[\mapkey{movoir.result}]:
						\begin{description}
							\item[\mapkey{movoir.description}]:
							\begin{description}
								\item[\mapkey{type}]:
								\mapkey{"string"}
								\item[\mapkey{optional}]:
								\mapkey{"true"}
								\item[\mapkey{ucd}]:
								\mapkey{"meta.note"}
								\item[\mapkey{movoir.description}]:
								\mapkey{"Provides a general
								description of the MType
								being analysed[…]"}
							\end{description}
							
							\item[\mapkey{parameters}]:
							\begin{description}
								\item[\mapkey{type}]:
								\mapkey{"map"}
								\item[\mapkey{optional}]:
								\mapkey{"false"}
								\item[\mapkey{ucd}]:
								\mapkey{"movoir.parameters"}
								\item[\mapkey{movoir.description}]:
								\mapkey{"Map whose keys are
								the different parameters
								supported by the specified
								\mapkey{mtype} […]."}
							\end{description}
							
							\item[\mapkey{movoir.result}]:
							\begin{description}
								\item[\mapkey{type}]:
								\mapkey{"map"}
								\item[\mapkey{optional}]:
								\mapkey{"false"}
								\item[\mapkey{ucd}]:
								\mapkey{"movoir.result"}
								\item[\mapkey{movoir.description}]:
								\mapkey{"If present, it indicates
								the return values provided my the
								\mapkey{mtype} MType […]"}
							\end{description}
						\end{description}
						
					\end{description}
				\end{description}
				\end{small}
			\end{minipage}
			\caption[Verbose description of the
			\mapkey{table.load.fits} method.]
			{
				A more complex example of use of the
				\method{movoir.describe.mtype} message, where the
				MType being describe has parameters and provides
				results which themselves are maps. Compare this
				(abbreviated) result to the description of
				\method{movoir.describe.mtype} at the beginning of
				this section.
			}
			\label{fig:describeSelf}
		\end{figure}
		
		In any case, the textual descriptions allow any SAMP
		developer either to provide a similar support for a
		given MType, or to create clients able to send those
		messages.
		
		The most interesting feature of the
		\method{movoir.describe.mtype} message is that it depends
		on the application being called: in particular, the
		descriptions of the message correspond to the use of the
		parameters and of the return values (if any) by that
		particular module.
		
		\invisiblenote{
		We can see how this is valuable if we
		compare the result of sending the
		\method{movoir.describe.mtype} to two different
		applications. Figure~\ref{fig:compareVOTableLoadDescription}
		shows the comparison between a generic application which
		has implemented support for the \method{movoir.describe.type}
		message, and the VO Downloader.}
		
		Figure~\ref{fig:describeSelf} also shows how to complement
		the UCD1+ vocabulary with special vocabulary for
		parameters very specific to SAMP: for instance, there is no
		UCD1+ ---and there should be none--- to specify a SAMP
		result, so having either its own form (as in
		\mapkey{movoir.result}), or juxtaposing UC1+ terms with
		custom terms (as in \mapkey{meta.code; samp.mtype}), allows
		to maintain UCD semantics as much as possible, while adding
		a few atoms for clarity.
		
		Thus, the controlled vocabulary for the \mapkey{ucd} key is
		the union of the UCD1+ vocabulary with all terms starting
		with \mapkey{samp.*} ---interpreting the asterisk
		(\texttt{*}) as a wildcard--- defined in the SAMP
		Recommendation~\cite{2009samp.ivoav0904T}, plus the
		following terms specific for the MOVOIR (but which might be
		adopted/adapted by the IVOA):
		
		\begin{description}
			\item[\mapkey{movoir.result}] Identifies a map of
			MType result keys.
			
			\item[\mapkey{movoir.parameters}] Identifies a map
			of MType parameter keys.
		\end{description}
		
		
	% section describing_mtype_parameters (end)
	
	\section{Providing default values and settings to modules} % (fold)
	\label{sec:providing_default_values_settings}
		
		We had also identified the problem of setting default
		values for data processing algorithms implemented on
		top of SAMP/MOVOIR. For that, we propose two messages:
		
		\begin{description}
			\item[\method{movoir.configuration.set}] This message
			sends a \mapkey{url} to a VOTable with the different
			settings to be used by the module. An optional
			\mapkey{mtype} parameter can be used to restrict
			settings to the behaviour for a particular
			\mapkey{mtype}. In this case, the application SHOULD NOT
			take into account settings which configure behaviours
			for different \mapkey{mtype}s.
			In the case of asynchronous or synchronous calls to
			this message, the return value SHALL contain a
			\mapkey{url} to the new current values.
			If settings for
			different \mapkey{mtype}s were discarded, the application
			\sampstatus{} should be set to \sampwarning{}.
			
			\item[\method{movoir.configuration.get}] This message
			will return a \mapkey{url} key pointing to a VOTable
			containing the different parameters which can be set,
			and their current values, similar to that returned
			after a \method{movoir.configuration.set} message.
		\end{description}
		
		In particular, the MType definitions for these two messages
		are shown in figures~\ref{fig:movoirConfigurationSetMtype}
		and \ref{fig:movoirConfigurationGetMtype}.
		
		\begin{figure}[tbp]
			\centering
			\begin{minipage}{0.9\textwidth}
			\begin{framed}
			\mtypedef{movoir.configuration.set}
					{
					Applications receiving this message will use
					the \mapkey{url} parameter to retrieve a
					VOTable with two fields: parameter name,
					and parameter value (using SAMP types). These
					will be used to set those parameters affecting
					the behaviour of the module receiving the
					message.
					to provide a description of the MType
					specified by the \mapkey{mtype} parameter,
					including message parameters, their type, and
					whether they are optional or not, and the result
					map, also including message type. If the
					application does not support that MType, it MUST
					return an \samperror{} condition.
					}
					{
						\mtypeparam{url}{string}
						{
						URL to the VOTable containing the values to
						be set. FIELDs with a \mapkey{utype}
						attribute equal to
						\mapkey{movoir.pa\-ra\-me\-ter.key} will be
						considered to contain parameter names to be
						set, and FIELDs with \mapkey{utype} equal
						to \mapkey{movoir.parameter.value} will be
						considered as the value to set. Any
						additional fields will be discarded.
						}

						\mtypeparam[optional]{mtype}{string}
						{String containing a single MType to which
						the parameters to be set conform}
					}
					{
						\mtypeparam[optional]{url}{string}
						{
						For synchronous or asynchronous messages,
						the returned \mapkey{url} parameter holds a
						URL to a VOTable with the new settings of
						the application. In addition to the two
						fields with \mapkey{movoir.parameter.key}
						and \mapkey{movoir.parameter.value}
						U\-Types, a FIELD with \mapkey{name} equal
						to \mapkey{type}, and \mapkey{utype} equal
						to \mapkey{movoir.parameter.type} specifies
						the SAMP type of the parameter. If the
						parameter is only relevant to a particular
						MType, an additional field with
						\mapkey{name} equal to \mapkey{mtype}, and
						\mapkey{utype} equal to \mapkey{samp.mtype}
						should be provided.
						}
					}
			\end{framed}
			\end{minipage}
			\caption
			[\method{movoir.configuration.set} MType description]
			{Description of the \method{movoir.configuration.set}
			MType.}
			\label{fig:movoirConfigurationSetMtype}
		\end{figure}
		
		\begin{figure}[tbp]
			\centering
			\begin{minipage}{0.9\textwidth}
			\begin{framed}
			\mtypedef{movoir.configuration.get}
					{
					Applications receiving this message will
					provide a result with a \mapkey{url} key, which
					should point to a VOTable with four fields:
					parameter name, parameter value, parameter type
					(using SAMP types), and parameter mtype (empty
					for parameters not specific upon particular
					mtypes). If a non-supported mtype is specified
					in the optional \mapkey{mtype} parameter, the
					\sampstatus{} is set to \sampwarning{}.
					}
					{
						\mtypeparam[optional]{mtype}{string}
						{String containing a single MType to which
						the parameters of interest relate to.}
					}
					{
						\mtypeparam{url}{string}
						{
						URL to a VOTable with the current settings
						of the module. In addition to the two
						fields with \mapkey{movoir.parameter.key}
						and \mapkey{movoir.parameter.value} UTypes,
						a FIELD with \mapkey{name} equal to
						\mapkey{type}, and \mapkey{utype} equal to
						\mapkey{movoir.parameter.type} specifies
						the SAMP type of the parameter. If the
						parameter is only relevant to a particular
						MType, an additional field with
						\mapkey{name} equal to \mapkey{mtype}, and
						\mapkey{utype} equal to \mapkey{samp.mtype}
						should be provided.
						}
					}
		
			\end{framed}
			\end{minipage}
			\caption
			[\method{movoir.configuration.get} MType description]
			{Description of the \method{movoir.configuration.get}
			MType.}
			\label{fig:movoirConfigurationGetMtype}
		\end{figure}
		
	% section providing_default_values_settings (end)
	
	\section{MOVOIR modules to implement} % (fold)
	\label{sec:movoir_modules}
		
		After having presented how different SAMP-based processing
		strategies can be built, we will present here the different
		MOVOIR modules, and their corresponding MTypes they support
		with the actual behaviour being implemented.
		
		In particular, we have implemented the modules which were
		suggested in section~\ref{sec:legacy_conclusions} of the
		previous chapter:
		
		\begin{description}
			\item[VO Downloader] The VO Downloader is a SAMP
			application which can receive
			\method{table.load.votable}, \method{table.load.fits},
			and \method{image.load.fits} MTypes, and downloads the
			received file. If a \mapkey{name} parameter is
			provided, the file is renamed to that filename. A
			\mapkey{movoir.vo\-down\-load\-er.dlpath} configuration
			key in a \mapkey{.movoir} file is used to set the
			download path, while the current directory from where
			the VO Downloader has been launched In addition, the VO
			Downloader application can send the downloaded file
			back to any other \method{table.load.votable},
			\method{table.load.fits}, or \method{image.load.fits}
			subscriber, depending on file format.
			
			\item[VO Registry and DAL module] As the VO Desktop
			application has been ported to SAMP, we will use the
			VO Desktop as Registry access and DAL module: only
			the data results will need to be sent to SAMP
			applications. Any other SAMP-enabled registry and
			VO data access layer front-end can be used instead.
			
			\item[VOTable to FITS converter] Similar to the VO
			Downloader, the VOTable converter is an application
			which can receive messages with the
			\method{ta\-ble.load.votable} MType, and creates and
			saves the converted FITS file in the same path as the
			VO Downloader. In addition, the VOTable to FITS
			converter interface allows the user to send the
			converted FITS file to SAMP applications supporting the
			\method{ta\-ble.load.fits} MType.
			
			\item[AMIGA ConeSearch module] This module uses the
			standard \method{coord.point\-At.sky} M\-Type
			to provide results
			of a ConeSearch of 0.5 degrees radius around the
			\mapkey{ra} and \mapkey{dec} coordinates provided.
			It is also able to understand an optional,
			non-standard \mapkey{radius} parameter, to set the
			radius for the ConeSearch in degrees.
			
			\item[Sesame query module] MOVOIR provides a
			multi-object Sesame query message, as support for
			a future multi-cone search service, in order to
			resolve multiple object names at once. This is
			another example of a service which can be easily
			implemented in any language ---in this case,
			Python has been used---, and which can be called from
			languages with support for XML-RPC calls.
		\end{description}
		
		All MOVOIR modules will also support the
		\method{movoir.describe.mtype} message, as described in
		the previous section, in order to provide detailed 
		description not only on available parameters, or  
		
	% section movoir_modules (end)
	
	\section{Salient features of the MOVOIR} % (fold)
	\label{sec:salient_movoir_features}
	
		By the way it has been developed, the MOVOIR has the
		following properties:
		
		\begin{description}
			\item[Multi-language support] Astronomical data tools
			used to be written in FORTRAN, and compiled for
			UNIX-like systems. As of late, many astronomical
			applications, specially if they need to communicate
			with the VO, are being written in Java or in advanced
			scripting languages such as Python, which are available
			not only for UNIX-like systems, but for common
			platforms such as Windows or the Mac. The MOVOIR has
			parts written in Java, and in Python, but the interface
			is common to any kind of application supporting XML-RPC
			calls. In particular, it will support
			language-independent VO application messaging ---see
			section~\ref{sec:vo_application_messaging}---.
			
			\item[Multi-platform support] Stemming from the usage
			of platform neutral languages, such as Java and Python,
			the MOVOIR can be used in every major platform
			(Linux, Mac, Windows), and in any other platform
			supporting any language with an XML-RPC library, or
			even just able to create communication sockets.
			
			\newcommand{\stilurl}[0]
			{http://www.star.bristol.ac.uk/~mbt/stil/} \item[Reuse
			of existing packages] The MOVOIR uses the STIL table
			library\urlnote{\stilurl} for its internal table
			operations, and the AstroRuntime (the server part of
			the VO Desktop) for VO query support. The
			application-specific part of the MOVOIR uses also the
			STIL table library and the
			\texttt{nom.tam.fits}\urlnote{\nomtamfitsurl} Java
			library for Java applications, and the standalone
			STILTS package\urlnote{\stiltsurl} and the
			PyFITS\urlnote{\pyfitsurl} for Python applications.
			This allows for the combination of MOVOIR features with
			those independently exposed by the STIL and
			AstroRuntime, and the combination of scripts supporting
			both tools with those making use of MOVOIR's
			capabilities.
			
			\item[Single-operation, Multiple-targets] The MOVOIR is
			able to work on multiple targets as the same time in
			every function which would normally expect a single
			target. This allows more complex workflows to be built,
			and to enhance query throughput.
			
			 \item[Data Oriented Interface] Most VO applications
			tend to be built around the existing Data Access Layer,
			or the VO protocols used for data access. They also
			treat the coordinate resolution systems such as Sesame,
			or Registry queries, as separate parts of the system.
			The MOVOIR interface is data focused, and filtering is
			performed from the coordinates first, with particular
			queries fitted to each particular service. This is also
			true for the scripting interface.
			
			\item[Dynamic module update] As the MOVOIR uses a hub
			based mechanism for module registration, modules can be
			added or removed dynamically, and SAMP-enabled
			applications expect registered applications to change
			over time. This makes modules newly registered with the
			hub available to all applications able to send the
			MTypes the new modules are subscribed to. This feature
			eases development, deployment, and the updating of
			the MOVOIR, or any other SAMP-based package.
			
			 \item[Publish/subscribe, message-based API]
			As the MOVOIR uses the
			Simple
			Application Messaging Protocol (SAMP) as the main
			interaction bus, all operation requests are sent as
			SAMP messages, and the synchronous or asynchronous
			responses are also in the form of SAMP messages. This
			decision allows for the definition of an initial
			\emph{plug-in API} which is message-based, instead of
			language- or library-based.
		\end{description}
		
	
	% section salient_movoir_features (end)
	
	\section{Main issues} % (fold)
	\label{sec:movoir_main_issues}
		
		The main issues with the MOVOIR approach to enhancing VO
		applications are the following:
		
		\subsection{SAMP UI scalability} % (fold)
		\label{sub:samp_ui_scalability}
			The main problem with the MOVOIR is its scalability as
			a plug-in API. As the number of modules included with
			the MOVOIR, together with the number of SAMP
			applications, increases, so does the time needed by the
			user to find the correct module for sending the data.
			
			 The modularity of the MOVOIR helps, as modules do not
			need to be launched until they are needed, and are
			available right after launch in all applications, but
			still the problem of a likely increase of the number of
			SAMP-enabled applications and modules remains.
			
			Possible UI solutions to that problem would be:
			
			\begin{description}
				\item[SAMP specific communication panes] Instead of
				using a simple pop-up me\-nu, a SAMP-specific pane
				could be used, either as a floating or auxiliary
				window, or as a pop-up drawer, which allowed for
				a better access to metadata descriptions not just
				in the hub UI, but also in the client.
				
				\item[Hierarchical display of SAMP Mtypes]
				Applications could show available applications by
				the kind of message they can handle. However, only
				for SAMP standard messages there is a clear
				identification of the data type they can receive:
				for custom messages, such as those supported by
				the MOVOIR, only if all applications support the
				\method{movoir.describe.mtype} message can data
				types be discovered.
			\end{description}
		% subsection samp_ui_scalability (end)
		
		\subsection{High delay, unexpected responses} % (fold)
		\label{sub:high_delay_responses}
			Some users might think that the MOVOIR can create
			“non-causal” responses. That is, users might send
			a message to a module which takes a long processing
			time, and even reiterate the message, and then the
			application they are using might show two tables, or
			two images, that they are no longer expecting.
			
			\newcommand{\ariadneurl}
			{http://www.ariadneproject.org/index.php?id=63}
			In order to avoid this processing time should be
			less than real time expectations. The typical golden
			rule for interactive systems is delivering a response
			in under 8 to 15 seconds\urlnote{\ariadneurl}.
			Independently of the actual value, responses taking
			longer than that threshold might not be perceived as
			related to the initiating task, startling users. In
			any case, systems taking longer than one minute to
			process data should use messaging just to retrieve
			data, providing their own UI to initiate the
			processing and send results back. They should also
			provide progress information during the task
			processing, and either let the user send the result
			once it is available, or identify results with the
			originating task.
		% subsection high_delay_responses (end)
		
	% section movoir_main_issues (end)
	
	\section{Conclusions} % (fold)
	\label{sec:movoir_conclusions}
		
		In this chapter we have shown how a data processing API
		can be built on top of the SAMP messaging protocol, and
		what ideas can be developed in order to create an
		extension mechanism for already existing VO applications.
		
		The main achievement for the MOVOIR, being based on
		SAMP, is that it provides a way to create plug-ins for
		existing VO applications, which can be made independent of
		both the platform and language those modules can be written
		into, and completely independent of applications being
		enhanced.
		
		We have also proposed a few extra messages which can be
		used to learn about plug-in parameters, and configurability,
		with the aim to start a discussion within the IVOA on
		what would be the best one to implement such a plug-in
		API.
		
	% section movoir_conclusions (end)
	
% chapter movoir (end)
