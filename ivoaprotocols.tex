% chapter vo_protocols (fold)
\chapter{VO protocols}
\label{cha:vo_protocols}

In this appendix we present a quick guide to the usage of the different
available IVOA proposed Virtual Observatory protocols, with the aim of
making easier to understand the official standard documents. It can be
used as a fast reference of what is needed both for clients and servers
compliant with these protocols.

	\section{Simple ConeSearch} % (fold)
	\label{sec:simple_conesearch}

	\subsection{Query syntax} % (fold)
	\label{sub:query_syntax_scs}
	
	
	ConeSearch v1.0 query with all parameters. Mandatory 
	parameters in black. \textcolor{red}{Optional parameters in 
	red} (light grey in black and white).

	\begin{adjustwidth}{\parindent}{0 cm}
		\texttt{endPointURL?RA=rightAscension\&\\
				DEC=declination\&\\
				SR=searchRadius\&\\
				\textcolor{red}{VERB=verbosity}}
	\end{adjustwidth}

	% subsection simple_query_syntax (end)
	
	\subsection{Parameter description} % (fold)
	\label{sub:parameter_description_scs}
	
	\begin{description}
		\item[\texttt{endPointURL}] represent the Uniform Resource 
		Locator of a deployed ConeSearch web service. 
		Usually of the form:
		
		\texttt{http://host.com/path/to/service}

		\item[\texttt{rightAscension}] codifies the Right Ascension 
		of the region of interest. It has to be provided in decimal 
		degrees, in the ICRS coordinate system.

		\item[\texttt{declination}] codifies the Declination of the 
		region of interest. It has to be provided in decimal 
		degrees, in the ICRS coordinate system.

		\item[\texttt{searchRadius}] is given in decimal degrees.

		\item[\textcolor{red}{\texttt{verbosity}}] specifies reply
        verbosity; \texttt{verbosity} can be one of \texttt{1},
        \texttt{2} or~\texttt{3},
		with \texttt{1} providing the least verbose
        reply, and \texttt{3} the most verbose.
	\end{description}
	
	% subsection parameter_description (end)
		
	\subsection{Service reply} % (fold)
	\label{sub:service_reply_scs}
	
	A ConeSearch service reply consist of a VOTABLE, with complies with 
	the following :

	\begin{itemize}
		\item The VOTABLE response contains just one RESOURCE, which 
		contains just one TABLE.

		\item Table fields:
		\begin{itemize}
			\item Exactly ONE field has \ucd{"ID\_MAIN"}, 
			with the \texttt{string} data type, which provides an 
			unique ID for each table row/record.

			\item Exactly ONE field has 
			\ucd{"POS\_EQ\_RA\_MAIN"},
			with data type \texttt{double}, representing the Right 
			Ascension of the observed source in the ICRS coordinate 
			system.

			\item Exactly ONE field has 
			\ucd{"POS\_EQ\_DEC\_MAIN"}, with data type 
			\texttt{double}, representing the Declination of the 
			observed source in the ICRS coordinate system.

			\item An additional \xmlopen{FIELD} or \xmltag{PARAM} 
			with \ucd{"OBS\_ANG-SIZE"} can be used to 
			specify a per-record (\xmlopen{FIELD}) or per-table 
			(\xmlopen{PARAM}) positional error on source positions, 
			or angular size for resolved observations.

			\item The rest of the returned fields depend on the 
			catalog,but all \xmlopen{FIELD} or \xmltags{PARAM} 
			should specify description, data type, and UCD.
		\end{itemize}
	\end{itemize}
	
	% subsection service_reply (end)
	
	% section sec:simple_conesearch (end)
	\section{Simple Image Access Protocol} % (fold)
	\label{sec:siap}

	\subsection{Query syntax} % (fold)
	\label{sub:query_syntax_siap}
	
	Mandatory query parameters listed in black. Optional parameters
	listed in \textcolor{blue}{blue}. Not all services 
	have to
	implement all 
	optional parameters, except for the \texttt{FORMAT=formatType}. Use 
	\texttt{metadata} as \texttt{formatType} to retrieve supported 
	parameters.
	
	\begin{adjustwidth}{\parindent}{0 cm}
		\texttt{endPointURL?POS=RA,DEC\&\\
				SIZE=searchRadius\&\\
				\textcolor{red}{FORMAT=formatType\&}\\
				\textcolor{red}{INTERSECT=intersectionMechanism\&}\\
				\textcolor{red}{NAXIS=axisSizeVector\&}\\
				\textcolor{red}{CFRAME=coordFrame\&}\\
				\textcolor{red}{EQUINOX=equinoxSpec\&}\\
				\textcolor{red}{CRPIX=refPixCoordVector\&}\\
				\textcolor{red}{CRVAL=refPixValueVector\&}\\
				\textcolor{red}{CDELT=pixScaleVector\&}\\
				\textcolor{red}{ROTANG=rotationAngle\&}\\
				\textcolor{red}{PROJ=projectionKind\&}\\
				\textcolor{red}{VERB=verbosity}}
	\end{adjustwidth}
			
	% subsection query_syntax_siap (end)
	
	\subsection{Parameter description} % (fold)
	\label{sub:parameter_description_siap}
	\begin{description}
		\item[\texttt{endPointURL}] represents the Uniform Resource 
		Locator of a deployed SIAP web service. 
		Usually of the form:
		
		\texttt{http://host.com/path/to/service}

		\item[\texttt{RA,DEC}] is a pair of \texttt{double} 
		quantities, separated by a \comma, 
		representing Right Ascension and Declination 
		in decimal degrees. No official consideration of coordinate 
		system, but \texttt{ICRS} can be assumed.

		\item[\texttt{searchRadius}] is either a \texttt{double}
        quantity, or a pair of \texttt{double} quantities separated
        by a \comma. If a pair of values is specified, the first is
        assumed to be an RA \texttt{width}, and the second a DEC
        \texttt{width}. The region thus defined has constant RA and
        DEC edges. The combination of \texttt{POS} and \texttt{SIZE}
        will be called \emph{region of interest} (ROI) in the
        following.
	
		\item[\textcolor{red}{\texttt{formatType}}] is a \texttt{string} 
		representation of the MIME media type of files being asked 
		for to the service. It will be one of:
		
		\begin{itemize}
			\item \texttt{image/fits}
			\item \texttt{image/png}
			\item \texttt{image/jpeg}
			\item \texttt{text/html}
			\item \texttt{fits} (equivalent to \texttt{image/fits})
			\item \texttt{graphic} (equivalent to 
								    \texttt{image/png; image/jpeg})
			\item \texttt{all} (default value)
		\end{itemize}
		
		Additionally, if \texttt{formatType} is 
		\texttt{metadata}, the service will reply with a 
		VOTABLE with all supported input parameters.
		
		\item[\textcolor{red}{\texttt{intersectionMechanism}}] specifies
        how to evaluate the intersection of candidate images with the
        ROI specified. It has to be ONE of the following:
		
		\begin{description}
			\item[\texttt{COVERS}:] returned images completely 
			includes (covers) the entire ROI.
			
			\item[\texttt{ENCLOSED}:] returned images are completely 
			covered (enclosed) by the ROI. Uses the same mechanism 
			of \texttt{COVERS}, reversing the role of ROI and test 
			image.
			
			\item[\texttt{CENTER}:] returned images include at least 
			the center of the ROI. This is an special case of 
			\texttt{OVERLAPS}.
			
			\item[\texttt{OVERLAPS}:] returned images have some 
			common points with the specified ROI. \texttt{OVERLAPS}
			should return at least all images returned by 
			\texttt{CENTER}.
		\end{description}
		
		\item[\textcolor{red}{\texttt{axisSizeVector}}] is the size of
        the output image in pixels. This is a vector-valued quantity,
        expressed as \texttt{NAXIS}\texttt{=}\texttt{<width>,<height>}.
		
		\item[\textcolor{red}{\texttt{coordFrame}}] is a \texttt{string}
        specifying a coordinate system reference frame. It is one of:
		
		\begin{itemize}
			\item \texttt{ICRS} (default)
			\item \texttt{FK5}
			\item \texttt{FK4}
			\item \texttt{ECL}
			\item \texttt{GAL}
			\item \texttt{SGAL}
		\end{itemize}
		
		\item[\textcolor{red}{\texttt{equinoxSpec}}] is the epoch of the
        mean equator and equinox for the coordinate system reference
        frame specified by \texttt{coordFrame}. It is not required for
        \texttt{ICRS}, and defaults to \texttt{B1950} for \texttt{FK4},
        and to \texttt{J2000} otherwise.
		
		\item[\textcolor{red}{\texttt{refPixCoordVector}}] provides the
        coordinates of the reference pixel, in pixel coordinates of the
        output image. This is a vector-valued quantity of the same size
        as \texttt{axisSizeVector}.
		
		\item[\textcolor{red}{\texttt{refPixValueVector}}] is a vector
        of the world coordinates relative to \texttt{coordFrame} at the
        \texttt{refPixCoordVector}. Defaults to the ROI center
        coordinates, transformed to the output coordinate system
        reference if other than \texttt{ICRS}.
		
		\item[\textcolor{red}{\texttt{pixScaleVector}}] provides the
        scale of the output image in decimal degrees per pixel. A
        negative value implies an axis flip. Since the default image
        orientation is N up and E to the left, the default sign of
        \texttt{pixScaleVector} is [-1,1]. Can be given as comma
        separated vector, or as a single value; if only one value is
        given, it applies to both image axes.
		
		\item[\textcolor{red}{\texttt{rotationAngle}}] specifies the
        rotation angle for the output image in degrees relative to
        \texttt{coordFrame} (an image which is unrotated in one
        reference frame may be rotated in another). Rotation of the WCS
        declination or latitude axis with respect to the second axis of
        the image, measured in the counterclockwise direction (following
        FITS WCS convention, which in turn is based on the old AIPS
        convention). Defaults to 0 (no rotation).
		
		\item[\textcolor{red}{\texttt{projectionKind}}] specifies the
        celestial projection of the output image; is one of
        \texttt{TAN}, \texttt{SIN}, \texttt{ARC}, et cetera. Defaults to
        \texttt{TAN}.
		
		\item[\textcolor{red}{\texttt{verbosity}}] is an
        \texttt{integer} value, ranging from \texttt{0} to \texttt{3}
        that specifies reply verbosity (how much additional meta data
        are provided).
		
	\end{description}
	
	% subsection parameter_description (end)
	\begin{description}



		\item[Optional parameters:] query with all mandatory and 
		optional parameters in version 1:


		Remember that parameters not supported by the service 
		will be discarded. Call the service with 
		\texttt{FORMAT}\texttt{=}\texttt{metadata} to retrieve 
		supported parameters (see below).

% 		\begin{description}
% 
% 		\end{description}

		\item[SIAP v2 extensions:] query with mandatory and extended 
		parameters:
		
		\texttt{endPointURL?POS=RA,DEC\&\\
				SIZE=searchRadius\&\\
				REQUEST=requestKind\&\\
				TIME=timeSpec\&\\
				ANGRP=angularResolution\&\\
				PUBDID=pubID\&\\
				CREATORDID=creatorID\&\\
				COLLECTION=collectionID\&\\
				TOP=topNumResults\&\\
				MAXREC=maxRecords\&\\
				MTIME=incorporationTime\&\\
				COMPRESS=compressionFlag\&\\
				RUNID=jobRunID\&\\
				REDSHIFT=redShiftRange\&\\
				TARGETNAME=targetName}
		
		See SSAP for explanations on 
		\texttt{TIME}, \texttt{PUBDID}, \texttt{CREATORDID}, 
		\texttt{COLLECTION}, \texttt{TOP}, \texttt{MAXREC}, 
		\texttt{MTIME}, \texttt{COMPRESS}, \texttt{RUNID}, 
		\texttt{REDSHIFT} and \texttt{TARGETNAME}.
		
		\begin{description}
			\item[\texttt{requestKind}] specifies the kind of image 
			services being required: cut-out, mosaicing, atlas 
			archive, pointed archive. A service can support all or 
			some (at least one) of these request kinds.
			
			\item[\texttt{angularResolution}] specifies the angular 
			resolving power of the image, for interpolated images.
		\end{description}
		
		
		\item[SIAP multidimensional extensions:] for SIAP to be 
		extended to support multidimensional data sets, keywords 
		\texttt{NAXIS}, \texttt{CRPIX}, \texttt{CRVAL} and 
		\texttt{CDELT} should be enhanced to specify $N$ dimensions 
		(typically, $N \in \{3,4\}$); first two should always be RA and 
		DEC, with either time, frequency wavelength or velocity as 
		the third dimension, and polarisation as a fourth axis, if 
		present.
		
		\item[Service reply:] a SIAP service reply consist of a 
		VOTABLE, with complies with the following :

		\begin{itemize}
			\item \todoinlinesuspended{Finish list of restrictions to
			                  SIAP reply.}
		\end{itemize}
		
	\end{description}
	
	% section siap (end)
	\section{Simple Spectral Access Protocol} % (fold)
	\label{sec:ssap}
	
	\subsection{Query syntax} % (fold)
	\label{sub:query_syntax}

	Parameters all SSAP compliant services need to implement are in 
	black. Parameters whose implementation is optional, but 
	recommended, appear in \textcolor{blue}{blue}, and parameters 
	whose implementation is completely optional appear in 
	\textcolor{red}{red}.

	\begin{adjustwidth}{\parindent}{0 cm}
		\noindent\texttt{endPointURL?POS=RA,DEC\&\\
				SIZE=searchRadius\&\\
				BAND=freqRange\&\\
				TIME=timeRange\&\\
				FORMAT=formatType\&\\
				\textcolor{blue}{SPECRP=specResol\&}\\
				\textcolor{blue}{SPATRES=spatialResol\&}\\
				\textcolor{blue}{PUBDID=pubID\&}\\
				\textcolor{blue}{CREATORDID=creatorID\&}\\
				\textcolor{blue}{COLLECTION=collectionID\&}\\
				\textcolor{blue}{TOP=topNumResults\&}\\
				\textcolor{blue}{MAXREC=maxRecords\&}\\
				\textcolor{blue}{MTIME=modificationTime\&}\\
				\textcolor{blue}{COMPRESS=compressionFlag\&}\\
				\textcolor{blue}{RUNID=jobRunID\&}\\
				\textcolor{red}{APERTURE=apertAngle\&}\\
				\textcolor{red}{TIMERES=timeResol\&}\\
				\textcolor{red}{SNR=signal2noise\&}\\
				\textcolor{red}{REDSHIFT=redShiftRange\&}\\
				\textcolor{red}{VARAMPL=amplitudeVariability\&}\\
				\textcolor{red}{TARGETNAME=targetName\&}\\
				\textcolor{red}{TARGETCLASS=targetClass\&}\\
				\textcolor{red}{FLUXCALIB=fluxCalibKind\&}\\
				\textcolor{red}{WAVECALIB=waveCalibKind}}
	\end{adjustwidth}
	
	None of the service mandatory parameters is needed \emph{per se} 
	in a particular query: the following are legal, minimal SSAP 
	queries:
	
	\begin{adjustwidth}{\parindent}{0 cm}
		\noindent\texttt{endPointURL?POS=RA,DEC\&\\
				SIZE=searchRadius}
	\end{adjustwidth}
	
	\begin{adjustwidth}{\parindent}{0 cm}
		\noindent\texttt{endPointURL?TIME=timeRange\&\\
				BAND=freqRange}		
	\end{adjustwidth}
	
	% subsection query_syntax (end)
	
	\subsection{Parameter description} % (fold)
	\label{sub:parameter_description_ssap}

	The following is a description of the different SSAP parameters.
	In SSAP, parameters can be qualified by juxtaposing
	additional qualifiers, separated by a \semicolon. Appropriate 
	qualifiers will be pointed out as needed.

	\begin{description}
		\item 
		
		\item[\texttt{endPointURL}] represents the Uniform Resource 
		Locator of a deployed SIAP web service. 
		Usually of the form:
		
		\texttt{http://host.com/path/to/service}

		\item[\texttt{RA,DEC}] is a pair of \texttt{double} 
		quantities, separated by a \comma, 
		representing Right Ascension and Declination 
		in decimal degrees, in the ICRS coordinate system unless 
		otherwise specified.
		
		\texttt{POS=RA,DEC} can be substituted by 
		\texttt{GALACTICPOS=GalLat,GalLong}. Not all services need 
		to support \texttt{GALACTICPOS}.
		
		\item[\texttt{searchRadius}] is either a \texttt{double} 
		quantity, or a pair of \texttt{double} quantities separated 
		by a \comma. If a pair of values is specified, 
		the first is assumed to be an RA \texttt{width}, and the 
		second a DEC \texttt{width}. The region thus defined has 
		constant RA and DEC edges. The combination of \texttt{POS} 
		and \texttt{SIZE} will be called \emph{region of interest}
		(ROI) in the following.
		
		\item[\texttt{freqRange}] specifies a frequency range the 
		query should be restricted to, expressed as low/high 
		wavelengths in meters: \ttequals{BAND}{2.7E-7/0.13}. A 
		\texttt{source} qualifier can be used to specify that
		the frequencies are being specified at the source LSR:
		\ttequals{BAND}{0.001/0.01;source} 

		\item[\texttt{timeRange}] restricts the query to a given 
		date, specified as an ISO 8601 [1] date-time string (e.g., 
		\texttt{TIME=1998-05-21}), or a date-time range in UTC: 
		\texttt{TIME=1998-05-21/1999}). Qualifiers can be used to 
		specify Local Sidereal Time or other time zones:
		
		\texttt{TIME=2003-04-01T02:00:00/2003-04-01T06:00:00;LST}.
	
		\item[\texttt{formatType}] is a \texttt{string} 
		representation of the MIME media type of files being asked 
		for to the service. It will be one of:
		
		\begin{itemize}
			\item \texttt{application/fits}
			\item \texttt{compliant} (\texttt{votable} or 
									  \texttt{xml})
			\item \texttt{native} (native format of the archive)
			\item \texttt{graphic} (PNG, JPEG or EPS)
			\item \texttt{votable} (compare with XML)
			\item \texttt{fits} (equivalent to 
								 \texttt{application/fits})
			\item \texttt{xml} (SED XML serialisation)
			\item \texttt{all} (default value)
		\end{itemize}
		
		Additionally, if \texttt{formatType} is 
		\texttt{metadata}, the service will reply with a 
		VOTABLE with all supported input parameters.
		
		\texttt{formatType} can be qualified by the conventions 
		used for spectral reduction, such as 
		\texttt{convention=STScI-STIS}.
		
		\item[\texttt{specResol}] is a \texttt{double} specifying 
		spectral resolving power, $\frac{\lambda}{\Delta\lambda}$; 
		\texttt{specResol} is a dimensionless quantity.
		
		\item[\texttt{spatialResol}] is a \texttt{double} specifying 
		the minimum spatial resolution, in decimal degrees, 
		corresponding to signal PSF.
		
		\item[\texttt{pubID}] is a \texttt{string} holding the 
		NASA ADS publication ID.
		
		\item[\texttt{creatorID}] is a \texttt{string} specifying 
		the IVORN for the data set creator.
		
		\item[\texttt{collectionID}] is a \texttt{string} specifying 
		an IVORN for a particular collection, or a short-name: for 
		instance, the short-name \texttt{HST Spectra}, or its IVORN:
		\texttt{ivo://mast.stsci/ssap/hst}.
		
		\item[\texttt{topNumResults}] is an \texttt{integer}
        specifying the maximum number of results to be returned by
        the query, but requesting them sorted by an \emph{heuristic
        score} defined by the service. That score is built to be
        higher for better matching records. In any case, the same
        query should always return the same \texttt{topNumResults}
        records. Default value is up to the service, but should be
        small enough to deliver fast responses.
		
		\item[\texttt{maxRecords}] is an \texttt{integer} 
		specifying the maximum number of results to be returned by 
		the query. The same query might return different records 
		depending on service implementation. Default value is up 
		to the service, but should be small enough to deliver fast 
		responses.
		
		\item[\texttt{modificationTime}] is similar to 
		\texttt{timeRange} in syntax, but refers to record 
		creation or modification date-time.
		
		\item[\texttt{compressionFlag}] is a \texttt{string} 
		containing either \texttt{true} or \texttt{false}. If 
		\texttt{compression\-Flag} is \texttt{true}, returned 
		data might be compressed. Default is \texttt{false}.
		
		\item[\texttt{jobRunID}] is a \texttt{string} with 
		contains a client side request ID, so that records from 
		different services, called with the same jobRunID, can 
		be easily \texttt{aggregated}.
		
		\item[\texttt{apertAngle}] is a \texttt{double} used to
		specify a synthetic aperture angle in decimal degrees in
		order to extract spectra from data cubes, event lists or 
		other fundamental data. For extraction services, this 
		parameter is mandatory, and the service should provide a
		sensible default.
		
		\item[\texttt{timeResol}] is a \texttt{double} specifying
        minimum time resolution in seconds (typically, the bounds of
        the exposure time coverage).
		
		\item[\texttt{signal2noise}] is a dimensionless
        \texttt{double} specifying the Signal to Noise ratio, as a
        number of times noise RMS.
					
		\item[\texttt{redShiftRange}] is a \texttt{string} encoding
        a \texttt{double} or a range of \texttt{double}s, which
        specifies a particular dimensionless redshift or range,
        calculated using the optical convention $z =
        \frac{\Delta\lambda}{\lambda}$: \texttt{REDSHIFT=1.3/3.0}
        means $z \in [1.3,3.0]$, \texttt{REDSHIFT=1.3/} means $z \in
        [1.3,+\infty)$, while querying with \texttt{REDSHIFT=/3.0}
        means $z \in [0,3.0]$. Negative values can be used to
        indicate \emph{blueshifts}.

		\item[\texttt{amplitudeVariability}] is dimensionless 
		\texttt{double} in the range [0.0,1.0] indicating amplitude 
		variations from the maximum.
		
		\item[\texttt{targetName}] is a \texttt{string} providing 
		the name of target that will need to be resolved by SIMBAD,
		NED, o similar services. To be used instead of coordinates.
		
		\item[\texttt{targetClass}] is a \texttt{string} providing
        comma delimited list of types of astronomical objects to be
        searched for. Valid types form controlled vocabulary coming 
		from Simbad [3]. Eventually types should come from the IVOA 
		Thesaurus.
		
		\item[\texttt{fluxCalibKind}] is a \texttt{string} encoding 
		a boolean value (\texttt{true} or \texttt{false})
		indicating whether we require returned data fluxes to be 
		calibrated. No distinction is made between absolute or 
		relative flux calibration.
		
		\item[\texttt{waveCalibKind}] is a \texttt{string} encoding 
		a boolean value (\texttt{true} or \texttt{false})
		indicating whether we require returned data wavelengths to 
		be calibrated. No distinction is made between absolute or 
		relative wavelength calibration.
	\end{description}
	
	
	% subsection parameter_description (end)
	
	\subsection{Service reply} % (fold)
	\label{sub:service_reply_ssap}
	SSAP services reply with a VOTABLE, with complies with the 
	following:

	\begin{itemize}
		\item VOTABLE condition
	\end{itemize}
	
	% subsection service_reply (end)
	
	% section ssap (end)

% chapter vo_protocols (end)
