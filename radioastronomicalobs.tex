% chapter specific_characteristics_or_radio_signals (fold)
\chapter{Radio astronomical observations}
\label{introradiospecifics}

This chapter attempts to summarise the main concepts in radio
astronomical observations relevant to astronomical data archival and
retrieval. For an excellent overview of radio astronomy, used as a source
for this thesis, the reader might consult Rohlfs and Wilson's
\emph{Tools of radio astronomy}~\cite{2004tra..book.....R}.

Radio telescopes detect electromagnetic (EM) radiation generated all
around the universe. The first astronomical radio source ever detected
was the nucleus of our own galaxy, when Karl Jansky received an
extraterrestrial signal in his array of dipole
antennas~\cite{Jansky:1933db, Jansky:1935lq}.

The mechanisms for generating radio EM radiation involve the deceleration
of electrons, whose energy losses are radiated away. There are two main
mechanisms for such radiation: slow electrons (moving at a small fraction
of the speed of light, $c$, but can be moving at hundreds of kilometres
per second) interacting with hot clouds of ionised gas, or relativistic
electrons accelerated by some kind of large scale explosion and moving
within a magnetic field. The first kind of emission is normally called
\emph{thermal emission}, as it is mainly dominated by the thermal
properties of the gas, and the second one is called \emph{synchrotron
emission} or radiation, as the emission mechanism is dominated by strong
circular acceleration.

The range for radio waves can be seen in
table~\ref{tab:RadioWavelenghts}.

\todo{Build \texttt{tab:RadioWavelenghts} table, including the name of
radio bands, and their typical extent.}

\section{Angular resolution} % (fold)
\label{sec:angular_resolution}

Single-dish radio astronomical telescopes receive all incoming radiation
within a beam whose angular resolution depends on the wavelength of the
radiation, and on the diameter of the telescope.

\todo{Look up \emph{Tools of radio astronomy}, \emph{The invisible
universe}, y el TIT de Vicent para la parte de la descripción del
patrón del haz y demás. Repasar curso de radio astronomía de IRAM.}



% section angular_resolution (end)

\section{Observing modes} % (fold)
\label{sub:observing_modes}

\todo{Coger tesis de Itziar y explicar observaciones espectroscópicas.
Buscar también observaciones de contínuo. Aprovechar también manual de
HIFI, y repasar manual de IRAM.}

\todo{
	\begin{itemize}
		\item single-dish vs. interferometry
		\item spectra
		\item continuum observations
		\item maps
		\item OTF
	\end{itemize}
}

% section observing_modes (end)

\section{Data reduction} % (fold)
\label{sub:data_reduction}

\todo{Specify data reduction steps in connection with the
      observations that have to be combined.}

% section data_reduction (end)

\section{Data products} % (fold)
\label{sub:data_products}

\begin{itemize}
	\item single-dish continuum
	\item single-dish spectra
	\item single-dish maps
	\item OTF data-cubes
	\item interferometry data-cubes
\end{itemize}

\section{Radio astronomical units} % (fold)
\label{sec:radio_astronomical_units}

% section radio_astronomical_units (end)

% section data_products (end)

% chapter specific_characteristics_or_radio_signals (end)
