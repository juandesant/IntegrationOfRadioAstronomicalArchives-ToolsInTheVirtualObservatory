\chapter*[Agradecimientos]{Agradecimientos}
\addcontentsline{toc}{chapter}{Agradecimientos}
\label{agradecimientos}

	Terminar y entregar una tesis doctoral no es cosa sencilla. Se
	pasa a veces mucha angustia, porque uno llega a pensar que su
	trabajo no merece la pena, o que jamás lo acabará. Y aunque se
	trate de un esfuerzo eminentemente personal, la ayuda de otras
	personas es crucial, y a todos ellos quiero ofrecer mi
	agradecimiento más sincero.
	
	 A mis padres los primeros, por su apoyo incondicional, que me
	ha facilitado todo, especialmente porque ellos siempre
	fomentaron mi curiosidad en el sentido del término latín
	\emph{curiositas}, tal y como lo recoge el Diccionario de
	Autoridades: \emph{Deseo, gusto, apetencia de ver, saber y
	averiguar las cosas, cómo son, suceden, o han pasado.}

	 No deja de parecerme sintomático que, desde 1992, la primera
	acepción del DRAE sea \emph{Deseo de saber o averiguar alguien
	lo que no le concierne.} Sin \emph{curiositas} no hay ciencia
	posible.
	
	 A continuación tengo que agradecerle a Lourdes, directora de
	esta tesis, su gran valentía: valentía a la hora de contratar a
	una persona de cierta edad, con un perfil técnico-comercial,
	sobre cuya capacidad de integración en un grupo de
	investigación tenía razonables dudas; valentía a la hora de
	permitir que una persona que, en principio, iba a hacer un
	trabajo determinado, realizase a su vez una tesis doctoral bajo
	su supervisión; y valentía por tener siempre miras altas para
	sí, para el proyecto, y todos nosotros, incluso en los momentos
	en que podía no ser políticamente correcto. Pero por encima de
	eso tengo que agradecer, además, su amistad. Y también es
	mérito suyo que amistad y dirección de tesis nunca hayan estado
	en conflicto. El rigor y calidad de este texto se debe a ella
	en gran medida, mientras que los errores son míos sin
	discusión.
	
	A mi co-director de la tesis, Enrique Solano, tengo que
	agradecerle que hace años se diera cuenta de que el
	Observatorio Virtual representaba el futuro de la astrofísica,
	y que se lanzarse a crear el Observatorio Virtual Español (SVO,
	Spanish Virtual Observatory), sin cuyo apoyo me habría sido
	imposible completar mi formación. Además él ha confiado en mi
	para ser representante del SVO en diferentes órganos, y ha
	apoyado siempre mi trabajo desde su puesto como miembro del
	ejecutivo de IVOA. Y sin lugar a dudas, tengo que agradecerle
	sus valiosas aportaciones para completar esta tesis.
	
	 A José Francisco Gómez, el ser co-supervisor de mi trabajo de
	investigación tutelada, durante el cuál desarrollé la versión
	preliminar del RADAMS. El rigor del RADAMS se debe en buena
	medida a sus aportaciones.
	
	 This work would have not been possible without the work and
	input of many people from the different IVOA working groups:
	thanks to Thomas Boch, François Bonnarel, Mireille Louys (double
	thanks!), Alberto Micol, Pedro Osuna, and Mark Taylor, among
	many others.
	
	 A todo el grupo AMIGA, por ser uno de los grupos más
	simpáticos y acogedores en los que jamás me haya integrado. Y
	es que tengo cosas que agradecer a todos y cada uno de ellos:
	
	\begin{itemize}
		\item A Gilles Bergond, su sentido del humor, y siempre
		estar dispuesto para explicar lo que se le pregunte,
		incluso cuando yo no era más que un recién llegado.
		
		 \item A Dani Espada, sus consejos sobre la tesis, su
		gran capacidad para visualizar y contar todo el proyecto
		AMIGA, y su perenne buen ánimo; he intentado inspirarme
		en esas dos capacidades.
		
		 \item A Víctor Espigares, sus extensos conocimientos del
		NCS del IRAM, de bases de datos, y su talento informático,
		sin el cual no existiría TAPAS. Y cómo no agradecerle su
		ayuda con la preparación de la que a la postre fue mi
		exitosa entrevista de postdoc.
		
		 \item A Emilio García, su guía sobre la literatura de
		modelos de datos de IVOA y el primer gran empujón, pero
		especialmente el permitirme relajarme con una de las cosas
		con las que más disfruto, la divulgación científica, en ese
		magnífico programa suyo que es \emph{A Través del
		Universo}.
		
		 \item A Stéphane Leon, sus invitaciones a café con churros
		en las que acabábamos hablando de trabajo, y también de lo
		que no es trabajo. Y por confiar en que un ingeniero podía
		convertirse en observador radioastronómico,
		proporcionándome uno de los momentos más felices de mi
		vida.
		
		 \item A Ute Lisenfeld, su hospitalidad en múltiples
		ocasiones; y tengo que agradecerle también a su marido su
		paciencia en esas mismas ocasiones, y las sonrisas de sus
		niñas, tímidas al principio, pero con ganas de divertirse a
		costa de los visitantes al final.
		
		 \item A Vicent Martínez, ser siempre una fuente constante
		de ánimo, una enciclopedia musical, que además además es
		capaz ponerse a cantar o tocar lo que le echen.
		
		 \item A Breezy Ocaña, el no ser jamás una \emph{galaxia
		aislada}, sino toda un ejemplo de \emph{galaxia en
		interacción}, disparando la formación de estrellas anímicas…
		Y no hay que olvidar su capacidad de enseñarme todo tipo de
		bailes.
		
		 \item A José Enrique Ruiz, la oportunidad tan singular de
		continuar una amistad de hace 18 años como si el tiempo no
		hubiera pasado, y sin que eso le haya impedido ser
		constructivamente crítico con mi trabajo
		
		 \item A Pepe Sabater, su curiosidad por todo lo que tiene
		que ver con técnicas relacionadas con la astrofísica, y sus
		aportaciones al desarrollo futuro de todos los becarios, de
		las que también me he beneficiado.
		
		 \item A Simon Verley, pese a no haber podido pasar mucho
		tiempo con él, todo el trabajo que ha realizado para su
		tesis, que hace que por comparación cualquier otra parezca
		trivial, y que le profese una enorme admiración.
		
		\item a Susana Sánchez, reciente incorporación a la rama
		e-Científica del grupo, siempre hay que agradecerle su
		perenne sonrisa, y también que velara por que no me
		molestaran con el teléfono en los últimos momentos de la
		tesis. ¡Viva el Palo!
		
		\item a Ana Rejón, ultimísima incorporación formal del
		grupo (aunque se había incorporado antes de corazón), le
		agradezco el cariño que le pone siempre a todo lo que hace,
		y a la relación con los demás miembros del equipo. Y no hay
		que olvidar sus conocimientos de sicología, que me han
		ayudado a superar los momentos de máxima tensión durante la
		escritura de la tesis. Danke schön!
	\end{itemize}
	
	Y cómo no agradecer al insigne Jack Sulentic sus reflexiones
	sobre la astrofísica en general y su relación con la vida
	usando como medio sus degustaciones de vinos. Long live to
	Giacomo Imperatore!
	
	 El resto de compañeros del IAA, becarios, post-docs, y
	personal científico, también han contribuido a hacer de mi
	estancia un paso de lo más agradable. Comencemos por los
	futboleros: Diego Bermejo, Daniel Cabrera,
	\emph{Maligno} Cantero, \emph{Charly} Carrasco, Darío Díaz,
	René Duffard, Javier Gorosabel, Carlos
	Gracia, Jonatan Hernández, Jorge Iglesias, José Luis Jaramillo,
	Martin \emph{Mates} Jelinek, David Martín, Pablo Mellado,
	Daniel Reverte, Miguel Ángel Sánchez, \emph{Wanchope} Suárez,
	Antonio de Ugarte, y algunos más con los que no he podido pasar
	tanto tiempo.
	
	 También ha habido cantidad de compañeros de risas y alegrías:
	Marcos Aparicio, Begoña Ascaso, A.J. Cuesta, Antonio García,
	Maya García, Gabriella Gilli, Inma González, Marta González,
	Omaira González, Luc Jamet, Yolanda Jiménez, Carolinha Kehrig,
	Francisco López, Vicent Martínez, Mar Roca, Cristina Rodríguez,
	Pepe Sabater, Walter Sabolo, Meme Sánchez, Charo Sanz
	(enhorabuena, mamaita). Cuando ha habido momentos no tan
	alegres, vosotros sabéis quiénes han estado ahí.
	
	 E interesantísimas conversaciones de café o sobremesa, con
	Iván Agudo, Víctor Aldaya, Emilio Alfaro, Pedro Amado, Carlos
	Barceló, Miguel Cerviño, Víctor Costa, Rafael Garrido, José
	Luis Jaramillo, Isabel Márquez, Jaime Perea, Enrique Pérez,
	Pepe Vílchez, y muchos más. Mención especial merece Paco
	Navarro: ¡se le saluda, caballero!
	
	 Quisiera destacar a todos los compañeros que han dedicado
	parte de su escaso tiempo libre a realizar labores de
	representación del colectivo de estudiantes predoctorales: Pepe
	Sabater, Daniel Espada, Geli Carballo, Marcos Aparicio
	(enhorabuena, papaito), Antonio García, Meme Sánchez, y Marta
	González (espero no dejarme a nadie). Su trabajo para que no se
	menoscabe nuestra labor de investigación y nuestro aporte a la
	actividad científica del centro como colectivo es fundamental.
	Las Sesiones de Ciencia, Cine y Debate (CCD) del IAA son un
	invento vuestro del que disfruta todo el centro, y que han sido
	posibles gracias a Daniel Guirado, Mar Roca, Alberto Molino y
	Fabio Zandanel, que coordinan o han coordinado dichas
	actividades.
	
	 Y hablando del centro, aprovecho para mandarle un beso a María
	de los Ángeles Cortés, sin la cual creo sinceramente que el IAA
	no funcionaría ni la mitad de bien.
	
	 Una de las actividades que más me ha servido para relajarme,
	aunque no la he podido disfrutar todo lo que habría querido, es
	bailar tango. Tengo que agradecer a William y Carina sus
	excelentes clases; a la gente que tuvo la iniciativa de sacar
	adelante las milongas de jueves y domingos, todas las
	oportunidades de baile; y a todos mis compañeros su paciencia y
	consejos. Un abrazo a Aleida, Ana (todas vosotras), Breezy,
	Carina, Natalia, Silvia y muchas más.
	
	 He agradecido antes a Emilio García el poder participar en
	\emph{A Través del Universo}. Pero el disfrute no habría sido
	el mismo sin Pablo Santos, Ana Tamayo, Felipe Astrologuito, el
	Capitán Kirk, Chewie, el Reportero Urbanita, ni el Astromático
	y Blanquita. Y aunque Silbia López no quiera considerarse parte
	del equipo, le mando un beso porque ella también es muy
	importante. Y a Ana Rejón y Nieves Fiestas también las cuento
	aquí, porque también han tenido sus \emph{apariciones
	estelares}.
	
	 No me quiero olvidar de mi vida pre-científica: sin lo que
	aprendí mientras estaba trabajando para BK Brokers, IMPURSA,
	Trevenque Sistemas de Información y Nadales Libros, tampoco
	estaría escribiendo estos agradecimientos, siendo a estas dos
	últimas compañías a quienes más debo, por diferentes razones.
	Mi agradecimiento a Juan Ramón Olmos de Trevenque Sistemas de
	Información, y a Francisco Martínez de Zócalo Libros, así como
	a mis compañeros de Trevenque Jairo Bolívar, Alejandro Morales,
	José Antonio Vacas, y Antonio Díaz. Y también a alguien que
	tuvo una corta estancia, pero me abrió los ojos al camino de la
	investigación: Manuel Díez-Minguito. Compañeros menos directos,
	pero también memorables, fueron Rafael Maroto, Paco Martínez
	Liñán (qué conversaciones de sobremesa), e incluso Francisco
	Varo (del que aprendí mucho, incluyendo lo que no pretendió
	enseñarme, aunque fue incluso más útil). And my gratitude to
	Mark Cameron and John Weisberg, two people I enjoyed working
	with as we shared the same passion for detail, and for enjoying
	ourselves the few times we were able to meet together.
	
	 También quiero contar en esta parte a Alfonso Tejedor y Carlos
	Burges por prestarme una
	\href{http://www.entremaqueros.com/bitacoras/memoria/}{
	\emph{Memoria de Acceso Aleatorio}}, una voz en Internet, a
	veces válvula de escape, y origen de la difusión en
	\emph{podcast} de \href{http://universo.iaa.es/}{\emph{A Través
	del Universo}}.
	
	 Una parte más formal de estos agradecimientos: tengo que
	reconocer el soporte del Plan Nacional de Astronomía y
	Astrofísica del Ministerio de Ciencia e Innovación, ya que
	directamente a través de sus proyectos AYA2002-03338,
	AYA2005-07516-C02-01 y AYA2008-06181-C02-01 he disfrutado de
	financiación para realizar esta tesis, e indirectamente por la
	financiación a la Red Temática del Observatorio Virtual Español
	(Spanish Virtual Observatory, AYA2008-02156, AYA2005-04286).
	También agradezco al CSIC la concesión de una beca del programa
	I3P durante 2006, y han colaborado en mi formación los
	proyectos con financiación europea EuroVO-DCA (RI031675),
	VOTech (011892-DS), y EuroVO-AIDA (RI2121104). ¡Y cómo no
	agradecer a la Organización Europea de Investigación
	Astronómica en el Hemisferio Sur (ESO) que haya valorado
	positivamente este trabajo, tanto como para contratarme!
	
	 Quisiera terminar dando las gracias a mis amigos de toda la
	vida, a los que no he podido ver tanto como quisiera por
	dedicarle tiempo a esta tesis: Ángel, Fermín, Hortensia, Ilu,
	Isi, JR, Lola (qué niña más linda es Elsa), Rafa, Raquel,
	\emph{Ruly}… Lo único que me apena es que terminar la tesis no
	me va a dar mucho más tiempo para estar con vosotros… pero sí
	espero que algo más, sobre todo si decidís visitarme en aquél
	lugar del mundo donde finalmente acabe. Y le mando un beso
	fortísimo a mi hermano y mis sobrinos Alba y Álex, que no sé si
	llegarán a poder recibirlo.
	
	 Un último mensaje: si al lector de estos agradecimientos le
	parece que me lo pasé demasiado bien escribiendo esta tesis,
	sólo una cosa tengo que decirle: los cuatro años que he
	pasado en este centro han pasado volando, y mi trabajo no
	habría sido ni la mitad de bueno si no me lo hubiera pasado así
	de bien. Y si de alguien me olvidé, sentiré tanta vergüenza
	cuando me lo diga que espero que se pueda considerar castigo
	suficiente.
	
	 \textbf{¡Gracias!}