\usepackage{glossaries}

\makeglossaries

\newglossaryentry{ALMA}
{
    label={ALMA},
    name={ALMA},
    description={
    	the \emph{Atacama Large Millimetre Array} is going to be the
        largest radio interferometer in the world. It will consist of
        more than 56 12-m diameter antennas in different
        configurations, together with a compact array of 8 6-m antennas
        and four more 12-m antennas to allow for smaller baselines
        (below 12m), allowing for much more extended emission to be
        captured. The antennas will be above 5000m in plateau of
        Chajnantor, at the Atacama desert, while the operation support
        facilities will be found just below 3000m.
    }
}

\newglossaryentry{AMIGA}
{
    label={AMIGA},
    name={AMIGA},
    description={
    	bilingual acronym which stands in English for \emph{Analysis of
        the interstellar Medium of Isolated GAlaxies}, and in Spanish
        for \emph{Análisis del Medio Interestelar de Galaxias
        Aisladas}. AMIGA is a research project funded by the Spanish
        National Call for Space and Astrophysics wich started in 2004
        [TODO search this date], whose PI is Lourdes Verdes-Montenegro,
        devoted to the study of the largest optically selected sample
        of isolated galaxies (1050), following strict isolation
        criteria, in order to provide a baseline that can be used to
        differentiate whether particular galactic traits are intrinsic,
        or depend on the interaction with neighbouring galaxies. This
        study uses multi-wavelength information from many different
        telescopes and wave bands, and joined the \gls{SVO} in 2005
        in order to create \gls{VO} tools needed for performing
        AMIGA's work, but also useful for the community at large. AMIGA
        VO actitivies have a strong emphasis on the radio band, both
        because of our interest in \gls{ALMA}, and because of the
        relative under-development of the radio-VO, both in terms of
        number of archives, available tools, and data models.
    }
}

\newglossaryentry{API}
{
    label={API},
    name={API},
    description={
    	acronym standing for \emph{Applications Programming Interface}.
        An API is a set of related functions that allow applications to
        make use of the operating system (or specific functions). For
        instance, in UNIX-like systems, the POSIX API is the one
        allowing file access, memory management, multi-tasking, et cetera.
        Programs written using the POSIX API do not need to implement
        file access, memory management, and so one, but instead they
        will call functions of the POSIX API suitable for those tasks.
    }
}

\newglossaryentry{Applications Working Group}
{
    label={Applications Working Group},
    name={Applications Working Group},
    description={
    	The \gls{IVOA} Applications Working Group is concerned
        primarily with the software tools that astronomers use to
        access VO data and services for doing astronomy. This WG
        provides a means for VO Applications development and
        implementation to be closely linked to the standards
        development in the IVOA, and where necessary proposes and
        develops standards for VO Applications to interoperate. The
        role of the Applications WG is to identify missing or desirable
        technical capabilities for VO applications, as well as
        components in terms of scientific usability. This WG is the
        IVOA front-end for announcing and discussing new VO
        applications and standards for application interoperability.
        The \gls{PLASTIC} protocol was the first attempt at tool
        interoperability, and now the WG is promoting \gls{SAMP},
        which is still a \gls{Working Draft}.
    }
}

\newglossaryentry{ASCII}
{
    label={ASCII},
    name={ASCII},
    description={
    	acronym standing for \emph{American Standard Code for
        Information Interchage}. ASCII is the standard for the
        codification of textual data in binary form. In only specifies
        the first 127 characters out of the 256 that can be encoded
        with one byte (2^8), and it can only properly show characters
        from the US-English alphabet and symbols. Other European
        characters, such as ñ, ç, ß, å, and the like are not covered by
        ASCII, and require additional encoding. As ASCII specifies 8
        bits per character, no single encoding can provide all
        characters from each different language, much less at the same
        time. The present solution for character interoperability is
        \gls{Unicode}.
    }
}

\newglossaryentry{AstroGrid}
{
    label={AstroGrid},
    name={AstroGrid},
    description={
    	UK research program devoted to the exploitation of grid
        capabilities in the astronomical enviroment. Since the creation
        of the Astrophysical Virtual Observatory (AVO) prototypes,
        AstroGrid has focused on \gls{Virtual Observatory}
        interoperability of remote services, under what is called
        Global Worker Services, and the Common Execution Architecture.
        They also develop the VODesktop and the AstroRuntime. The
        latter is a set of functions for VO access that can be called
        either using \gls{XML-RPC} or \gls{Java-RMI} from any
        other application, while the former is a set of VO
        visualisation and data access and filtering tools built on top
        of the \gls{AstroRuntime}.
    }
}

\newglossaryentry{AstroRuntime}
{
    label={AstroRuntime},
    name={AstroRuntime},
    description={
    	the AstroRuntime is an \gls{AstroGrid} subproject that tries
        to provide a complete \gls{\gls{Virtual Observatory}} toolbox to
        any application able to communicate with it. Technically, it is
        a client-side middleware that provides uniform access to all
        \gls{Virtual Observatory} services, from any programming language
        supporting \gls{XML-RPC} calls, direct \gls{HTTP}
        communications, or \gls{Java-RMI} interfaces.
    }
}

\newglossaryentry{Astronomy}
{
    label={Astronomy},
    name={Astronomy},
    description={
    	several dictionaries define \emph{astronomy} as the study of
        the properties of objects and matter outside the Earth's
        atmosphere. Etymologically, comes from the greek
        $\alpha\sigma\tau\epsilon\rho\ipsilon$ (\emph{astro-asteri},
        meaning star), and the greek $\nu\omicron\mu\omicron\sigma$
        (\emph{nomos}, meaning law), and that would be the study of the
        laws governing celestial objects. If those laws are restricted
        to the relative movement of astronomical objects, is considered
        a part of Mathematics. However, when scientists try to derive
        physical or chemical properties from celestial objects, it is
        used interchangeably with \gls{astrophysics}.
    }
}

\newglossaryentry{Astrophysics}
{
    label={Astrophysics},
    name={Astrophysics},
    description={
    	the part of \gls{astronomy} dealing with the acquisition of
        physical information from astronomical objects. Examples of
        such information include object velocities, distances,
        oscillations, temperatures, object dynamics, et cetera.
		However, many
        people use astronomy and astrophysics interchangeably, as today
        it is very difficult to perform non-physical astronomical
        observations, as even when measuring object positions and
        distances physical models are used to enhance the distance
        estimation.
    }
}

\newglossaryentry{Attribute}
{
    label={Attribute},
    name={Attribute},
    description={
    	in \gls{XML}, an attribute is information that qualifies a tag,
        and goes alongside with it. For instance, \texttt{PARAM}
        \gls{tags} in a \gls{VOTable} can have \texttt{ID},
        \texttt{unit}, \texttt{datatype}, \texttt{precision},
        \texttt{width}, \texttt{ref}, \texttt{name}, \texttt{ucd},
        \texttt{utype}, \texttt{arraysize} and \texttt{value}
        attributes. Some attributes might be required for a particular
        tag. For instance, \texttt{PARAM} tags require an \texttt{ID},
        a \texttt{name}, and a \texttt{value}, making the minimum valid
        \texttt{PARAM} tag: \texttt{<PARAM name=``resolution''
        value=``50'' />}. Of course, \texttt{units} and \texttt{ucd}
        are not required attributes (in the sense that the VOTable XML
        schema does not define them as required), but are absolutely
        required to understand the quantity: \texttt{<PARAM
        name=``resolution'' value=``50'' units=``arcsec''
        ucd=``pos.angResolution'' />}).
    }
}

\newglossaryentry{Batch processing}
{
    label={Batch processing},
    name={Batch processing},
    description={
    	Non-interactive processing of a bunch of data (\emph{batch}).
        Typically used for lengthy operations on the same set of data,
        and launched from a \gls{CLI}. Involves programming the series of
        operations, either with a script, or a complete application or
        tool. Batch processing is useful when processing different data
        sets with the same parameters, in order to obtain consistent
        results. Compare with \gls{Interactive session}.
    }
}

\newglossaryentry{Binary-based file formats}
{
    label={Binary-based file formats},
    name={Binary-based file formats},
    description={
    	a binary-based file format is one where the data representation
        closely resembles the in-memory representation for programs
        doing computations with such data. Usually, binary-based file
        formats perform better at reading and writing, because there is
        no need to parse the input. However, binary-based file formats
        suffer from flexibility problems, and scripting tools need to
        decode the data first in order to manipulate them. Compare with
        \gls{text-based file formats}.
    }
}

\newglossaryentry{Browser}
{
    label={Browser},
    name={Browser},
    description={
    	In computer science, a browser is a piece of software that
        allows navigating around, and perusing, a particular kind of
        information, such as \emph{Web browsers} for accessing the
        \gls{World Wide Web}, or \emph{file browsers} for navigating
        around a computer's file system.
    }
}

\newglossaryentry{Character encoding}
{
    label={Character encoding},
    name={Character encoding},
    description={
    	The mapping between a particular sets of bits and a particular
        character printed or shown on screen is called character
        encoding. The best known encoding is \gls{ASCII}, but it fails
        to represent non-US-English characters, as it only specifies
        127 characters out of 256. The ISO provides character encoding
        specifications such as Latin1 for Western European countries,
        Latin2 for cyrillic, and the family of ISO 8859 standards for
        Arabic, Hebrew, Greek, and many other languages. Even different
        manufacturers used different encodings, such as IBM's EDBIBC,
        or Apple's MacRoman. The problem with ASCII based character
        encoding is that each encoding is exclusive: a Latin2 document
        cannot provide Arabic text, nor can Arabic and Hebrew be mixed,
        for instance. The solution consists in breaking the 256
        characters limit, and providing consistent support across
        operating systems, languages, and scripting systems, from an
        international consortium. See \gls{Unicode}.
    }
}

\newglossaryentry{CLI}
{
    label={CLI},
    name={CLI},
    description={
    	acronym for \emph{Command Line Interface}. A CLI is a
        text-based interface, similar to a UNIX shell, where commands
        are entered with a keyboard, and parameters are entered
        together with the command for batch processing sessions, or
        after entering the command for interactive processing sessions.
        Compare with \gls{GUI}.
    }
}

\newglossaryentry{CSIC}
{
    label={CSIC},
    name={CSIC},
    description={
    	Spanish acronym standing for \emph{Consejo Superior de
        Investigaciones Científicas}, or \emph{High Council of
        Scientific Research} in English. CSIC is the largest research
        organisation in Spain, and the \gls{IAA} belongs to it.
    }
}

\newglossaryentry{CSS}
{
    label={CSS},
    name={CSS},
    description={
    	acronym standing for \emph{Cascading Style Sheets}. CSS is a
        \gls{W3C} standard to define presentation by means of a
        hierarchical cascade of style sheets —similar to styles in word
        processing packages—, so that a given \gls{HTML}/\gls{XHTML}
        document content can show different appearances depending on
        the style sheet being applied.
    }
}

\newglossaryentry{CSV}
{
    label={CSV},
    name={CSV},
    description={
    	A \gls{text-based file format} (compare with \gls{binary file
        formats}) where data records are separated in different rows
        (i.e., the text line separator acts as record separator), and
        data belonging to different columns are separated by commas
        (CSV stands for \emph{Comma Separated Values}). Optionally, the
        last row of comments prior to the first row of data can contain
        table headers. Compare with \gls{TST}.
    }
}

\newglossaryentry{Data Access Layer Working Group}
{
    label={Data Access Layer Working Group},
    name={Data Access Layer Working Group},
    description={
    	The task of the Data Access Layer (DAL) working group is to
        define and formulate VO standards for remote data access.
        Client data analysis software uses these services to access
        data via the VO framework; data providers implement these
        services to publish data to the VO. The DAL working group will
        define the scope of the DAL standards, outline a process by
        which DAL standards are defined, and generate the initial
        version 1.0 of the DAL standard. This standard will provide
        guidance to data centres and survey projects when designing VO
        compliant interfaces. It will allow them to justify the
        allocation of resources for its implementation and maintenance.
        Once the work on Version 1.0 is accomplished the working group
        will coordinate future development of the standard. Currently,
        one \gls{IVOA} \gls{Recommendation} from this WG exists:
        \gls{SSAP}, and \gls{SCS}. \gls{SIAP} should follow soon, but
        it is still a \gls{Working Draft}.
    }
}

\newglossaryentry{Data Modelling Working Group}
{
    label={Data Modelling Working Group},
    name={Data Modelling Working Group},
    description={
    	The role of the Data Modeling working group (WG) is to provide
        a framework for the description of \gls{metadata} attached to
        observed or simulated data. The WG activity focuses on logical
        relationships between these metadata, examining how an
        astronomer wants to retrieve, process, and interpret
        astronomical data, and providing an architecture to handle
        them. This WG standards can then be re-used in the protocols
        defined by the \gls{Data Access Layer Working Group} or in
        VO-aware applications.
    }
}

\newglossaryentry{DOI}
{
    label={DOI},
    name={DOI},
    description={
    	Acronym for \emph{Digital Object Identifier}. A DOI consists of
        a permanent identifier given to an electronic document that, in
        contrast to a \gls{URL}, is not dependent upon the electronic
        document's location. A DOI has to be resolved by the
        [http://doi.org/ DOI System], which provides both the metadata
        associated with the document, and a known location for an
        electronic copy. DOIs are particular cases of the CNRI Handle
        System, the first system for resolving digital document codes
        into URLs. DOIs are CNRI Handles with prefix \texttt{10}.
        Compare with \gls{URN}s.
    }
}

\newglossaryentry{Driver}
{
    label={Driver},
    name={Driver},
    description={
    	In computer technology, a driver is a piece of software in
        charge of talking to classes of devices, so that an application
        only has to talk to the class of devices, and the difference in
        detail between different devices is abstracted by the driver.
        For instance, \gls{JDBC} drivers provide database connectivity
        for applications written in \gls{Java}, and hide the
        implementation differences between different database vendors,
        such as MySQL or PostgreSQL.
    }
}

\newglossaryentry{DTD}
{
    label={DTD},
    name={DTD},
    description={
    	acronym standing for \emph{Document Type Definition}. DTD is a
        non-XML-based purpose-specific language for the definition of
        constraints for \gls{XML} documents, also known as \gls{XML
        schema}. A DTD document specifies the \gls{tags} that can be
        used in documents using a particular DTD, tag \gls{attributes},
        tag values, and the order, repetition, and hierarchical
        relationship between tags. Compare with \gls{XML Schema}. (Mind
        the capitalisation.)
    }
}

\newglossaryentry{Dublin Core}
{
    label={Dublin Core},
    name={Dublin Core},
    description={
    	set of \gls{metadata} for archive resources shared by all
        registries which are \gls{OAI}-compliant. The Dublin Core
        mandates metadata keywords for resource titles, creators,
        subject, description, publishers, contributors, publishing
        data, resource type, resource format, resource identifiers,
        resource sources, language, relationships with other resources,
        resource coverage, and resource-related rights.
    }
}

\newglossaryentry{End-point}
{
    label={End-point},
    name={End-point},
    description={
    	in \gls{web services} parlance, an end-point is an \gls{URL} to
        which \gls{REST} or \gls{SOAP} conforming queries can be
        performed. End-points for web services are usually defined via
        \gls{WSDL} documents, whereas for \gls{VO} services end-points
        are defined in the VO \gls{Registry}.
    }
}

\newglossaryentry{Euro3D}
{
    label={Euro3D},
    name={Euro3D},
    description={
    	\gls{FITS}-based data format developed within the \gls{OPTICON}
        network for storing 3D spectra coming from \gls{IFU}s. It can
        be made general enough to hold data coming from radio
        interferometry data, but the overhead becomes enormous compared
        to more traditional FITS-based ways of storing this data.
    }
}

\newglossaryentry{FITS}
{
    label={FITS},
    name={FITS},
    description={
    	acronym standing for \emph{Flexible Image Transport System}.
        FITS is a file format standard developed by NASA in the '70s,
        which allows for metadata headers (\emph{cards}) of the form
        \texttt{KEYWORD = VALUE}, and arbitrary shaped binary tables
        for which column number and column descriptions exist in the
        headers. FITS keywords are made of at most 8 ASCII characters,
        and the whole keyword, equal separator (\texttt{=}), and value,
        cannot be more than 79 characters long. FITS files are
        organised around HDUs (Header Data Unit), described by the
        primary FITS header, which in turn can contain additional
        headers and HDUs. Only a few FITS headers are standardised.
    }
}

\newglossaryentry{FORTRAN}
{
    label={FORTRAN},
    name={FORTRAN},
    description={
    	acronym for \emph{FORmula TRANslator}. FORTRAN is a programming
        language invented by John Backus in the '50s for IBM, and has
        an special emphasis on matrix computations, for which is best
        suited. This native support of vectors, arrays, and matrices
        made FORTRAN the most popular programming language in the
        scientific world. As many of the astrophysical and
        astrometrical calculations make extensive use of this kind of
        processing, FORTRAN soon became the preferred programming
        language in astrophysics.
    }
}

\newglossaryentry{GDL}
{
    label={GDL},
    name={GDL},
    description={
    	acronym for \emph{GNU Data Language}. GDL is an
        \gls{open-source} clone of the \gls{IDL} language, without the
        IDL libraries. It can be used to perform data analysis
        operations using IDL compatible libraries, such as the \gls{IDL
        Astronomy Library}.
    }
}

\newglossaryentry{Grid and Web Services Working Group}
{
    label={Grid and Web Services Working Group},
    name={Grid and Web Services Working Group},
    description={
    	The aim of the \gls{IVOA} Grid & Web Services Working Group is
        to define the use of \gls{grid} technologies and \gls{web
        services} within the VO context and to investigate, specify,
        and implement required standards in this area. This group was
        formed from a merger of the previously existing Web Services
        group and the Grid group, as ordered at the IAU General
        Assembly in 2003.
    }
}

\newglossaryentry{Grid}
{
	label={Grid},
	name={Grid (computing)},
	description={
		Grid computing is a form of distributed computing, where
        computation and/or storage nodes from a pool of geographically
        distributed resources can be used to perform tasks in parallel,
        with more or less interaction between nodes, and which use
        common \gls{middleware} that provides the resource allocation
        \gls{API}s to grid computing applications. A grid can be
        functionally defined as a system of computers without a
        centralised administration, which communicate by means of open
        standards, and which provide a quality of service beyond
        \emph{best effort} (i.e., programs in the grid are launched on
        suitable nodes, abnormal program termination events are logged
        and execution retried, and needed execution times can be
        specified).
	}
}

\newglossaryentry{GUI}
{
    label={GUI},
    name={GUI},
    description={
    	Acronym for \emph{Graphical User Interface}. A GUI is a
        graphical way to show data on a computer screen, and perform
        user interaction with graphical devices, as mouse-controlled
        pointers for interacting with on-screen buttons. GUIs are also
        useful for \gls{interactive sessions}, such as interactive
        visualisations where the graphical representation can change
        interactively following user inputs. Example of GUI
        environments are the X-Window environment for general computing
        tasks, the IDLDE, or IRAF's graphical xgterm windows. Compare
        with \gls{CLI}.
    }
}

\newglossaryentry{HTML}
{
    label={HTML},
    name={HTML},
    description={
    	The \emph{Hyper Text Markup Language} is the main invention of
        Tim Berners-Lee, and is the cornerstone of the \gls{World Wide
        Web}. HTML uses \gls{tags} to markup (i.e., to label) text
        content, specifying both content, links, and presentation. The
        latest HTML versions, and \gls{XHTML}, try to provide content
        with markup that provides semantics to the content, while the
        presentation is left to CSS.
    }
}

\newglossaryentry{HTTP}
{
    label={HTTP},
    name={HTTP},
    description={
    	The \emph{Hyper Text Transfer Protocol} is the other invention
        and cornerstone of the \gls{World Wide Wed}, and was also
        invented by Tim Berners-Lee. HTTP initially was a protocol for
        hypertext data retrieval, allowing for GET operations on
        \gls{URL}s pointing to HTML documents, or hypermedia elements
        pointed by the HTML documents such as images, sounds, videos,
        et cetera. It was later upgraded to allow information transport from
        the browser to the server, via HTTP POST calls which sent
        information from web forms, and additional PUT and DELETE calls
        for better semantic support of remote operations. This protocol
        is also used for the query of web services, which must provide
        an HTTP server.
    }
}
\newglossaryentry{Hypermedia}
{
    label={Hypermedia},
    name={Hypermedia},
    description={
    	See \gls{hypertext}.
    }
}

\newglossaryentry{Hypertext}
{
    label={Hypertext},
    name={Hypertext},
    description={
    	term coined by Ted Nelson in 1965, at the same time as
        hypermedia, when developing a collaborative document editing
        that allowed text and media insertion and labelling, with links
        between document sections in the same or another document. Ted
        Nelson work can be seen as an implementation of Vannevar Bush
        vision as expressed in his 1945 essay \emph{As we may think}.
        For instance, in a text dealing with web services, links may
        exist to terms such as \gls{HTTP}, \gls{XML}, \gls{REST},
        \gls{SOAP}, \gls{WSDL}, et cetera,
		which would be placed exactly where
        those terms appeared. Such a \emph{link-enriched} text is
        called hypertext. If the system goes beyond text to use other
        kinds of media (graphics, images, sounds, video, et cetera) the term
        hypermedia is used.
    }
}

\newglossaryentry{IAA}
{
    label={IAA},
    name={IAA},
    description={
    	Spanish acronym standing for \emph{Instituto de Astrofísica de
        Andalucía}, or \emph{Institute of Astrophysics of Andalusia}.
        Usually written as IAA-CSIC, to show the IAA is an institute
        belonging to the \gls{CSIC}.
    }
}

\newglossaryentry{IAU}
{
    label={IAU},
    name={IAU},
    description={
    	The \emph{International Astronomy Union} was founded in 1919.
        Its mission is to promote and safeguard the science of
        astronomy in all its aspects through international cooperation.
        Its individual members are professional astronomers all over
        the world, at the Ph.D. level and beyond, and active in
        professional research and education in astronomy. Besides, the
        IAU maintains friendly relations with organisations that
        include amateur astronomers in their membership.
        http://www.iau.org/
    }
}

\newglossaryentry{IDL}
{
    label={IDL},
    name={IDL},
    description={
    	acronym standing for \emph{Interactive Data Language}. The IDL
        is a language and \gls{CLI} environment for performing
        mathematical operations, with an emphasis on matrix algebra, on
        \gls{interactive sessions}. IDL has a strong share of computing
        the astronomy and astrophysics world due to its similarity to
        an interactive version of FORTRAN, and to the huge \gls{IDL
        Astronomy Library} providing packages for many astronomical and
        astrophysical \gls{reduction} duties. IDL is not open software,
        and initially belonged to Research Systems Inc. (RSI), and now
        belongs to ITT Industries and operates as ITT Visual
        Information Solutions). However, the Astronomy Library is
        \gls{open-source} software, and can be used with clones such as
        \gls{GDL}.
    }
}

\newglossaryentry{IEC}
{
    label={IEC},
    name={IEC},
    description={
    	acronym for the \emph{International Electrotechnical
        Commission}, a non-profit international standards organisation
        that prepares and publishes International Standards for all
        electrical, electronic, and related technologies. IEC also
        manages conformity assessment, in order to certify that
        equipment complies with its standards.
    }
}

\newglossaryentry{IETF}
{
    label={IETF},
    name={IETF},
    description={
    	The \emph{Internet Engineering Task Force} develops and
        promotes Internet standards, cooperating closely with the
        \gls{W3C} and \gls{ISO}/\gls{IEC} standard bodies and dealing
        in particular with standards of the \gls{Internet}
        \gls{protocol suite}. It is an open standards organisation,
        with no formal membership or membership requirements.
        \gls{IVOA}'s organisation is loosely modelled after the IETF's,
        as their goals a very similar: promote open, interoperable
        standards to be used throughout the Internet.
    }
}

\newglossaryentry{INTA}
{
    label={INTA},
    name={INTA},
    description={
    	Spanish acronym standing for \emph{Instituto Nacional de
        Técnica Aeroespacial}, or \emph{National Institute for
        Aerospatial Technologies} in English. INTA is the organisation
        funding the \gls{LAEFF}, and ensures continued funding both for
        LAEFF and the \gls{SVO}.
    }
}

\newglossaryentry{Interactive session}
{
    label={Interactive session},
    name={Interactive session},
    description={
    	Session of data processing which is performed in an environment
        that uses simple steps that return control to the user as soon
        as they have finished. Some of the steps might imply \gls{batch
        processing}, but after each step is finished control is
        returned to the user. The user has control of a particular
        environment, where data resides, and can directly make change
        to the data. Some \gls{CLI} environments for interactive
        processing are \gls{IRAF}, \gls{IDL}, \gls{Python}, \gls{shell}
        command lines, \gls{MATLAB}, et cetera.
		Typically, \gls{GUI}s are also
        useful for interactive sessions.
    }
}

\newglossaryentry{IRAF}
{
    label={IRAF},
    name={IRAF},
    description={
    	acronym standing for \emph{Image Reduction and Analysis
        Facility}. IRAF is a huge collection of software modules
        written by astronomers and programmers at the \gls{NOAO},
        needed for the \gls{reduction} of astronomical images provided
        as pixel matrices, such as those provided by imaging array
        detectors like CCDs. IRAF provides a \gls{CLI} that handles the
        loading of such images, mathematical operations on them such as
        image subtraction or division, with specific analysis tasks to
        extract scientific products such as 1D spectra, 2D spectra,
        pixel counts, etc, during \gls{interactive sessions}. Scripts
        can be written to perform \gls{batch processing}.
    }
}

\newglossaryentry{ISO}
{
    label={ISO},
    name={ISO},
    description={
    	short-name of the \emph{International Organisation for
        Standardisation}, an international-standard-setting body
        composed of representatives from various national standards
        organisations. The form \emph{ISO} is used to avoid changing
        the organisation name in different languages, and comes from
        the Greek $\ipsilon\sigma\omicron\psi$ (\emph{isos}), meaning
        \emph{equal}. The ISO sanctions international standards. In the
        case of standards dealing with information technologies, there
        is a Joint Technical Committee with \gls{IEC}, and standards
        coming from that joint committee are known as ISO/IEC
        standards.
    }
}

\newglossaryentry{IVOA}
{
    label={IVOA},
    name={IVOA},
    description={
    	acronym standing for \emph{International Virtual Observatory
        Alliance}. An alliance of different national \gls{Virtual
        Observatory} initiatives, formed in June 2002, whose mission is
        to facilitate the international co-ordination and collaboration
        necessary for enabling global and integrated access to data
        gathered by astronomical observatories. The IVOA Executive is
        present at the \gls{IAU} via the Virtual Observatory Working
        Group. http://ivoa.net/
    }
}

\newglossaryentry{IVORN}
{
    label={IVORN},
    name={IVORN},
    description={
    	acronym standing for \emph{International Virtual Observatory
        Resource Name}. An IVORN is a kind of \gls{URN} with protocol
        \texttt{ivo:}, that encodes both an \emph{Authority ID} and a
        \emph{Resource Key} within that particular authority. The
        encoding is such that the URN is of the form
        \texttt{ivo://AuthorityID/ResourceKey}. An additional parameter
        can be specified for particular, local uses, after either a
        question mark (\texttt{?}) or a number mark (\texttt{#}). Thus,
        the URN \texttt{ivo://AuthorityID/ResourceKey#temp1}
        corresponds to an IVORN (protocol \texttt{ivo://}) for a
        \texttt{ResourceKey} resource under \texttt{AuthorityID}, and
        #temp1 is a particular reference of local value (for instance,
        in order to specify message dates). IVORNs are an \gls{IVOA
        Recommendation}. \emph{Authority ID}s must be registered as
        \gls{VO Resource}s of type \emph{identity}.
        http://www.ivoa.net/Documents/latest/IDs.html
    }
}

\newglossaryentry{Java}
{
    label={Java},
    name={Java},
    description={
    	Platform-independent programming language developed by Sun
        Microsystems with the aim of writing applications that can work
        across different computing platforms by means of a \gls{Virtual
        Machine} that executes compiled Java byte-code, and isolates
        the differences between operating systems. In the context of
        the \gls{Virtual Observatory}, many tools are being written in
        Java so that they can reach the maximum number of potential
        users, regardless of platform. http://www.java.com/
    }
}

\newglossaryentry{Java-RMI}
{
    label={Java-RMI},
    name={Java-RMI},
    description={
    	short for \emph{Java Remote Method Invocation}, Java-RMI is an
        \gls{API} that provides \gls{RPC} capabilities within the
        \gls{Java} object model. In other words, it allows Java objects
        remotely call methods from objects in other packages, without
        the burden of setting up a whole RPC server, because Java-RMI
        is embedded in the language.
    }
}

\newglossaryentry{JDBC}
{
    label={JDBC},
    name={JDBC},
    description={
    	\emph{Java Data Base Connectivity} is a standard way of
        specifying connections to \gls{SQL}-compliant databases, so
        that an application written in \gls{Java} uses the same
        database query source code, and only the configuration of the
        JDBC driver is different.
    }
}

\newglossaryentry{LAEFF}
{
    label={LAEFF},
    name={LAEFF},
    description={
    	Spanish acronym for \emph{Laboratorio de Astrofísica Espacial y
        Física Fundamental}, or \emph{Spatial Astrophysics and
        Fundamental Physics Laboratory} in English. LAEFF is the seed
        of the \gls{SVO}, and was the first organisation to develop a
        \gls{VO} compliant archive (INES) for the newly extracted
        spectra from the IUE mission. LAEFF belongs to \gls{INTA}.
    }
}

\newglossaryentry{Markup}
{
    label={Markup},
    name={Markup},
    description={
    	As a noun, refers to textual labels (\gls{tags}) that are
        applied to parts of data to specify either different visual
        representations, particular data semantics, or classification
        within a hierarchy. Examples of markup are \gls{XHTML} tags,
        which either specify visual properties (i.e, \texttt{<b>} for
        boldface, or \texttt{<i>} for italics), hierarchical
        relationships (i.e., \texttt{<h1>} to \texttt{<h6>} to
        represent headers within a document outline), or semantics
        (again, \texttt{<hn>} to represent headers, \texttt{<ol>} to
        represent ordered lists or dictionary entries, or
        \texttt{<abbrv>} for abbreviations). In some languages, markup
        is used to mark the start and end of applicability of the tags,
        while in others they only mark the start, and the end is
        implicit in the language, such as LaTeX' section hierarchy
        using \texttt{\part}, \texttt{\chapter}, \texttt{\section} and
        related commands.
    }
}

\newglossaryentry{Metadata}
{
    label={Metadata},
    name={Metadata},
    description={
    	data about the data, or additional information describing the
        actual data. For instance, in a \gls{FITS} file, headers
        provide metadata via several keywords, in order to specify
        telescope (\texttt{TELESCOP}), instrument (\texttt{INSTRUME}),
        date of observation (\texttt{DATEOBS}), et cetera. Metadata is
        usually not needed if the \emph{system} performing the
        observation is the same as the one performing the reduction, as
        data \gls{reduction} procedures can be built with direct
        knowledge of the observing conditions. However, if data are to
        be treated by a different system, they need the additional
        metadata so that all system configuration is made explicit.
        Metadata scope can be just a single datum, a data table, a data
        file, or a complete file collection. In the \gls{VO}, metadata
        is split into several data models, which specify what kinds of
        metadata are needed, for instance, for observing target
        specification, astronomical data characterisation in different
        domains, et cetera.
    }
}

\newglossaryentry{Method}
{
    label={Method},
    name={Method},
    description={
    	Part of an objects that contains code to perform a function.
        See \gls{Property}.
    }
}

\newglossaryentry{Middleware}
{
    label={Middleware},
    name={Middleware},
    description={
    	is the software layer that lies between the operating system
        and applications on each side of a distributed computing system
        in a network. In the case of \gls{grid computing}, middleware
        manages access to remote storage elements, the replication of
        local archives into the grid storage pool, job submission and
        control, et cetera.
    }
}

\newglossaryentry{Model}
{
    label={Model},
    name={Model},
    description={
    	See \gls{MVC}.
    }
}

\newglossaryentry{Moore's Law}
{
    label={Moore's Law},
    name={Moore's Law},
    description={
    	Named after Intel's engineer Gordon Moore, this \emph{law}
        should be better known as \emph{Moore's Observation}. Moore
        observed the number of transistors that different integrated
        chips fabrication facilities where able to integrate on a
        single chip on different moments since the invention of the
        integrated circuit, and saw that they fitted an exponential
        law, where the number of transistors per chip doubled each
        year. The exponential increase in integrated transistors is
        followed independently by different equipment at different
        rates: computer memories show the highest transistor densities,
        doubling every [TODO find rate], while microprocessors double
        every 18-24 months.
    }
}

\newglossaryentry{MVC}
{
    label={MVC},
    name={MVC},
    description={
    	acronym standing for \emph{Model-View-Controller}, a
        programming paradigm that advocates the separation between
        \gls{Models} (the internal data representation), \gls{Views}
        (the way the information gets to the user, and the user
        interface users can act upon), and \gls{Controllers} (the
        intermediate layer between Model and Views.
    }
}

\newglossaryentry{NOAO}
{
    label={NOAO},
    name={NOAO},
    description={
    	Acronym standing for \emph{National Optical Astronomy
        Observatory}, organisation for the management of the US
        National Solar Observatory, the Inter-American Observatory at
        Cerro Tololo, and the Kitt Peak National Observatory, together
        with the US participation in the Gemini Observatory. The NOAO
        is responsible for astronomical archives for all managed
        observatories.
    }
}

\newglossaryentry{OAI}
{
    label={OAI},
    name={OAI},
    description={
    	acronym standing for \gls{Open Archives Initiative}.
    }
}

\newglossaryentry{Object (computing)}
{
    label={Object (computing)},
    name={Object (computing)},
    description={
    	an object, in computer science, is either a blueprint for
        building useful computer constructs which tie together
        operations and data on which those operations are to be
        performed, or an instance of such computer construct built
        following that blueprint. Objects have properties and methods,
        and both properties and methods can only be used by the same
        object, by other objects of the same blueprint, by other
        objects which have parts of their blueprint in common, or by
        any other object, depending on the scope we have defined both
        for the object and its particular properties and methods.
    }
}

\newglossaryentry{Open Archives Initiative}
{
    label={Open Archives Initiative},
    name={Open Archives Initiative},
    description={
    	a non-profit organisation which develops and promotes
        interoperability standards that aim to facilitate the efficient
        dissemination of content. It has roots in the Open Access
        Movement, but over time its role has expanded to promote broad
        access to digital resources of all kinds for \gls{e-Learning}
        and \gls{e-Science}. The VO \gls{Registry} uses OAI's
        harvesting protocol (OAI-PMH) for data interchange between
        Registries.
    }
}

\newglossaryentry{OPTICON}
{
    label={OPTICON},
    name={OPTICON},
    description={
    	acronym standing for \emph{OPTical and Infrarred COordination
        Network}. OPTICON's objective is to standardise astronomical
        procedures (regarding both data acquisition and manipulation,
        including software packages) as much as possible between the
        \gls{optical} and the \gls{IR} bands. OPTICON funding comes
        from European Commission's Framework Programmes. [TODO review
        this is correct]
    }
}

\newglossaryentry{Parser}
{
    label={Parser},
    name={Parser},
    description={
    	part of program code devoted to interpret (parse) user or file
        input. Parsers differ for different applications, but they
        usually cope with finding data delimiters, converting text
        representations into machine representations, and giving access
        to particular pieces of data.
    }
}

\newglossaryentry{Pipeline}
{
    label={Pipeline},
    name={Pipeline},
    description={
    	In software engineering, a pipeline consists of a chain of
        processing elements (processes, threads, coroutines, et cetera),
        arranged so that the output of each element is the input of the
        next. In astronomy, a pipeline is the chain of processes that
        need to be performed on an astronomical observation in order to
        derive one or more scientific products.
    }
}

\newglossaryentry{PlasKit}
{
    label={PlasKit},
    name={PlasKit},
    description={
    	implementation written in \gls{Java} of the \gls{PLASTIC}
        protocol developed by Mark Taylor for his \gls{TOPCAT}
        application. It implementsts both the PLASTIC hub, and PLASTIC
        client classes.
    }
}

\newglossaryentry{PLASTIC}
{
    label={PLASTIC},
    name={PLASTIC},
    description={
    	Acronym of \emph{PLatform for AStronomical Tool
        InterConnection}, a protocol for communication between
        client-side \gls{Virtual Observatory} astronomy applications.
        Applications use PLASTIC to perform tasks such as instruct
        other PLASTIC-enabled applications to load \gls{VOTables},
        highlight a subset of rows or load an image of a particular
        area of sky. PLASTIC allows a modular approach to choosing VO
        applications, where each application is used for the task it
        performs best. There are two kinds of PLASTIC applications:
        Application hubs, and client applications. Client applications
        register themselves with the PLASTIC hub, and use the hub to
        query for other registered applications, and the PLASTIC
        services those applications provide.
    }
}

\newglossaryentry{Product\gls{, }Data Product\gls{ or }Scientific Product}
{
    label={Product\gls{, }Data Product\gls{ or }Scientific Product},
    name={Product\gls{, }Data Product\gls{ or }Scientific Product},
    description={
    	a processed observation product in the form of an image,
        spectrum, set of spectra, flux measurement, et cetera, that it
        is calibrated and compensated for all instrumental and
        observational effects. Synonym of \emph{reduced data}.
    }
}

\newglossaryentry{Property}
{
    label={Property},
    name={Property},
    description={
    	Part of an object devoted to store pieces of information. Those
        properties are accessible to object methods, and can be made
        invisible to other objects, so that particular object method's
        can perform their tasks based on their properties, while
        (usually) not accessing other object properties. In \gls{XML},
        the names property and attribute are used ambiguosly.
    }
}

\newglossaryentry{Proposed Recommendation}
{
    label={Proposed Recommendation},
    name={Proposed Recommendation},
    description={
    	Within \gls{IVOA}, a Proposed Documentation is a document that,
        after starting as a Working Draft, has received sufficient
        input from the \gls{Working Group} (WG), possibly with a
        WG-wide \gls{RFC} period, and has not suffered a significant
        revision after the RFC.
    }
}

\newglossaryentry{Protocol}
{
    label={Protocol},
    name={Protocol},
    description={
    	Set of standard rules for data representation, signalling,
        authentication, and error detection required to send and
        retrieve information over a communications channel. In the
        context of the \gls{IVOA}, a protocol regulates how to retrieve
        information from a \gls{Virtual Observatory} service, and depends on
        the kind of information being provided. Astronomical images are
        available through servers conforming to the \gls{SIAP}
        protocol, whereas the spectral information is made available
        through the \gls{SSAP} protocol.
    }
}

\newglossaryentry{Python}
{
    label={Python},
    name={Python},
    description={
    	Name of a computer programming and scripting language,
        developed by Guido van Rossum in 1991, and available under an
        \gls{Open Source} license for many different computer
        platforms. Python programs are compiled at execution time, and
        the compiled code is executed by the interpreter. Python has
        been widely adopted in the astronomical community, and its
        \gls{XML-RPC} and \gls{web services} capabilities allow for
        interoperation with a large basis of services written in any
        language that supports \gls{XML-RPC}, and with \gls{Virtual
        Observatory} web services, making it a suitable candidate for
        interoperation with Java and the Virtual Observatory.
    }
}

\newglossaryentry{Recommendation}
{
    label={Recommendation},
    name={Recommendation},
    description={
    	Within \gls{IVOA}, a document which is in the Recommendation
        state has already reached the \gls{Proposed Recommendation}
        stage, has passed an IVOA-wide four weeks RFC period without
        significant revisions, the Chair of the corresponding
        \gls{Working Group} has asked the IVOA Chair to promote the
        document, and the \gls{IVOA Executive} has sanctioned the
        document to \gls{Recommendation}. If the Recommendation is
        relevant enough, the IVOA Executive might send the document to
        the joint IAU VO Working Group to obtain the endorsement of the
        \gls{IAU}, and reach the state of \gls{IAU Standard}.
    }
}

\newglossaryentry{Reduction}
{
    label={Reduction},
    name={Reduction},
    description={
    	in astrophysics parlance \emph{reduction}, or more precisely
        data reduction, refers to the operations needed to convert raw
        data coming from astrophysical instruments into scientific
        products. For instance, the raw information coming from a radio
        spectrograph is the form of photon counts in different
        frequency channels. The astrophysicist also needs the usual
        photon count on the detector when the instrument observes an
        empty patch of the sky; the opacity of the atmosphere at
        different angles to correct for atmospheric extinction at the
        elevation of the source; the average photon count from a
        calibration source, so that received photon fluxes can be
        converted into photon fluxes emitted by the source; and the
        Doppler-effect correction due to the relative movements between
        the source and the observatory. The \emph{reduced} scientific
        products are flux and wavelength calibrated spectra, integrated
        flux within a waveband, and/or the recession velocity of an
        object relative to the Earth, our solar system, or our own
        galaxy.
    }
}

\newglossaryentry{Registry}
{
    label={Registry},
    name={Registry},
    description={
    	In the \gls{Virtual Observatory} architecture, a Registry holds
        resource \gls{metadata} (in the form of a \gls{VOResource}
        record) for the different services available within the VO,
        allowing astronomers to locate, get details of, and make use
        of, any resource located anywhere. There is no \emph{prefered}
        IVOA registry, but instead registries are interoperable,
        allowing the \gls{harvesting} of records from one registry to
        another, so that eventually all registered services can be
        found in any other registry. Registries conform to the
        \gls{OAI} specifications, so that existing Registries for other
        digital repositories have been enhanded to be able to cope with
        VO records. This also means that VO registry records, VO
        Resources, use the \gls{Dublin Core} metadata for maximum
        interoperability.
    }
}
 
\newglossaryentry{REST}
{
    label={REST},
    name={REST},
    description={
    	Computer acronym for \emph{REpresentational State Transfer}.
        REST is a collection of network architecture principles, first
        enunciated by Roy Fielding in his Ph.D. thesis, which outline
        how to define and address resources. REST architectures do not
        rely on \gls{HTTP} to keep the state of an interaction.
        Instead, the state is encoded in the service invocation, both
        in the kind of HTTP operation to be performed (GET for
        completely stateless calls, such as retrieving information from
        a particular user or resource; POST for operations that update
        existing resources; PUT for operations which add new resources;
        DELETE for operations which remove records), and a URL
        containing all information needed for resource identification,
        and additional data for resource filtering (GET) or updates
        (POST, PUT, DELETE). Both GET and POST operations, in an
        architecture following REST principles, can be repeated as many
        times as desired without side effects. PUT operations should
        also be called additional times, and the receiver has to be
        able to identify identical requests not as new resources, but
        resource modifications. And DELETE operations should provide an
        error when called more than once.
    }
}

\newglossaryentry{RFC}
{
    label={RFC},
    name={RFC},
    description={
    	acronym standing for \emph{Request For Comments}. In \gls{W3C}
        parlance, an RFC is a document that states the current best
        practices, or recommended behaviour, for applications,
        protocols, and other interoperation tools, as well as best
        practices for the development of new practices. In the
        \gls{IVOA}, an RFC is issued for a document in \gls{IVOA
        Working Draft} status prior to become an \gls{IVOA Proposed
        Recommendation}.
    }
}

\newglossaryentry{RPC}
{
    label={RPC},
    name={RPC},
    description={
    	acronym of \emph{Remote Procedure Call}. RPC is an
        inter-process communication technology that allows a computer
        program to cause a subroutine or procedure to execute in
        another environment (commonly on another computer on a shared
        network, or in another running application). In languages
        supporting RPC, the programmer does not have to specify the
        details for this remote interaction. More modern flavours of
        RPC are \gls{XML-RPC}, and \gls{Java-RMI}.
    }
}

\newglossaryentry{SAMP}
{
    label={SAMP},
    name={SAMP},
    description={
    	\emph{Simple Application Messaging Protocol}, a local messaging
        protocol intented to replace \gls{PLASTIC}, in order to avoid
        programming-language dependencies (parts of PLASTIC depended
        upon the Java specific RPC, Java-RMI), and to make protocol
        semantics richer and simpler. SAMP hubs exist in \gls{Perl},
        \gls{Python} and \gls{Java} in alpha versions, as well as
        clients, whereas true PLASTIC hubs only existed in Java. See
        \gls{PLASTIC} for comparison.
    }
}

\newglossaryentry{SCS}
{
    label={SCS},
    name={SCS},
    description={
    	acronym standing for \emph{Simple Cone Search}, an \gls{IVOA}
        \gls{Recommendation} from the \gls{Data Access Layer}
        \gls{Working Group} that specifies how to perform positional
        queries catalogues of astronomical sources within a prescribed
        search radius around a central position, hence the term
        \emph{Cone Search}.
    }
}

\newglossaryentry{SDSS}
{
    label={SDSS},
    name={SDSS},
    description={
    	is the acronym for the \emph{Sloan Digital Sky Survey}, a major
        multi-filter imaging and spectroscopic survey using a dedicated
        2.5-m wide-angle optical telescope at Apache Point Observatory
        in New Mexico. The project was named after the Alfred P. Sloan
        Foundation. The survey was begun in 2000, and aims to map 25%
        of the sky with wide-angle observations of around 6 square
        degrees in five filters (ugriz). The observations are processed
        through a dedicated pipeline that differentiates between stars,
        galaxies, and other kinds of objects, and is finally expected
        to provide a catalogue of 100 million objects, and spectra for
        1 million objects. The processing includes high-precision
        redshift determination from the spectroscopic observations, and
        more coarse photometric redshift estimation from the source
        intensity in the different filters.
    }
}

\newglossaryentry{SED}
{
    label={SED},
    name={SED},
    description={
    	acronym for \emph{Spectral Energy Distribution}. In
        astrophysics, the SED is a function that gives energy per
        wavelength or frequency unit. In the \gls{VO}, SED is a data
        model --expressed as an \gls{XML} \gls{Schema}-- for such
        astrophysical spectral energy distributions, that can be also
        used to specify single frequency fluxes (continuum
        measurements).
    }
}

\newglossaryentry{Semantics Working Group}
{
    label={Semantics Working Group},
    name={Semantics Working Group},
    description={
    	This \gls{IVOA} working group (WG) aims to systematise the way
        in which meaning is attributed to particular data or metadata
        fields within the VO. This WG started by specifying \gls{UCD}s,
        which provided broad physical meaning to quantities in generic
        \gls{VOTable}s. Later, it has expanded its scope to keep
        maintaining UCDs, but also specify an IVOA Standard Vocabulary,
        and the exploration of Semantic Web/Ontologies technologies
        within the Virtual Observatory.
    }
}

\newglossaryentry{Serialisation}
{
    label={Serialisation},
    name={Serialisation},
    description={
    	Serialisation is the process of converting a set of different
        objects and entities, belonging to different data models, into
        a single, linearly accessible, document, in such a way that a
        univocal reconstruction of the serialised objects and entities
        is possible.
    }
}

\newglossaryentry{SIAP}
{
    label={SIAP},
    name={SIAP},
    description={
    	acronym standing for \emph{Simple Image Access Protocol}. SIAP
        is a protocol specially crafted for the discovery and retrieval
        of astronomical images within the \gls{Virtual Observatory}.
        The SIAP is an \gls{IVOA} \gls{Working Draft} [TODO check], and
        is one of the IVOA's \gls{Data Access Layer} \gls{Working
        Group} protocols.
    }
}

\newglossaryentry{SOAP}
{
    label={SOAP},
    name={SOAP},
    description={
    	acronym originally standing for \emph{Simple Object Access
        Protocol}. SOAP is a web services communication protocol which
        extends \gls{XML-RPC} with additional capabilities, such as
        support for different transports (i.e., SOAP could not only be
        used on \gls{HTTP}/HTTPS, but on \gls{SMTP}, and even instant
        messaging if a suitable transport layer were built). SOAP web
        services are usually described by a \gls{WSDL} document.
    }
}

\newglossaryentry{SQL}
{
    label={SQL},
    name={SQL},
    description={
    	The \emph{Standard Query Language} is the most standardised way
        to query databases. It specifies database operations such as
        \texttt{SELECT} rows and columns, \texttt{INSERT} data into
        existing tables, \texttt{UPDATE} particular rows, or even
        \texttt{CREATE} the tables. When querying archives, the most
        usual SQL construct is the \texttt{SELECT}, which specifies
        which tables we want to obtain data \texttt{FROM}, and what
        conditions do the data conform to (\texttt{WHERE} is valid data
        in parameter space).
    }
}

\newglossaryentry{SSAP}
{
    label={SSAP},
    name={SSAP},
    description={
    	acronym for \emph{Simple Spectrum Access Protocol}, a protocol
        devised after the \gls{SIAP}, and that allows for discovery and
        retrieval of instrumental or synthetic spectra within the
        \gls{Virtual Observatory}. The SSAP is an \gls{IVOA Working
        Draft}, nearing the status to \gls{IVOA Proposed
        Recommendation} [TODO check].
    }
}

\newglossaryentry{Starlink Project}
{
    label={Starlink Project},
    name={Starlink Project},
    description={
    	 The Starlink project was a UK funded computing project which
         supplied general-purpose data reduction software. It was
         funded from 1980 by the Particle Physics and Astronomy
         Research Council, which withdrew funding in 2005. The Joint
         Astronomy Centre took over Starlink's software assets, with
         the latest software release dated February 2008.
    }
}

\newglossaryentry{STIL}
{
    label={STIL},
    name={STIL},
    description={
		Acronym for \emph{Starlink Tables Infrastructure Library}.
		Developed by Mark Taylor for the \gls{Starlink project},
		the STIL is a pure \gls{Java} library for generic input,
		output, and processing of tabular data. It presents to the
		application programmer a view of a table which looks the
		same regardless of whether it came from a \gls{FITS} file,
		a \gls{VOTable}, an \gls{ASCII} text file, a query on a
		relational database, or whatever. It was initially
		developed to allow Java packages to deal with the Starlink
		table format, and has been enhanced to deal with VOTables.
    }
}

\newglossaryentry{SVO}
{
    label={SVO},
    name={SVO},
    description={
    	acronym standing for \emph{Spanish Virtual Observatory}. The
        SVO is the organisation that represents VO efforts within
        \gls{IVOA} and the \gls{Euro-VO} since 2004. SVO is leaded by
        the VO and Archives group at the \gls{LAEFF-INTA}, and its PI
        is Enrique Solano. LAEFF-INTA provides most of the FTEs for the
        SVO, but the \gls{IAA-CSIC} also provides 1 FTE, belonging to
        the \gls{AMIGA} group.
    }
}

\newglossaryentry{Tag}
{
    label={Tag},
    name={Tag},
    description={
    	Piece of metadata used to label a particular piece of data. In
        \gls{XML}, tags are enclosed between angular brackets
        (\texttt{<tag>}), and either surround a piece of data between
        corresponding start and end labels (e.g., in
        \gls{XHTML},\texttt{<p></p>} are used for paragraphs, as in
        \texttt{<p>This is a tagged paragraph</p>}), or are
        self-contained (e.g., in a \gls{VOTable}, common table
        parameters are defined as \texttt{<PARAM name=``Right
        Ascension'' ucd=``pos.eq.ra'' unit=``degrees'' value=``180.0''
        />}). Notice that tag semantics are provided (preferably) by
        the XML Schema, or the XML DTD if the XML document does not use
        an Schema.
    }
}

\newglossaryentry{Text-based file format}
{
    label={Text-based file format},
    name={Text-based file format},
    description={
    	a text based file format is one where data is stored in a
        human-readable form, so that text manipulation tools (shell
        scripts, UNIX tools such as \texttt{sed} and \texttt{awk}, for
        instance) can deal with such files, at the expense of data
        retrieval speed. Programs which use text-based files need to
        parse the input in order to convert the text into an internal
        binary format suitable for manipulation. Please compare with
        \gls{binary-based file formats}.
    }
}

\newglossaryentry{TOPCAT}
{
    label={TOPCAT},
    name={TOPCAT},
    description={
    	acronym standing for \emph{Tool for OPerations on CAtalogues
        and Tables}, and name of an application written in \gls{Java}
        developed by Mark Taylor for the \gls{Starlink project}. It
        provides an interactive graphical viewer and editor for tabular
        data, and uses the \gls{STIL} library as its internal table
        model. In the latest versions it has incorporated \gls{VO}
        capabilities, both by enhancing the STIL to deal with
        \gls{VOTables}, and by incorporating \gls{PLASTIC} (through the
        \gls{PlasKit}).
    }
}

\newglossaryentry{TST}
{
    label={TST},
    name={TST},
    description={
    	Tab-separated text is a \gls{text-based file format} where
        records are separated by the text line separator, and fields
        are separated by tabs (\gls{ASCII} character \texttt{0x08}).
        Optionally, the last row of comments prior to the first row of
        data can contain table headers. Compare with \gls{CSV}.
    }
}

\newglossaryentry{UCD}
{
    label={UCD},
    name={UCD},
    description={
    	acronym of \emph{Unified Content Descriptor}. UCDs are one of
        the attributes (meta-data) that table columns in the
        \gls{Virtual Observatory} can have. In particular, is one of
        the attributes for the \texttt{FIELD} and \texttt{PARAM} tags
        in a \gls{VOTable}. The UCD is built from a controlled
        vocabulary, managed by the \gls{IVOA Semantics Working Group},
        that provides physical semantics to the \texttt{FIELD} or
        \texttt{PARAM}eter using the UCD. There are two versions of the
        UCD standard: UCD1, which uses long strings of underscore (_)
        separated tags (e.g. \texttt{POS_EQ_RA}), and the newer,
        recommended, UCD1+, which consists of a series of atoms that
        can change or refine their meaning by juxtaposition. E.g.,
        \texttt{pos.eq.ra} indicates that some field provides the right
        ascension for a position in equatorial coordinates, while
        \texttt{pos.eq.ra; meta.main} indicates that for all the fields
        that might provide a right ascension, this is the main
        reference.
    }
}

\newglossaryentry{Unicode}
{
    label={Unicode},
    name={Unicode},
    description={
		International standard character encoding capable of
		holding a minimum of 65536 unique characters, with
		different graphic renditions per character. Unicode
		presents a uniform encoding of all present day characters
		and glyphs in any language (including modern and ancient
		Greek, Cyrillic, Arabic, Hebrew, Chinese, Japanese, Korean,
		Devanagari, et cetera), and with room for languages such as
		the Aramaic, and even for embellishment glyphs. Two
		different flavours of Unicode exist: UTF-16, where each
		character always occupies two bytes (16 bits, hence the
		name); and UTF-8, where characters in common with
		\gls{ASCII} occupy just one byte (8 bits) for backwards
		compatibility, and non-ASCII characters use a prefix
		followed by the two-byte Unicode identifier.
    }
}

\newglossaryentry{URI}
{
    label={URI},
    name={URI},
    description={
    	acronym for \emph{Universal Resource Identifier}. Defined by
        \gls{IETF}'s [http://www.ietf.org/rfc/rfc2396.txt RFC 2396],
        the URI is a compact string of characters useful for
        identifying an abstract or physical resource. A particular URI
        specifies a protocol, a location, and a particular resource. If
        the particular resource resides in a hierarchy, the protocol
        ends in \texttt{//}. It can be seen that an \gls{IVORN} is a
        hierarchical \gls{URI} (in particular, a \gls{URN}) where the
        protocol is \texttt{ivo://} (the Virtual Observatory protocol),
        the location is the Authority ID, and the particular resource
        is the Resource Key.
    }
}

\newglossaryentry{URL}
{
    label={URL},
    name={URL},
    description={
    	acronym for \emph{Universal Resource Locator}, a particular
        kind of \gls{URI} that can be directly used to access a
        resource using the protocol encoded in the protocol part of the
        URI, and with the path encoded after the protocol. There is no
        need for URLs to remain persistent. Compare with \gls{URN}.
    }
}

\newglossaryentry{URN}
{
    label={URN},
    name={URN},
    description={
    	acronym for \emph{Universal Resource Name}. URNs are the subset
        of \gls{URI}s that are required to remain \emph{globally
        unique} and \emph{persistent} even when the resource they
        identify ceases to exist or becomes unavailable. A URN differs
        from a \gls{URL} in that it's primary purpose is the
        \emph{persistent labelling of a resource with an identifier}.
        Such identifiers are drawn from one of a set of defined
        namespaces, each of which has its own set, name, structure, and
        assignment procedures. Resolvers for the namespace of a URN
        must exist, and they will be able to deal with all valid URNs
        for that namespace. URIs for URNs start with the \texttt{urn:}
        namespace.
    }
}


\newglossaryentry{View}
{
    label={View},
    name={View},
    description={
    	Particular rendition of data coming from a \gls{Model}. Views
        can be elements of a \gls{GUI}, such as table in a computer
        application; can be particular web pages from a web server; or
        can be data in a particular data format, such as a \gls{FITS}
        file or a VOTable generated by an application. See \gls{MVC}.
    }
}

\newglossaryentry{Virtual Observatory}
{
    label={Virtual Observatory},
    name={Virtual Observatory},
    description={
    	[TODO provide suitable Virtual Observatory definition(s)]
        Collection of interoperating data archives and software tools
        which utilise the internet to form a scientific research
        environment in which astronomical research programs can be
        conducted. VO-compliant tools access such data archives in a
        way that is transparent to the user.
    }
}

\newglossaryentry{Virtual Organisation}
{
    label={Virtual Organisation},
    name={Virtual Organisation},
    description={
    	Group of users of the same \gls{grid} infrastructure which
        similar interests, in order to share computing resources.
    }
}

\newglossaryentry{VO}
{
    label={VO},
    name={VO},
    description={
    	acronym standing for \gls{Virtual Observatory}. In \gls{grid}
        parlance, VO stands for \gls{Virtual Organisation}, but in this
        glossary (and throughout the thesis) Virtual Organisations are
        shortened into \gls{VOrg}. When referring the Virtual
        Observatory and Virtual Organisations at the same time,
        \gls{VObs} and \gls{VOrg} are used.
    }
}

\newglossaryentry{VO Query Language Working Group}
{
    label={VO Query Language Working Group},
    name={VO Query Language Working Group},
    description={
    	This \gls{IVOA} working group is in charge of defining a
        universal query language to be used by applications accessing
        distributed data within the \gls{Virtual Observatory}
        framework. Such query language is called VOQL, and will be an
        evolution of the prototype ADQL (Astronomical Data Query
        Language). Future VO standards, such as the Table Access
        Protocol, will make use of VOQL as its query language.
    }
}


\newglossaryentry{VObs}
{
    label={VObs},
    name={VObs},
    description={
    	\gls{Virtual Observatory}.
    }
}

\newglossaryentry{VOEvent Working Group}
{
    label={VOEvent Working Group},
    name={VOEvent Working Group},
    description={
    	This \gls{IVOA} working group's objective is to define the
        content and meaning of a standard information packet for
        representing, transmitting, archiving, and publishing the
        discovery of an immediate event in the sky. This packet is
        called VOEvent. The objective is to drive robotic telescopes
        and archive searches, alert the community, and to build
        interoperable archives for such events. The scope for these
        events includes not just ``photon'' events, but also
        gravitational waves, neutrinos, air showers, et cetera.
    }
}

\newglossaryentry{VOrg}
{
    label={VOrg},
    name={VOrg},
    description={
    	\gls{Virtual Organisation}.
    }
}

\newglossaryentry{VOTable}
{
    label={VOTable},
    name={VOTable},
    description={
    	An \gls{XML} document that provides \gls{metadata} for one or
        more resources, each of them possibly containing one or more
        tables, which in turn may contain actual data, or links to the
        data itself. The VOTable allows for serialisation of different
        data models by means of an special attribute field, the
        \gls{\texttt{utype}}. An additional attribute, the
        \gls{\texttt{ucd}}, is used in order to provide physical
        meaning to the data model attribute. The VOTable data format
        was the first \gls{IVOA} standard to reach the status of
        \gls{IVOA Recommendation}, and the Commission 5 of the
        \gls{IAU} (dealing with data formats, archives, and the VO) is
        studying its promotion to IAU Standard.
    }
}

\newglossaryentry{VOTable Working Group}
{
    label={VOTable Working Group},
    name={VOTable Working Group},
    description={
    	Working group of \gls{IVOA} devoted to maintain and extend the
        \gls{VOTable} definition and schema.
    }
}

\newglossaryentry{W3C}
{
    label={W3C},
    name={W3C},
    description={
    	Short version of \emph{World Wide Web Consortium}. The W3C
        develops standards applicable to the \gls{World Wide Web}, such
        as the \gls{HTML} and \gls{XHTML} specifications for document
        mark-up languages, or the \gls{CSS} specification for
        conditional X/HTML formatting. W3C's members come from the
        companies and other organisations developing web browsers and
        other web-related technologies. \gls{IVOA}'s document promotion
        system loosely resembles W3C's, with Working Drafts, Proposed
        Recommendations, and IVOA Recommendations.
    }
}

\newglossaryentry{Web}
{
    label={Web},
    name={Web},
    description={
    	usually, short form of the \gls{World Wide Web}.
    }
}

\newglossaryentry{Web services}
{
    label={Web services},
    name={Web services},
    description={
    	web services are defined by the \gls{W3C} as \emph{software
        systems designed to support interoperable machine-to-machine
        interaction over a network}, but many systems exist which
        fulfil this definition. In a more specific definition, web
        services are computer programs which are exposed and invoked
        via traditional web-based protocols (\gls{HTTP}, \gls{HTTPS}).
        Two flavours of web services exist: \gls{REST}, and \gls{SOAP}.
        REST web services can be consumed by any tool that can generate
        an HTTP GET, POST, or even DEL communication (any web browser,
        or the \texttt{curl} command line utility,
		for example), and their output
        format is defined by the service provider in human readable
        documentation. SOAP web services describe themselves by a
        \gls{WSDL} document, and all the communication is enclosed in
        an \gls{XML} \emph{envelope}, following the SOAP specification.
    }
}

\newglossaryentry{Workflow}
{
    label={Workflow},
    name={Workflow},
    description={
    	set of steps performed on different observational data sets in
        order to produce one or more scientific products. Similar to
        pipeline in meaning, but used more frequently when processing
        is not just linear, but needs different inputs and intermediate
        decisions.
    }
}

\newglossaryentry{Working Draft}
{
    label={Working Draft},
    name={Working Draft},
    description={
    	Within an \gls{IVOA} \gls{Working Group} (WG), a new technical
        specification starts as a \gls{Working Draft} (WD). Only when
        the WD is mature enough, an \gls{RFC} is started, and if there
        is agreement within the WG, the document will be promoted to a
        \gls{Proposed Recommendation}.
    }
}

\newglossaryentry{Working Group}
{
    label={Working Group},
    name={Working Group},
    description={
    	Within \gls{IVOA}, standardisation work occurs in the scope of
        several Working Groups (WGs), each one devoted to a part of the
        Virtual Observatory. Presently, the following WGs exist:
        \gls{Applications}, \gls{Data Access Layer}, \gls{Data
        Modeling}, \gls{Grid & Web Services}, \gls{Resource Registry}',
        \gls{Semantics}, \gls{VO Event}, and \gls{VO Query Language}.
    }
}

\newglossaryentry{World Wide Web}
{
    label={World Wide Web},
    name={World Wide Web},
    description={
    	system of interlinked \gls{hypertext} documents accessable
        through the internet. More technically, the system comprises
        servers which can answer \gls{HTTP} requests, serving
        \gls{HTML} documents which contain the actual content, together
        with images, sound, and other media. The resulting mesh of
        servers and links between them, together with the world-wide
        nature of the system is what gave it its name. A user needs a
        \gls{browser} in order to access the web.
    }
}

\newglossaryentry{WSDL}
{
    label={WSDL},
    name={WSDL},
    description={
		acronym for \emph{Web Services Description Language}. WSDL
		is an \gls{XML}-based language and model for describing
		\gls{web services}. WSDL documents define the messages
		understood by the web service, the parameters, and the data
		types to be used with those messages and the information
		being returned, and the different end-points and transport
		protocols which provide a given service. WSDL documents use
		\gls{XML Schema} to provide data types for the different
		queries and responses.
    }
}

\newglossaryentry{XHTML}
{
    label={XHTML},
    name={XHTML},
    description={
    	\gls{XML}-based variant of \gls{HTML}, allowing XHTML content
        to be embedded in other XML documents, perhaps in a different
        namespace, which allows for data aggregation. As an aside of
        being a form of XML, XHTML documents are well-formed (i.e.,
        \gls{tags} do not intermingle, but are completely contained
        within other tags; e.g., \texttt{<b><i>I'm not
        well-formed</b></i>} is not well-formed, as the \texttt{<b>}
        tag closes before the \texttt{<i>} tag, which opened last,
        while \texttt{<b><i>well-formed!</i></b>} is well-formed),
        which allows easier parsing. If integration with other XML
        documents is not a concern, the latest HTML specification
        mandates compliant documents to be well-formed, gaining most of
        XHTML benefits, except for being valid XML documents.
    }
}

\newglossaryentry{XML}
{
    label={XML},
    name={XML},
    description={
    	acronym standing for \emph{eXtensible Markup Language}, a
        language whose purpose is to \gls{markup} data elements, i.e.
        label them, to express data meaning and purpose next to said
        data elements. Both data and markup are text-based, with
        support for different character encoding specifications,
        including Unicode, which makes it readable for humans and
        unambiguous for computers.Besides, XML markup is always
        well-formed, in the sense that all elements are completely
        included in other elements, and there is no element overlap,
        what makes XML parsers easier to write than for other markup
        languages (i.e., for old-style \gls{HTML}). The particular
        \gls{tags} used for each particular XML document are expressed
        by an \gls{XML schema}, which allows automated validation of
        XML documents: a valid document must conform to the particular
        XML schema specified in its header, and has to be well-formed.
        \gls{Virtual Observatory} \gls{Data Access Layer} protocols use
        different XML-based representations for data transfer. The most
        common XML document within the VO is the \gls{VOTable}.
    }
}

\newglossaryentry{XML schema}
{
    label={XML schema},
    name={XML schema},
    description={
    	the schema of an \gls{XML} document is the expression of the
        set of rules that allow creating valid XML documents. These
        rules include the data type for values of \gls{tags}, tag
        \gls{attributes}, inclusion, and order relationships, i.e., how
        many times and in what order a particular tag can appear, what
        kind of tags can be included inside. Different XML schema
        specifications exist, being the first the \gls{DTD} (Document
        Type Definition), the \gls{W3C} \gls{XML Schema} (observe the
        difference in capitalisation), and others.
    }
}

\newglossaryentry{XML Schema}
{
    label={XML Schema},
    name={XML Schema},
    description={
    	standard of the \gls{W3C} for the specification of \gls{XML
        schema}s. It uses a purpose specific \gls{XML}-based language
        for the specification of the restrictions of XML documents. XML
        Schema restrictions on data types can be very sophisticated,
        using \gls{regular expressions} to validate values and
        attributes. Compare with \gls{DTD}.
    }
}

\newglossaryentry{XML-RPC}
{
    label={XML-RPC},
    name={XML-RPC},
    description={
    	\gls{XML}-based Remote Procedure Call, a way of invoking
        software functions from running applications within the same
        machine, that uses XML for parameter encoding, method
        invocation, and response encoding and parsing. The \gls{IVOA}
        messaging protocols, \gls{PLASTIC} and \gls{SAMP}, use XML-RPC
        (in particular, PLASTIC uses \gls{Java-RMI}).
    }
}
