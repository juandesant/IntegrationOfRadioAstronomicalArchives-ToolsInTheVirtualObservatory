\chapter*{Dedicatoria} % for my granny (fold)
\begin{adjustwidth}{0.6\columnwidth}{0cm}
	\begin{flushright}
		\emph{A mi abuela, que nunca creyó vivir para ver este momento. 
		Y a quienes siempre han estado a mi lado, incluso cuando menos 
		me lo merecía: sé que siempre, de una forma u otra, podré 
		contar con vosotros pase lo que pase.}
	\end{flushright}
\end{adjustwidth}
% for my granny (end)

\cleardoublepage

\chapter*{ } % (fold)
\begin{adjustwidth}{0.15\columnwidth}{0.15\columnwidth}
	
	\emph{Desde que orbitaron los primeros satélites, hacía unos 
	cincuenta años, billones y cuatrillones de impulsos de información 
	habían estado llegando del espacio, para ser almacenados para el 
	día en que pudieran contribuir al avance del conocimiento. Sólo 
	una minúscula fracción de esa materia prima sería tratada; pero no 
	había manera de decir qué observación podría desear consultar algún 
	científico, dentro de diez, o de cincuenta, o de cien años. […] 
	Formaban parte del auténtico tesoro de la Humanidad, más valioso 
	que todo el oro encerrado inútilmente en los sótanos de los bancos.
	\\ \\}
	
	\begin{flushright}
		Arthur C. Clarke (1917-2008),\\ 
		\emph{2001: Una Odisea Espacial} (1968).
	\end{flushright}
\end{adjustwidth}
% quote 2001 (end)

When a model makes a model that models a model, what model
models the model that modelled the model's model? (Classic
English tongue-twister)

 Essentially, all models are wrong, but some are useful. (George
E. P. Box, Professor Emeritus of Statistics at the University of
Wisconsin, and pioneer in the areas of quality control, time
series analysis, design of experiments, and Bayesian inference.)

 We should demand that when models are used, the assumptions and
model simplifications are clearly stated. (Orrin H. Pilkey,
Emeritus Professor of Geology at Duke University.)

 Successful model building is generally non-theory driven.
(Mauricio Suárez, UCM, and Nancy Cartwright, London School of
Economics and UCSD, in \emph{Theories: Tools versus Models})

 [Physically] representative models are supposed to be models
[...] such that all the claims of [theoretical knowledge] are
satisfied and no additional assumptions are imported. (Mauricio
Suárez, UCM, and Nancy Cartwright, London School of Economics
and UCSD, in \emph{Theories: Tools versus Models})

 The most important motivation for the research work that
resulted in the relational model was the objective of providing
a sharp and clear boundary between the logical and physical
aspects of database management. (E. F. Codd, British computer
scientist, inventor of the relational approach to database
management, and winner of the 1981 Turing Award.)

 We used to think that if we knew one, we knew two, because one
and one are two. We are finding that we must learn a great deal
more about \emph{and}. (Sir Arthur Stanley Eddington
(1882-1944), Plumian Professor of Astronomy at the University of
Cambridge, most famous for his observations of the bending of
starlight near the eclipsed sun as predicted by Albert
Einstein's General Theory of Relativity.)

 For the truth of the conclusions of physical science,
observation is the supreme Court of Appeal. (Sir Arthur Stanley
Eddington (1882-1944), in \emph{The Philosophy of Physical
Science}.)

 The progress of science requires more than new data; it needs
novel frameworks and contexts, [which] are not simply discovered
by pure observation; they require new modes of thought. (Stephen
Jay Gould (1941-2002), geologist, palaeontologist, evolutionary
biologist, and popular-science author, in \emph{The Flamingo's
Smile}.)

 Computer Science is no more about computers than astronomy is
about telescopes. (Edsger Wybe Dijkstra (1930-2002), computer
scientist, and winner of the 1972 Turing Award, considered one
of the fathers of \emph{structured programming}.)

 Simplicity is prerequisite for reliability. (Edsger Wybe
Dijkstra (1930-2002), computer scientist, in \emph{How do we
tell truths that might hurt?})

 Program testing can be used to show the presence of bugs, but
never to show their absence. (Edsger Wybe Dijkstra (1930-2002),
computer scientist, and winner of the 1972 Turing Award, in
\emph{Notes On Structured Programming}.)

 If you're doing an experiment, you should report everything
that you think might make it invalid — not only what you think
is right about it... Details that could throw doubt on your
interpretation must be given, if you know them. (Richard Feynman
(1918-1988), physicist, Nobel prize winner for his contributions
to quantum electrodinamics.)

 A complex system that works is invariably found to have evolved
from a simple system that worked. The inverse proposition also
appears to be true: A complex system designed from scratch never
works and cannot be made to work. You have to start over,
beginning with a working simple system. (John Gall's Law, in
\emph{Systemantics: How Systems Really Work and How They Fail}.)

 A record, if it is to be useful to science, must be
continuously extended, it must be stored, and above all it must
be consulted. (Vannevar Bush (1890-1974), engineer, and inventor,
and father of the hyperlink concept, in \emph{As we may think})

 Every time one combines and records facts in accordance with
established logical processes, the creative aspect of thinking
is concerned only with the selection of the data and the process
to be employed, and the manipulation thereafter is repetitive in
nature and hence a fit matter to be relegated to the machines.
(Vannevar Bush (1890-1974), engineer and inventor, and father of
the hyperlink concept, in \emph{As we may think})
