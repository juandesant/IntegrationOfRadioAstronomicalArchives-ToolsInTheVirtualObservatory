% chapter conclusiones (fold)
\chapter*[Conclusiones y trabajo futuro]{Conclusiones y trabajo futuro}
\addcontentsline{toc}{chapter}{Conclusiones y trabajo futuro}
\label{conclusiones}
	
	En esta tesis hemos realizado un repaso extenso e inclusivo
	de qué significa el Observatorio Virtual (VO) desde el punto de
	vista de los radioastrónomos de hoy en día: por un lado,
	hemos visto cómo el VO mejora las capacidades de los archivos
	que utilizan, y cómo la descripción unificada de dichos datos
	permite su interoperabilidad; por otro, hemos visto la
	necesidad de que las herramientas software a las que están
	acostumbrados los radioastrónomos se adapten al VO, cómo
	se puede lograr esa integracion, y cuál es la forma de menor
	impacto en dichas aplicaciones de lograr esa integración.
	
	Durante dicho repaso a la literatura disponible, y a los
	estándares desarrollados por IVOA, hemos visto que no
	existía ningún modelo de datos sancionado por IVOA
	apropiado para todas las observaciones astronómicas.
	Además, los modelos existentes no soportaban las
	especificidades de las observaciones radioastronómicas.
	
	Por ello, se ha desarrollado un modelo de datos, pensado
	inicialmente para observaciones con radiotelescopios de
	antena única (RADAMS), aprovechando de un lado el boceto de
	modelo de datos de Observaciones~\cite{2005dmo..rept.....M}, y
	el modelo de datos de Caracterización~\cite{2008dmadcrept.....L}.
	Sobre ese modelo se han desarrollado las clases Provenance,
	Policy, Curation y Packaging, para las que no existían
	propuestas concretas.
	
	Esos añadidos se realizan de forma modular tanto en su 
	expresión en base de datos como en su serialización XML, de
	modo que pueden ser ignorados por aplicaciones que no los
	entiendan.
	
	Puesto que la parte más específicamente radioastronómica del
	RADAMS se halla en el modelo de datos Provenance (Procedencia,
	o Linaje), y dado el ya mencionado caracter modular del RADAMS,
	el resultado es la creación de una propuesta completa de modelo
	de datos de Observación para IVOA. En el caso del sub-modelo
	Provenance.Instrument, se ha tenido en cuenta su adaptabilidad
	para la descripción de interferómetros, y no sólo de
	radiotelescopios de antena única.
	
	Hemos validado este modelo al implementar dos archivos
	radioastronómicos basados en él, como son los archivos para las
	antenas DSS-63 e IRAM~30m. Esa validación es, a su vez,
	un respaldo a la viabilidad de los modelos de datos de
	IVOA como herramientas de diseño de archivos astronómicos
	en general.
	
	Para esta última tarea hemos proporcionado metadatos completos
	para la creación de las serializaciones XML de los servicios
	del archivo, consistentes en palabras clave FITS y Unified
	Content Descriptors (UCDs) para todos los atributos. Se ha
	usado el mecanismo de yuxtaposición de UCDs para proporcionar
	UCDs más específicos de los ya existentes. Se ha propuesto
	también la adición de algunos UCDs al vocabulario UCD1+.
	
	De otra parte, hemos analizado las opciones disponibles para
	la incorporación de herramientas astronómicas (no
	específicamente radioastronómicas) en el marco del VO, y
	hemos encontrado que utilizar un protocolo de mensajería
	basado en tecnologías multiplataforma y multilenguaje, como
	XML-RPC, nos permite hacer compatible e interoperable con el
	VO esas herramientas antiguas con un mínimo de esfuerzo, y 
	de cambios en el código, siempre que esas herramientas sean
	conformes a un esquema Modelo-Vista-Controlador (MVC).
	
	Hemos validado dicho análisis al implementar SAMP en varias
	aplicaciones, y haber demostrado su interoperabilidad. En
	concreto, hemos implementado SAMP, y los mensajes propios
	de MOVOIR, en las aplicaciones VODownloader, \massa{}, y otros
	módulos de MOVOIR.
		
	Por último, el MOVOIR es la primera muestra de una posible
	arquitectura para la creación de módulos de ampliación
	---\emph{plug-ins}--- que permitirían la ampliación de las
	aplicaciones existentes y que utilicen el protocolo SAMP,
	sin necesidad ninguna de tener acceso a su código fuente.
	
	\section*{Trabajo futuro} % (fold)
	\addcontentsline{toc}{section}{Trabajo futuro}
	\label{sec:trabajo_futuro}
		
		En el futuro, voy a trabajar con el Grupo de Gestión de
		Archivos (Archive Management Group) del Departamento de 
		Archivos (Archive Department) de la División de Gestión de
		Datos y Operaciones de la ESO (European Southern
		Observatory, Observatorio Europeo Austral) en la
		organización de sus archivos, y más concretamente de sus
		metadatos correspondientes en el marco del IVOA, lo que va
		a permitir la continuidad del presente trabajo. Además,
		seguiré siendo miembro del grupo de trabajo de IVOA
		dedicado a Modelado de Datos (IVOA Data Modelling Working
		Group), y uno de los técnicos a cargo del desarrollo de un
		modelo de datos para cubos de datos radio en el marco
		del proyecto europeo EuroVO-AIDA Technology Forum.
		
		En concreto, espero realizar las siguientes tareas en ese
		marco, y en colaboración con el grupo AMIGA:
		
		\begin{itemize}
			\item Someter las diferentes partes del RADAMS para su
			aprobación dentro del IVOA Data Modelling Working Group,
			para la creación final de una IVOA Recommendation
			correspondiente al Modelo de Datos de Observación.
			Existirá así al menos una implementación de referencia
			del modelo, un requisito del IVOA para sus estándares.
			
			\item Extender el uso de tecnologías de web semántica,
			propuestas en la forma de extensión de vocabularios,
			en substitución de mecanismos alternativos, no estándar,
			como las ESO Hierarchic Keywords.
			
			\item Presentar el MOVOIR en los IVOA InterOp, y
			el ADASS (Astronomical Data Analysis Software and
			Systems), y obtener sugerencias de los usuarios
			para crear más módulos para el MOVOIR.
			
			\item Extender y dejar listos para distribución
			los módulos de MOVOIR.
			
			\item Proponer MOVOIR como implementación de referencia
			para una arquitectura de \emph{plug-in} sancionada por
			IVOA. Si se sanciona una tal arquitectura, se permitiría
			una mejor integración de los módulos de MOVOIR, mediante
			el soporte de mensajes adicionales, con aplicaciones que
			implementen dichos mensajes.
			
			\item Desarrollar una métrica objetiva de calidad de
			datos basados en RADAMS, e implementar un módulo MOVOIR
			que sea capaz de proporcionarla.
			
			\item Usar MOVOIR para la adaptación del paquete de
			análisis de cubos de datos y modelado cinemático GIPSY
			al VO, en el marco del proyecto
			AMIGA\textsuperscript{3}.
			
			\item Generalizar el RADAMS para que soporte datos
			interferométricos, en colaboración con el ALMA Archive
			Team de la ESO, dentro del mismo marco.
		\end{itemize}
		
	% section trabajo_futuro (end)
	
% chapter conclusiones (end)