% chapter conclusions (fold)
\chapter*[Conclusions and future work]{Conclusions and future work}
\addcontentsline{toc}{chapter}{Conclusions and future work}
\label{conclusions}
	
	In this thesis we have performed an extensive and inclusive
	review of what the Virtual Observatory means from the eyes of
	current radio astronomers: how does the Virtual Observatory
	enhance their interaction with astronomical archives, and how data
	in those archives become interoperable thanks to a unified
	description; on the other hand, we have seen how the tools
	they are accustomed to use need to be incorporated into the
	Virtual Observatory, what different means exist for that,
	and which one has the less impact on said applications
	in order to provide VO compatibility.
	
	During that review of available literature, and of IVOA
	proposed and drafted standards, we have seen that the IVOA
	had not proposed any complete data model, and many radio
	astronomical specifics where not taken into account. As a
	result, no appropriate data model for radio astronomical
	observations existed.
	
	Thus, a data model has been developed (RADAMS) to support,
	initially, observations made with single-dish radio telescopes.
	The RADAMS uses as a basis the IVOA draft proposal for an
	Observation data model~\cite{2005dmo..rept.....M}, combined
	with the Characterisation data model~\cite{2008dmadcrept.....L}.
	On top of that model the Provenance, Policy, Curation, and
	Packaging classes have been created and specified, as no actual
	proposals existed.
	
	Those added classes are completely modular, both in their
	expression in database form as in their XML serialisation,
	as they use their own UTypes, and as such they can be safely
	ignored by applications not able to understand such metadata.
	
	As the most radio astronomy specific part of the RADAMS is
	found in the Provenance data model, and due to the already
	mentioned modularity, the RADAMS conforms a complete Observation
	data model proposal. In the case of the Provenance.Instrument
	sub-model, we have not only covered single-dish instruments,
	but also interferometers.
	
	The RADAMS has been validated by being used as the basis for
	the development of two radio astronomical archives, those for
	the DSS-63 and IRAM~30m antennas. That development, in turn,
	serves to back-up the feasibility of using IVOA data models
	as the basis for the development of new astronomical archives,
	in general.
	
	These archives have needed a complete proposal of metadata
	for their XML serialisations, consisting of FITS keywords
	for the Data-filler, and Unified Content Descriptors (UCDs)
	for all attributes. Many UCDs have been built by means of
	juxtaposition of existing ones, so that the resulting
	UCDs are more specific. A few UCDs have been proposed for
	addition to the UCD1+ vocabulary.
	
	Complementarily, we have analysed the different options
	available for bringing legacy astronomical (not necessarily
	radio specific) tools into the VO, and we have found that
	by using a messaging protocol based on multi-platform and
	multi-language intercommunication technologies such as 
	XML-RPC, we can provide VO compatibility and interoperability
	with minimal effort and code changes, as long as the tools
	to be updated conform to the Model-View-Controller (MVC)
	paradigm. A recently developed IVOA protocol, the
	Simple Application Messaging Protocol (SAMP), conforms to
	that description, and incorporates VO semantics in the way
	messages are created, providing a framework for the
	implementation of the proposed mechanism.
	
	We have validated such analysis, by implementing SAMP, and
	VO-specific messages, in several applications, and showing
	their interoperability. In particular, we have implemented
	SAMP, and MOVOIR messages, into \massa{}, the VODownloader, and
	several other MOVOIR modules.
	
	Finally, the MOVOIR is the first proposal for a plug-in
	development architecture in order to allow the enhancement of
	the operation of existing, SAMP compatible, VO applications,
	without needing access to application source code, and with
	additional discovery capabilities for MOVOIR-aware applications.
	
	\section*{Future work} % (fold)
	\addcontentsline{toc}{section}{Future work}
	\label{sec:future_work}
		
		In the future, I will work with the Archive Management
		Group of the Archive Department of the Data Management and
		Operations Division of the European Southern
		Observatory organisation. The scope of my work will be
		VO-related archive metadata
		management. In addition, I will remain a member of
		the IVOA Data Modelling Working Group, and will also be
		a member in charge of the development of a radio data
		cube data model in the framework of the EuroVO-AIDA
		Technology Forum.
		
		The organisations above will allow me to continue my
		research on astronomical archives and VO technologies,
		and I plan to perform the following tasks, also in
		collaboration with the team at the IAA:
		
		\begin{itemize}
			\item Submit the different RADAMS modules for approval
			to the IVOA Data Modelling Working Group, in order
			to finally issue an Observation Data Model IVOA
			Recommendation. The RADAMS would serve as a reference
			implementation of that Observation Data Model, a
			requisite of the IVOA for their Recommendations.
			
			\item Extend the use of semantic web technologies in
			the ESO as a substitution mechanism for the ESO
			Hierarchic Keywords.
			
			\item Introduce the MOVOIR at the IVOA InterOp meetings,
			within the Applications Working Group, and present it
			at the ADASS (Astronomical Data Analysis Software and
			Systems) conference, compiling user suggestions for
			new MOVOIR modules.
			
			\item Extend and prepare for distribution existing
			MOVOIR modules.
			
			\item Propose the MOVOIR as a reference implementation
			for an IVOA sanctioned plug-in architecture, in order
			to allow better integration of MOVOIR modules with
			plug-in aware applications, and create RADAMS-enabled
			MOVOIR modules.
			
			\item Develop a data quality assessment metric for
			RADAMS compatible archives, and implement it as a
			MOVOIR tool.
			
			\item Use MOVOIR in order to bring VO capabilities
			to the GIPSY high-level radio data cube analysis
			and kinematic modelling package GIPSY al VO, within
			the AMIGA\textsuperscript{3} project framework.
			
			\item Generalise the RADAMS with support for
			interferometry, in collaboration with the ALMA Archive
			Team at ESO, within that same framework.
		\end{itemize}
		
	% section future_work (end)
	
% chapter conclusions (end)