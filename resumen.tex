% chapter resumen (fold)
\chapter*[Resumen]{Resumen}
\addcontentsline{toc}{chapter}{Resumen}
\label{resumen}

% Mensaje a transmitir por el resumen
	
	El nacimiento de la astrofísica se produce cuando se pasa de
	la medición de los movimientos periódicos de los cuerpos
	celestes a la interrogación de luz por medio de la
	espectroscopía. Una forma más poética de decirlo sería
	afirmar que la astrofísica es la ciencia del análisis
	extremadamente cuidadoso de la luz de los cuerpos celestes.
	
	Desde hace tiempo, ese análisis se realiza de forma digital,
	pero sin que exista una uniformidad entre los datos
	proporcionados por cada tipo de observatorio, y ni siquiera
	entre observatorios del mismo tipo.
	
	Dado que la tendencia actual en la astrofísica nos dirige 
	hacia el análisis multifrecuencia de los objetos celestes
	(utilizando observatorios que barren el espectro
	electromagnético desde las ondas de radio hasta los rayos
	gamma, pasando por el infrarrojo, la luz visible, los rayos
	ultravioleta y los rayos-X), pero cada una de esas bandas de
	información se obtiene con instrumentos y observatorios
	diferentes (por ejemplo, es imposible observar rayos-X o 
	rayos gamma si no es desde un telescopio espacial), se
	hace necesario uniformar la forma de expresar información
	científica dentro del mundo de la astrofísica.
	
	Además, y tal y como expresa la cita de Arthur C. Clarke
	que da entrada a esta tesis, es posible encontrar información
	que no se esperaba en los datos guardados. Sin embargo, dado 
	que la capacidad de generación de información de los detectores
	astronómicos viene también dominada por la Ley de Moore, el
	incremento de la cantidad de información guardada es
	exponencial, por lo que de nuevo se hace necesario establecer
	un cambio en la forma de tratar los datos astrofísicos.
	
	Necesitamos, pues, una infraestructura que permita el acceso
	distribuido y uniforme (tanto en protocolos de acceso, como en
	la descripción de la información) a los datos que pueda
	necesitar el astrónomo, pero también que proporcione servicios
	de análisis remoto que minimicen al máximo la necesidad de
	transferir cantidades ingentes de información entre el archivo
	y la estación de trabajo.
	
	Esa infraestructura, basada en tecnología de servicios web,
	tecnología grid, y en la descripción de datos mediante modelos
	de datos basados en XML, se conoce como Observatorio Virtual, y
	viene desarrollándose desde 2001, y fue validada
	en 2002 con el desarrollo del Astrophysical Virtual Observatory
	(AVO), una aplicación que era capaz de mostrar imágenes que
	se obtenían a partir de diferentes archivos, y de dibujar sobre
	esas imágenes las localizaciones de medidas y observaciones
	tomadas por otros observatorios.
	
	Desde nuestro grupo de investigación se lidera el proyecto
	AMIGA (Análisis del Medio Interestelar de Galaxias Aisladas),
	que pretende realizar una caracterización estadística
	multifrecuencia de un conjunto de más de mil galaxias
	seleccionadas por un estricto criterio de aislamiento, lo que
	garantiza que se han visto libres de interacciones con galaxias
	de tamaño comparable durante el último millar de millones de
	años. Debido a que las propiedades que nos interesan son las 
	del hidrógeno neutro y gases en estado molecular como
	H${}_2$ o CO, las longitudes de onda de radio son de particular
	interés para nosotros.
	
	El desarrollo del Observatorio Virtual, sin embargo, ha venido
	dominado por la zona de luz visible, que es en la que contamos
	con mayor experiencia, pero también en la que existía un mayor
	número de archivos ya disponibles.
	
	El propósito de esta tesis, por tanto, es la de proporcionar
	un marco en el cual se puedan crear archivos radioastronómicos,
	y se puedan integrar en el Observatorio Virtual. Veremos que
	es necesario ampliar los modelos actualmente existentes
	dentro del Observatorio Virtual para poder acomodar los datos
	radioastronómicos, y aprovecharemos para proporcionar modelos
	de datos de observaciones más genéricos que los existentes.
	
	Además, es necesario poder integrar las actuales herramientas
	de análisis radioastronómicas con el Observatorio Virtual.
	En esta tesis, desarrollamos un mecanismo para la incorporación
	al Observatorio Virtual tanto de aplicaciones para las que se
	dispone de código fuente como para aquellas que no pueden
	manipularse.
	
	Dicho mecanismo de compatibilidad con el Observatorio Virtual
	vuelve a utilizar técnicas básicas de computación remota como
	XML-RPC para establecer un sistema de mensajería entre
	diferentes módulos basado en un modelo de
	publicación/subscripción, tanto de proceso de datos como de
	acceso al Observatorio Virtual. Se reduce así al mínimo la
	intervención en las aplicaciones, que sólo deben incorporar un
	pequeño módulo de mensajería, dependiendo del resto de módulos
	para el descubrimiento y manipulación de datos del Observatorio
	Virtual.
	
	Como efecto secundario de esta modularidad, y la existencia de
	los mecanismos de publicación/subscripción, se crea un
	mecanismo para la creación de módulos de funcionalidad
	dinámicamente descubribles, y que permite la extensión de
	cualquier aplicación que soporte la suscripción a una serie
	de mensajes ya establecidos.
	
	Por último, procedemos a validar el desarrollo de la tesis
	mediante el desarrollo e integración en el Observatorio Virtual
	de dos archivos radioastronómicos, para los radiotelescopios
	DSS-63 de 70m, ubicado en Robledo de Chavela (Madrid), e
	IRAM~30m de Pico Veleta, en Sierra Nevada (Granada),
	y la integración en el observatorio virtual de una herramienta
	para espectroscopía, \massa{} (MAdrid Simple Spectral
	Analysis).
	
	
\invisiblenote{
- Astronomía, y especialmente astrofísica, ciencia del análisis extremadamente 
cuidadoso de la luz de los cuerpos celestes.

- Ese análisis, desde hace tiempo, se realiza de forma digital, con un 
metaformato común (FITS), pero sin homogeneidad entre observatorios similares,
mucho menos entre observatorios de diferente tipo.

- Hoy en día, la astrofísica se mueve hacia el análisis multifrecuencia 
(desde radio a rayos gamma, pasando por IR, visible, UV, rayos X...)

- Además, el crecimiento exponencial ``Tipo Ley de Moore'' de la resolución 
y sensibilidad de los telescopios implica que \emph{cada año se genera tanta 
información como había disponible el año precedente}.

- NECESITAMOS el tratamiento automático de la información. Para ello se necesita:

   - Archivos: en muchos casos, los datos que se generan durante una observación
     se quedan en manos de quien la hizo; disponer de sistemas (archivos) que 
     proporcionen la información de las observaciones tras un tiempo prudencial 
     es el futuro. El máximo común denominador de los archivos actuales es el
     acceso mediante web.

   - Interoperabilidad de archivos: la existencia de archivos es el primer paso. 
     El paso siguiente debe ser su interoperabilidad, para poder acceder del 
     mismo modo a todos los archivos astronómicos del mundo. Eso implica 
     protocolos basados en el máximo común denominador de esos archivos: la web. 
     Por tanto, es fundamental utilizar protocolos basadas en servicios web,
     y archivos descritos por metadatos basados en el lenguaje de marca XML.

   - Herramientas: los datos científicos contenidos en los archivos 
     interoperables no tienen utilidad si no es para producir resultados 
     científicos. Para ello, es necesario manipularlos con herramientas 
     que permitan realizar tareas científicas complejas. Es necesario que 
     las herramientas puedan adquirir los datos directamente de los citados 
     archivos, y que puedan entender su formato común.

   - Interoperabilidad local: en la filosofía UNIX, cada herramienta se 
     construye para hacer bien una única tarea, permite que otras se agreguen 
     mediante un mecanismo de ``pipes'' (\emph{tuberías}) que permite la 
     interconexión de esas tareas. Si proporcionamos alguna forma sencilla de
     compartir información entre aplicaciones, podemos utilizar en cada paso 
     la aplicación que mejor se adapta a lo que queramos hacer. Parte de esta 
     interoperabilidad local se basa en las mismas herramientas de la 
     interoperabilidad entre archivos, y otra parte de basa en el uso de 
     protocolos de mensajería local basados en XML.

Llamamos Observatorio Virtual al conjunto de archivos, protocolos de 
interoperabilidad, y herramientas, que permiten trabajar a un científico con 
datos de calidad de cualquier instrumento, y utilizando varias herramientas 
que son interoperables entre sí.

Así pues, el Observatorio Virtual es un ejemplo de Arquitectura Orientada a 
Servicios (SOA, Service Oriented Architecture) en sentido amplio, aunque se 
trate de una infraestructura de propósito específico. 
La especificidad se concreta en la definición concreta de sus propios 
protocolos de acceso a datos ---basados en REST---, modelos de datos

En esta tesis:

- Estudiamos la arquitectura, modelos de datos y los protocolos que constituyen
  el observatorio virtual (VO)

- Creamos un modelo de datos radio astronómicos que se pueda utilizar dentro 
  del VO.

- Definimos la arquitectura de un archivo radio astronómico compatible VO,
  que se está implementando en base a dicha arquitectura.

- Transformamos una aplicación existente para que utilice el modelo
  de datos y los protocolos del VO.

- Demostramos la compatibilidad VO de la nueva herramienta y del archivo:

 	- Accedemos a datos VO del archivo desde herramientas VO pre-existentes

    - Accedemos a datos VO de archivos pre-existentes desde nuestra nueva 
      herramienta

    - Accedemos a propiedades específicas de nuestro archivo VO desde la
      nueva herramienta VO.

- Los pasos anteriores los realizamos utilizando tanto los protocolos de
  interoperación entre aplicaciones basados en XML-RPC, como mediante los 
  protocolos estándares de acceso a Espectros y a datos de catálogos.

Durante el trabajo de la tesis fue necesario abordar los siguientes problemas:

- Caracterización de datos radio astronómicos frente a otros datos.

- Determinación de la información de Provenance en radio astronomía en 
  particular, con generalización al resto del Observatorio Virtual.

- Extender la información de Provenance para incluir los pasos de procesamiento,
  obtención de datos y transferencia.

- Caracterizar diferentes servicios de astronomía infraroja, óptica y 
  radio astronomía para darlos de alta en los registros de IVOA.


\section*{Organización de la tesis} % (fold)
\label{sec:thesis_organisation}

The remaining of this thesis is organised as follows:

\todo{Describe thesis organisation first at the part level, and later
at the chapter level for each part.}

% section thesis_organisation (end)


\section*{Main thesis results} % (fold)
\label{introthesisadvancedresults}

This thesis tries to provide an answer to the problem of using radio
astronomical datasets within the VO by addressing the following
problems:

\begin{itemize}

	\item We define the properties of radio datasets in the
       framework of existing IVOA data models, and provide full
       semantics for radio data.

	\item We try to solve the problem of bringing existing legacy
       tools (already in use by the community) in the new framework of
       the Virtual Observatory.

	\item \todoinline{Complete list.}
\end{itemize}

% section introthesisadvancedresults (end)
}

% chapter resumen (end)