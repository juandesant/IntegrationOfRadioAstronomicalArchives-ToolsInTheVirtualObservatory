%2345678901234567890123456789012345678901234567890123456789012345678901

\chapter{Introduction} % (fold)
\label{cha:introduction}
	
	Due to the more abstract nature of radio astronomical
	observations, there are less astronomers and observatories
	providing radio astronomical data when compared to those
	working on the visible part of the spectrum. However, as
	multi-wavelength studies become more and more commonplace in
	astrophysics, and are essential for studies such as the one
	being carried out within the AMIGA group, the capability of
	having access to radio astronomical data becomes of the utmost
	importance.
	
	 There are few radio astronomical observatories with proper
	archives (most of them provide just logbooks, or observation
	registries), so providing VO-compatible radio
	astronomical archives is a valuable contribution to the
	community both because of the availability of the archive
	itself, and
	the bonus possibility of using VO-tools to access it.
	
	 For existing archives, the data model of the archive itself
	does not usually match the data model of the VO, mainly due to
	the difference in scope: archive data models reflect the data
	origin, and are built to answer the queries of engineers,
	commissioning scientists and scientists; VO data models, on the
	other hand, are built to describe observations so that the
	description is enough for astronomers, or computer queries, to
	assess the usefulness of a particular piece of data.
	
	 For a new archive, however, the internal archive data model
	can be built by mirroring VO data models, while adding non-VO
	information so that all needs can be covered.
	

	 In this part, we will describe the RADAMS, a data model for
	single-dish radio astronomical archives reflecting existing
	IVOA data models, but which at the same time provides
	definitions and proposals for modelling additional
	observational aspects. We will use this data model for the
	building of the IRAM 30m and DSS-63 archives, which will be
	shown  in
	Part~\ref{prt:thesis_applications}.
	
	 In addition, some parts
	of the RADAMS will be used in the development of the MOVOIR, 
	a modular VO interface and API to the VO, which will be
	described in Part~\ref{prt:legacyTools}.
	
	%  We will see both approaches when describing the archives for
	% the DSS-63 and IRAM 30m antennas. For the former, no archiving
	% infrastructure existed in the first place, and thus the RADAMS
	% was built with the intent of being used as a blueprint for
	% this archive development. For the latter, earlier archive
	% organisation attempts had been made, specially for pooled
	% observations\footnote{Pooled observations are observations which
	% are scheduled from a pool of available observations projects
	% regarding project priority, weather conditions, availability of 
	% sources, and  availability of allocated time for the project.
	% Pooled observations enhance the project completion rate and
	% scientific return for an observatory, as backup programas
	% exist for non-optimal weather conditions.}. In this case, the
	% RADAMS is built as a layer on top of the existing, instrument
	% specific data model.
	
	We will start by introducing which are the existing IVOA data
	models in chapter~\ref{cha:data_modelling_in_the_vo}, and then
	we will evaluate parts of those models to create the RADAMS
	(Radio Astronomical DAta Model for Single-dish observations) in
	chapter~\ref{cha:radams}. Chapter~\ref{cha:radamscharobs}
	explains how the RADAMS can be used to characterise
	astronomical observations, and the following chapters explore
	in detail which are the missing parts in IVOA data models which
	are needed for the RADAMS: Curation, Packaging and Policy are
	specified in
	chapter~\ref{cha:radams_curation_packaging_and_policy}, while
	data provenance in e-science is reviewed in
	chapter~\ref{cha:data_provenance_in_the_vo}, and the lessons
	learned applied to RADAMS' Provenance in
	chapter~\ref{cha:radams_data_provenance}.
	
% chapter introduction (end)