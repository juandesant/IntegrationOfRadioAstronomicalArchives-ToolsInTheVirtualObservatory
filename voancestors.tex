Due to the purely scientific, non-commercial nature of astrophysical
data, public access to them has always been the norm. With the
connection of most educational and research institutions to the
Internet, the first primitive online archives were born. One of the
first was that of the International Ultraviolet Explorer (IUE), which
was made available to the public in 1985. As the World Wide Web did not
exist yet at that time, astronomers had to login to remote servers from
which the data were downloaded through the File Transfer Protocol
(FTP).

Apart from the data from observations themselves, articles and other
publications on astronomical objects were also of great value, and
during the \emph{Astronomy from large databases}
conference~\cite{1988afld.book.....M}, held in 1987, a proposal was
made for the design and operation of a system to compile all
astronomical publications, the SAO/NASA Astrophysics Data System (ADS).
A prototype was built in 1990, with the system being released to the
public in early 1993~\cite{1993ASPC...52..132K}. The system was made
possible thanks to the collaboration of the publishing journals.

Simultaneously, systems like the Simbad Astronomical
Database\urlnote{http://simbad.u-strasbg.fr/simbad/}~\cite{
2000A&AS..143....9W} (an online database hosted by the Centre de Donées
Astronomiques de Strasbourg, CDS), or the NASA/IPAC Extragalactic
Database\urlnote{http://nedwww.ipac.caltech.edu/}~\cite{
1992ASPC...25...47M} (NED) contained data with diverse information on
observed astrophysical objects, which were compiled, maintained and
published online by the respective large scale astronomical data
centres.

\newcommand{\adsbibcodeurl}[0]
{http://doc.adsabs.harvard.edu/abs_doc/help_pages/bibcodes.html}
As all of these initiative had grown independently, the were developed
with different considerations in mind, and were not coordinated.
However, the possibility of joining publications available at the ADS,
and the Simbad/NED databases, allowed astronomers to have access to
publications on the objects of their interest, or to access more data
in those databases from the corresponding literature. The joint
ADS/NED/Simbad effort was called URANIA~\cite{1996AAS...189.0603B}, and
is one of the first joint initiatives for data service interoperability
in astronomy. The main product of that agreement is the
\emph{bibcode}\urlnote{\adsbibcodeurl}, a unique identifier of
publications which encodes details on the publication date, publication
kind, and/or journal, and author\footnote{When astronomers upload data related to a particular publication to the NED or Simbad services, or that data is gathered from the publication, the bibcode allows the cross-identification of sources and publications.}.

\newcommand{\vizierurl}[0]
{http://webviz.u-strasbg.fr/viz-bin/VizieR}
Outside the URANIA project, the
INES\urlnote{http://sdc.laeff.inta.es/ines/} (IUE Newly Extracted
Spectra)~\cite{1999A&AS..139..183R} archive, holding reprocessed IUE
observations, represents the first instance of interoperability between
two completely different astrophysical services: one holding spectral
information (INES), the other keeping bibliographical references (ADS).
It was possible to access the information available in ADS for a
particular object from the INES archive, and from the ADS the INES
spectra used for a particular article could be accessed. Other service
linked to the ADS through bibcodes is the VizieR\urlnote{\vizierurl}
service for Astronomical Catalogues~\cite{2000A&AS..143...23O}.

Another important milestone in the development of online, interoperable
archives was the NASA multi-mission archive initiative, which tried to
provide a common archival infrastructure across different space-borne
missions. Three portals were finally created:
MAST\urlnote{http://archive.stsci.edu/} (Multi-mission Archive at Space
Telescope)~\cite{1999AAS...194.8302I}, mainly for optical-ultraviolet
observations\footnote{VLA FIRST images are also available through
MAST.}; IRSA\urlnote{http://irsa.ipac.caltech.edu/} (InfraRed Science
Archive)~\cite{2000AAS...19711610B}, for the infrared; and
HEASARC\urlnote{http://heasarc.gsfc.nasa.gov/} (High Energy
Astrophysics Science Archive Research
Center)~\cite{1994BAAS...26..995R} for high-energy (X-rays, gamma rays)
astrophysical observations. These portals allowed for the first time a
unified view of data from widely different provenance under the same
query interface. MAST also provided some applications for data
visualisation across mission, such as the
COPLOT\urlnote{http://archive.stsci.edu/mast_coplot.html} tool.
However, data products from each mission were, still, quite different.

Finally, regarding the use of applications in order to combine and
manipulate data from different services, one of the first examples
prior to the VO development is the Aladin Sky Atlas. Initially, Aladin
was only able to access CDS-based catalogues and images, which were
accessed through custom built, Aladin-specific protocols. However,
Aladin helped in making developers and astronomers realise that any
other system which shared the data access and data description
protocols would allow either to improve Aladin, by allowing it to
access more datasets, or more easily recreate Aladin-like functionality, without the need of being a data and service provider.