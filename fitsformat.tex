\chapter{Description of the FITS file format} % (fold)
\label{cha:description_of_the_fits_file_format}

The FITS file format was created in 1979 to address the need for
astronomical information interchange between radio astronomical sites
(initially, the National Radio Astronomy Observatory for the Very Large
Array, the Netherlands Foundation for Radio Astronomy for the Westerbork
Synthesis Radio Telescope, later joined by the Kitt Peak Observatory).
The format was initially published in 1981~\cite{1981A&AS...44..363W}.

After the first definition, the IAU decided to make the FITS a IAU
standard, and appointed a committee for taking care of the development,
upgrade, and maintenance of the specification (the FITS Working Group of
Commission 5: Documentation and Astronomical Data). Some extensions have
been proposed to the standard regarding the specification of World
Coordinate Systems, Spectral Coordinates, Binary tables, and variable
distortion in coordinates~\cite{2002A&A...395.1061G,
2004ASPC..314..551C, 2006A&A...446..747G}, and the FITS Working Group
has finally delivered the version 3.0 of the FITS standard, containing
all said extensions~\cite{FITS-Working-Group:2008ty}.

So that most FITS concept become clear in order to make a comparison
with the XML VOTable, we provide a brief summary of the FITS file format
in this appendix.

\section{Units and Dimensional Analysis} % (fold)
\label{sec:units}

A quantity without an unit (and possibly a multiplicative scale prefix,
such as milli-, micro-, nano-, kilo-, mega-, et cetera) has no physical
meaning, and cannot be compared with any other similar quantity, making
impossible automatic discovery of comparable datasets. Even adimensional
quantities, such as similar quantity ratios, can only be identified if
units are specified for dimensional quantities.

The FITS standard recommends specifying units for all valued header
keywords, plus all table fields.

There is a basic, IAU-recommended, unit profile, which specifies basic
\emph{Système International} (SI) units for length, mass, time, angle,
plane angle, solid angle, temperature, electric current, luminous
intensity and amount of substance, plus additional astrophysical
units for plane angle, time, energy, mass, luminosity, length, event
counts, flux density, magnetic field, area, and several miscellaneous
units.



% section units (end)

% chapter description_of_the_fits_file_format (end)
