	\chapter{Conclusions} % (fold)
	\label{cha:radio_data_modelling_conclusions}

	\todo{
		\begin{itemize}
			\item Data modelling in the VO, necessary for
			interoperability.
			
			\item Existing VO data models, still incomplete in several
			particular models, specially in radio astronomy, but also
			for general astronomy and astrophysics.
			
			\item Our Provenance data model is a necessary addition for
			radio astronomy, and a useful addition for general
			astronomy.
		\end{itemize}
	}
	
%	El cuadrado de una suma corresponde al cuadrado del primer término,
%	más el doble producto del primer término por el segundo, más el
%	cuadrado del segundo:
%
%	\begin{align}
%	 (x+y)^2 &=  (x+y)(x+y)				\label{step1} \\
%	         &=  (x+y)x + (x+y)y		\label{step2} \\
%	         &=  x^2 + yx + xy + y^2	\label{step3} \\
%	         &=  x^2 + xy + xy + y^2	\label{step4} \\
%	         &=  x^2 + 2xy + y^2		\label{step5} 
%	\end{align}
%
%	Veamos cada una de estas transformaciones en detalle, para ver de
%	qué propiedades nos estamos valiendo en cada una:
%
%	\begin{description}
%		\item[Transformación \eqref{step1}]  el cuadrado de un elemento
%	          algebraico, es el producto de ese elemento por sí mismo.
%
%		\item[Transformación \eqref{step2}]  propiedad distributiva 
%			  del primer elemento sobre la suma del segundo.
%
%		\item[Transformación \eqref{step3}]  propiedad distributiva de
%		      los productos que quedaban.               
%                                          
%		\item[Transformación \eqref{step4}]  propiedad conmutativa para
%		      reordenar monomios.                   
%                                          
%		\item[Transformación \eqref{step5}]  agrupación de monomios
%		      semejantes.
%	\end{description}
	
	% chapter radio_data_modelling_conclusions (end)
