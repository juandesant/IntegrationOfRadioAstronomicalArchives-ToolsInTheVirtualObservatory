\chapter{MOVOIR Details} % (fold)
\label{cha:movoir_details}

	\section{SAMP MTypes description} % (fold)
	\label{sec:samp_mtypes}
		
		The MOVOIR messaging system/API is based on SAMP MTypes.
		The SAMP standard allows for the extensibility of the
		protocol by letting developers create custom MTypes for
		particular applications. An MType is described by:
		
		\begin{description}
			\item[MType string id] Administrative MTypes for the
			SAMP protocol start with \texttt{samp.}, and application
			defined MTypes need a non-null id starting with a
			different prefix. The custom, as recommended by section
			5.1 of the SAMP Proposed Recommendation, in order to
			make MTypes as generic as possible, is to use object (or
			data product) oriented prefixes, with a verb indicating
			the action, and clarification afterwards. Examples of
			this are \texttt{image.load.fits} or
			\texttt{table.load.votable}.
			
			 \item[List of named parameters] MType parameters need
			to be named, as SAMP messages use maps for parameters,
			to allow for easier mapping on the XML-RPC protocol, but
			also for extensibility. The list can be empty, if the
			message needs no parameters (for instance, for an
			action-triggering message).
			
			For each of the named parameters, and each of the named
			return values, the following information is needed:
			
			\begin{description}
				\item[name] As stated in the list of named
				parameters or named returned values.
				
				\item[data type] One of \texttt{map},
				\texttt{list} or \texttt{string}, as specified in
				section 3.3 of the SAMP
				IVOA Recommendation~\cite{2009samp.ivoav0904T}; for
				lists, appropriate scalar subtypes can be specified
				(section 3.4)
				
				\item[meaning] Parameter description in order to
				ensure that named parameters for the same message
				share their meaning.
				
				\item[is it optional?] If not stated, MType
				parameters are considered to be mandatory.
				
				\item[default value] Optional, and only for
				optional parameters.
			\end{description}
			
			
			 \item[List of named returned values] As the result of a
			MType message any number of returned values, each with
			its own name, can be returned. The list can be empty, as
			many messages return nothing. Irrespective of this, all
			MType calls should return a value. For each returned
			named parameter, its name, data type, meaning, optional
			character, and default value have to be specified.
			
			 \item[MType description] MTypes are a controlled
			vocabulary, and in order to increase interoperability,
			and the reuse of existing MTypes, a description of the
			MType is mandatory, in order to ensure all applications
			responding to the message behave in coherent way.
		\end{description}
		
		
	% section samp_mtypes (end)
	
	\section{MOVOIR MTypes} % (fold)
	\label{sec:movoir_mtypes}
	
		The MOVOIR API can be defined, then, by the collection of
		MTypes that it can respond to, together with the returned
		values.
		
		\mtypedef{movoir.conesearch.withPositions}
		{Queries ConeSearch services in the VO registry in order to
		retrieve data around the proposed list of objects and
		positions, at a particular angular distance radius. The
		result is a list of data result maps. If there is an error
		in resolving any of the names, \texttt{samp.status} is set
		to warning, and \texttt{samp.warning} is used to specify
		which ones where not resolved.}
		{
			\mtypeparam{objects}{list}{string}{List of object names
			or positions for which the ConeSearch services are to
			be queried. It must not be empty.}
			
			\mtypeparam{searchRadius}{float}{Radius (in degrees)
			around which the search is to be performed.}
			
			\mtypeparam[optional]{band}{list}{List of wavebands 
			the returned data must belong to. If empty, or non
			existing, there is no filtering of results by band.
			Default value: \emph{empty list}.}
			
			\mtypeparam[optional]{keywords}{list}{List of keywords
			that should appear in any part of the registry entry.
			If not empty, all of the keywords in the list appear
			in the returned value.
			Default value: \emph{empty list}.}
		}
		{
			\mtypeparam{coneSearchEntries}{map}{Map of all
			retrieved entries, using the service IVORN as a key.
			For each entry, a list of results is provided.}
		}
		
		movoir.conesearch.withPositions
		
		movoir.siapsearch.withPositions
		
		movoir.ssapsearch.withPositions
		
		movoir.dalsearch.withPositions
		
		movoir.dalservices
		
		movoir.dalservices.conesearch
		
		movoir.dalservices.siap
		
		movoir.dalservices.ssap
		
		movoir.arquery (for sending a query to the astroruntime,
		and have data back)

		
	% section movoir_mtypes (end)

	\section{MOVOIR Python scripting module} % (fold)
	\label{sec:movoir_python_scripting_module}
		\newcommand{\sampyurl}[0]
		{http://cosmos.iasf-milano.inaf.it/luigi/projects/vo/samp/}
		The MOVOIR Python scripting module uses the SAMPy
		module\urlnote{\sampyurl} by Luigi Paioro, but creates the
		following modules for abstracting the low-level SAMP
		functionality from scripts, providing direct access to the
		messages and results listed in the previous section:
		
		\begin{itemize}
			\item MOVOIR.arquery: for direct querying of the
			AstroRuntime.
			
			\item MOVOIR.conesearch: ConeSearch query module.
			Supports the MTypes above for allowing the query from
			different tools.
			
		\end{itemize}
	% section movoir_python_scripting_module (end)
	
	\section{MOVOIR based tools} % (fold)
	\label{sec:movoir_based_tools}
	
		Some tools have been developed based on SAMP and the
		MOVOIR in order to show different possibilities for
		the MOVOIR.
		
		\begin{description}
			\item[MOVOIR GUI] MOVOIR is the underlying 
		\end{description}
		[MOVOIR GUI]
	% section movoir_based_tools (end)
	
	\section{movoirDownloader MTypes} % (fold)
	\label{sec:vodownloader_registered_mtypes}
	
		The VO Downloader is a MOVOIR module which, apart from 
		registering its own set of MOVOIR-based MTypes, registers
		for notification of the following MTypes:
		
		\mtypedef{table.load.votable}
				{Download a table in VOTable format. In the case of
				the VO Downloader, causes the download of the
				provided file.}
				{\mtypeparam{url}{string}{URL of the VOTable
				document to download.}
				\mtypeparam[optional]{table-id}{string}{identifier
				which may be used to refer to the downloaded table
				in subsequent messages.}
				\mtypeparam[optional]{name}{string}{name which may
				be used to label the downloaded table in the
				application GUI.}
				\mtypeparam[optional, MOVOIR specific]{session-id}
				{string}{a string which can be attached to all
				downloaded data from the same session, so that they
				can be traced together.}
				}
				{\mtypeparamnone}
				
		\mtypedef{image.load.fits}
				{Download a two-dimensional FITS image.}
				{\mtypeparam{url}{string}{URL of the FITS image to
				download.}
				\mtypeparam[optional]{image-id}{string}{identifier
				which may be used to refer to the downloaded FITS
				image in subsequent messages.}
				\mtypeparam[optional]{name}{string}{name which may
				be used to label the downloaded FITS image in the
				application GUI.}
				\mtypeparam[optional, MOVOIR specific]{session-id}
				{string}{a string which can be attached to all
				downloaded data from the same session, so that they
				can be traced together.}
				}
				{\mtypeparamnone}
				
		It will also use the following generic MType:
		
		\mtypedef{url.download}
				{Download a generic URL.}
				{\mtypeparam{url}{string}{URL to download.}
				\mtypeparam[optional]{download-id}{string}{identifier
				which may be used to refer to the downloaded generic
				data in subsequent messages.}
				\mtypeparam[optional]{name}{string}{name which may
				be used to label the downloaded data in the
				application GUI.}
				\mtypeparam[optional, MOVOIR specific]{session-id}
				{string}{a string which can be attached to all
				downloaded data from the same session, so that they
				can be traced together.}
				}
				{\mtypeparamnone}
				
		Notice the addition of an optional, MOVOIR specific
		\texttt{session-id } parameter; the semantics remains the
		same for applications using the \texttt{table.load.votable}
		and \texttt{image.load.fits} MTypes, as they should ignore
		unknown parameters, as per the SAMP IVOA
		Recommendation~\cite{2009samp.ivoav0904T}.
	
	% section vodownloader_registered_mtypes (end)
	
	

% chapter movoir_details (end)